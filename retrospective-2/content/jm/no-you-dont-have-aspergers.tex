\articlehead{No, You Don't Have Asperger's}{JM}{2012}

There are a lot of furries who have Asperger's disorder, or at least a lot who think they have Asperger's disorder. As of May 2013, none of them will have it: it's being deleted from the Diagnostic and Statistical Manual of Mental Disorders. It's in DSM-IV; it will not be in DSM-5. (Roman numerals are out too, apparently.)

Along with three other conditions—autistic disorder, childhood disintegrative disorder, and `pervasive developmental disorder not otherwise specified'—Asperger's is being lumped into Autism Spectrum Disorder, or ASD. ASD reflects our understanding of autism as a universal aspect of personality rather than a specific condition. People with Asperger's will be recategorized as having mild ASD, or as having no disorder at all.

This is very good news for people with Asperger's, especially those with mild symptoms. (Anyone self-diagnosed with Asperger's is highly unlikely to have ASD, for reasons I'll explain later in this article.)

Autistic people fail to read social cues, and this leads to communication issues and inappropriate behaviour. In general, someone is classified as having ASD (as per DSM-5) if this condition is bad enough to be disabling.

Everyone, to some extent, has symptoms of autism: it's a natural outcome of how the brain works. Our species has evolved to have certain mental traits that support our social nature: we excel at facial recognition (to the extent that we might see faces in a grilled cheese sandwich, or in sand dunes on Mars); we are more empathetic towards fellow humans than towards other animals; we unconsciously negotiate sexual interest. Autistic people have poor social skills because these parts of brains are innately limited—a genetic throwback to a pre-evolved brain.

Autism is not an on/off condition, like having a broken leg. Some people have brains that are strongly socially-wired, perhaps actors or salesmen, others are less socially-wired and tend to be more logically minded, perhaps programmers or engineers. But everyone feels socially awkward, or out of place, or humiliated, from time to time.

Compared to society at large, furries are collectively further along the autistic spectrum. Symptoms of this might include our flair for technical work, such as IT and the sciences, and perhaps in our enjoyment of fursuits, which create a `deindividualized' social environment.

I see two causes that place furries further along the autistic spectrum:

Our demographics: we are young and overwhelmingly male. Put simply, men typically take longer than women to socially mature.
People with autistic traits might be attracted to furry:
Anyone who struggles to read social cues will feel disconnected from society, especially if they are going through puberty. People who are non-heteronormative or genderqueer -- and this somewhere around 70\% of furries (ref, ref) -- are more likely to feel alienated.

For young people who feel disconnected, it may be easier to identify with an anthropomorphic animal (as seen on TV or other media) than with other human beings. Growing up, furries may internalize this identification to the point that they start to see themselves as more like the animal-person and less human. This will be important to sexual development and may become a touchstone through puberty: the animal-person becomes an alter-ego that can safely experiment with new personality traits through introspection and roleplay—for example, alternative sexual or gender identities.

(I've explored the value of furry roleplay as an avenue to maturation previously, Growing Up.)

It's plausible that a sense of alienation when growing up is a strong contributor to our identity as furries. It helps that the concept of furry identity is open to interpretation, which means that we are free to explore personally useful aspects while discarding others. It also helps that the furry community is social and welcoming, filled with people with a similar internal world. Serendipitously, for many people, the social nature of the furry community provides a solution to the alienation that drew them to furry in the first place.

If the furry identity stems from this feeling of alienation, this offers an explanation for our unusual demographics:

\begin{itemize}
  \item Furry is largely male: men, on average, are less socially developed when they reach puberty.
  \item Furry is geeky: geeks, largely people with sharp logical minds, are often slower to develop socially.
  \item Furry is largely non-heterosexual: if you are sexually queer (or genderqueer), your social development can be more difficult.
\end{itemize}

In these three examples, furries may come to identify as an animal person as a way of unconsciously abnegating personal responsibility for social failure. It also explains why some furries might self-diagnose a social disorder: Asperger's.

It's common for people with Asperger's disorder to characterize themselves as feeling like a non-human, like an alien tourist in a strange society. It's easy to see why a young furry, who feels disconnected from the world and identifies as an animal-person, would find this compelling. Asperger's disorder is also fairly high-profile because it's relatable—mild autism is comparable to the less permanent condition of being a teenager—and also because of Mark Haddon's The Curious Incident of the Dog in the Night-Time, a novel with an apparently autistic narrator. The narrator, Christopher, is an easy character for any young adult to relate to in the Holden Caulfield sense: he's an outsider, confounded by his constant failure to act according to society's fluid and unsaid rules. It's an engaging read (although it flags badly in the second half as Haddon gamely tries to narrate action through Christopher's limited perception of the world).

Anyone identifying with Christopher from The Curious Incident is almost definitely not autistic. To identify is to demonstrate empathy, the very trait that Christopher—and anyone with Asperger's/ASD—lacks. The same logic can be applied more broadly: if you think you have autism, you almost certainly don't.

Autistic people are often unable to see themselves as part of society. Ironically, anyone who thinks that they don't fit in is demonstrating that they fit in well enough to be aware of society's norms. A feeling of alienation doesn't imply alienation. It's usually the opposite: a feeling of alienation implies that you are maturing and learning to assimilate.

This is the difference between being autistic and being a teenager: autistic people do not mature to the point that they can fully function within society. It's also worth considering that maturation continues until we are about 30 years old, and that the skills that help us feel part of society—empathetic skills—are the slowest to develop (ref).

So self-diagnosis of autistic disorders is usually wrong. It's also potentially damaging.

Labels are important things. If you believe you have Asperger's disorder, this means that you believe you will always struggle in many social situations. You believe that you cannot mature and improve beyond a certain point, because you believe you are innately limited. If you are younger than 30 (or so), this means that you are undermining your own ability to mature and develop these skills. In sociological and psychological circles this is known as a self-fulfilling prophecy, defined as `a false definition of a situation evoking a new behaviour which makes the original false conception come true' (ref).

The deletion of Asperger's from DSM-5 means that doctors can no longer diagnose autistic disorders without evidence of symptoms in early childhood (ref). It's easy to misdiagnose autism in an older child who is slow to socially mature. Psychologists have been long aware of the danger of such labels: a false diagnosis of Asperger's can harm someone who would otherwise mature a little later (possibly as a well-adjusted furry).

It's not just Asperger's. Among psychologists, there is growing awareness of the danger of labels. For example, a 1997 meta-study on child sex abuse concluded that many people have had positive sexual experiences when a child: consider the trope of the 14 year old boy who has sex with the babysitter. The study recommended that not all children be labelled (and treated) as victims of abuse, because doing so could retrospectively harm someone who would otherwise be fine. (The study, unfortunately, was ignored after it was formally condemned in the United ``Think of the Children'' States Congress, ref.)

Self-diagnosis of Asperger's is common because it's natural for a child, who is slow to develop socially, to define himself as different. All children feel that they are the centre of the universe. When an intelligent, analytical child looks around, it's clear that the outside world doesn't treat him as anything special. The inconsistency between his internal world and the external world creates conflict and a feeling of disconnection. This child may read about autism and falsely self-diagnose as having Asperger's. He would be much better off if he self-diagnosed as a furry, a label that encourages personal growth, as opposed to the self-limiting label of Asperger's.

How do you tell if you're autistic? Here's what the DSM-5 says:

\begin{quote}
  People with ASD tend to have communication deficits, such as responding inappropriately in conversations, misreading nonverbal interactions, or having difficulty building friendships appropriate to their age. In addition, people with ASD may be overly dependent on routines, highly sensitive to changes in their environment, or intensely focused on inappropriate items.
\end{quote}

It's impossible to self-diagnose. If you lack communication skills, you also lack the ability to assess the quality of your communication skills. People with ASD tend to be anosognosic, in that they are unable to perceive their disability (ref).

Your parents, or older siblings, are better placed to judge. They saw you grow up and will have noticed any symptoms in early childhood, which always occur in autistic people (by definition, ref). Otherwise, ask a doctor: they will use a simple written or verbal test to judge whether you have ASD. Everyone has autistic tendencies: it's a question of the level of impairment.

If you have self-diagnosed as having Asperger's, or if you were diagnosed when young, it may be time to reconsider. You may wish to think of yourself as logical and analytical, positive identity traits that allow room for you to learn and grow. Your analytical nature will help you learn new skills including improved empathy, if you apply your mind and approach the problem logically. You might begin by broaching the topic with similarly-minded furries.
