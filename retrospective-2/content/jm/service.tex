\articlehead{Service}{JM}{2013}

\textit{In which the author describes the meaning of life.}

We furries are a creative bunch. A large majority of us regularly engage in some sort of creative activity.

Looking at data from the 2012 furry survey:

\begin{itemize}
  \item 46\% of us are visual artists;
  \item 39\% of us are writers;
  \item 24\% of us are musical artists, and;
  \item 17\% of us are fursuiters.
\end{itemize}

Assuming that few of the responders are creating exclusively in non-furry spaces, this means that a large majority of furries are actively adding content to our community. (I'm doing my part right now, writing this article on a sunny winter's morning for [adjective][species].) Very few of us are pure consumers.

There is nothing wrong with being a consumer. All furries are consumers of furry art, from illustration to performance, because it is art that defines our culture. Without this furry-created art -— without furry-specific language, without drawn furry avatars, without fursuiters, without conventions -— we wouldn't be able to express our furriness. At risk of being a bit postmodern, the act of consumption gives meaning to the art.

This is what makes our community different from sci-fi fandoms and the like: fandoms are based around pre-existing art, whereas furry is not. If you're a brony, then you like My Little Pony, and the social aspects of MLP fandom are a nice bonus. Furries have no such common element, which makes our culture dynamic and exciting, if difficult to pin down.

(Here at [adjective][species], with a handful of different writers and over one hundred articles, we're mostly just exploring the question: what is furry? Or, maybe: what is a furry? Anyone with interesting ideas should visit our Contributing page and make their pitch to our fearless leader, Makyo.)

Furry creatives are collectively putting a lot of time and effort into the community. In the overwhelming majority of cases, that time and effort is donated with no expectation of monetary gain or even wide recognition.

To look at a few examples:

\begin{itemize}
  \item Kyell Gold and Phil Geusz, who are probably furry's two highest-profile and most successful writers (as well as being contributors to [adjective][species]). Both are doing well, but the time spent learning their craft and writing their first few books needed to be rewarding in its own right. They are the vanguard of an army of furry authors, most of whom are self-publishing in the morass of Sofurry, enjoying their time in front of the keyboard and appreciating the few readers they are able to find. Phil, who writes for [a][s] as Rabbit, has contributed many articles since joining us last year: he has done so for the same reasons as our anonymous Sofurry hordes, because he wants to contribute to the furry community.
  \item Potoroo, furry musician and friend of [a][s], runs a regular furry music podcast, Fuzzy Notes (fuzzynotes.podomatic.com). He collates and advocates music created by furries, for a furry audience. Music isn't really furry in the way that a drawing can be furry, at least outside of the likes of Kurrel the Raven (stop reading and go listen to his Commission Song right now if you haven't heard it). Yet Potoroo has an endless stream of quality music, by-furries for-furries. For our furry musicians, limiting their primary audience to furries comes at the benefit of contributing to the community as a whole.
  \item Oz Kangaroo is a fursuit performer, fursuit maker (www.crittercountry.com.au), and organiser of the Furry Down Under convention. Like Kyell and Phil, Oz didn't start with the expectation of success. His skills have been learned over time, and his early contributions to the furry community were largely anonymous, motivated by the enjoyment of the process.
\end{itemize}

Furry's creators—a group which, dear reader, probably includes you—are rarely motivated by personal gain. There is value in contributing to the greater whole, helping define the culture of the furry community. It's not selfless (and therefore not an act of charity), but it is immensely generous.

Furry, with its cartoon animals and imaginary worlds, is easy to interpret as a childish pastime. But this is missing the point: furry is about internal exploration of identity, something which niftily translates into an external world of intimate human friendships. This external world is something which we create ourselves. Our apparently childish pastime enriches our culture and informs our tight-knit community.

And it's the community aspects of furry that make it really special. A broad, interesting community isn't an easy thing to find in the twenty-first century. Today, people tend to exist in urban environments or online, where it is easy to find people who share niche interests. We group ourselves with people who are similar, and often define ourselves based on these delineations: we spend time with people of similar age, or education, or obsession, or professions, or even something as fundamentally unreal (if important) as money. Once you tar yourself with such an brush, it can be difficult to grow as a person, and explore the world outside of your niche.

Furry is a group without such limited horizons. Our community is the product of the things we create. Our decentralized nature means that our creations grow informally, like a meme: ideas become widespread as they are adopted by the separate, overlapping subgroups within furry. And even unsuccessful ideas have value: they are consumed and appreciated by an immediate audience of friends and similarly-minded furries. The furry culture is one that respects the act of creation, regardless of perceived quality.

We creators are serving the community. Our acts of service help build our world, and being a part of this communal effort provides meaning to life. Time spent in the act of creation, the act of service, can be internally rewarding and be appreciated by those we share our lives with. We contribute to our own environment, we build a culture that we can enjoy, and this adds to the feeling of inclusion within our community.

We furries are lucky. Other communities are often not as tolerant or welcoming, a point which leads me to a final story:

\begin{quotation}
  D. Michael Quinn is a Mormon historian. His research and publications put him at odds with the Mormon church leadership, who didn't accept his findings on past actions within the church that were at odds with the church's current moral stance. (Further reading here.)

  As a scholar with a belief in truth and evidence, Quinn continued to publish before his eventual exile and formal excommunication from the church. Stripped of his career, at age 63 he was reduced to sleeping on a sofabed in his mother's one-bedroom condominium.

  Just before his excommunication, Quinn was given an opportunity to testify before the church. He expressed his gratitude to the church for providing, throughout his life, a vehicle for service. The Mormon church, he said, drew him out of his largely monastic life and compelled him to help the men and women he saw every Sunday.
\end{quotation}

As far as I can tell, that's a life with meaning.
