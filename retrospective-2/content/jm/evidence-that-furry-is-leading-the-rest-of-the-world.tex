\articlehead{Evidence That Furry is Leading the Rest of the World}{JM}{2013}

\textit{The Stranger}, a well-regarded alternative weekly newspaper from Seattle, has just published their Queer Issue for 2013 to coincide with the Seattle Pride Parade.

There's a remarkable article titled Floating in Shades of Grey, written by Ray Van Fox, which talks about the furry experience. Except that Ray isn't a furry -- his vulpine nom de plume is coincidental -- and his article doesn't reference furry. Ray talks about a largely online community, where he and the other members ``are re-creating ourselves in our own image in order to be seen for who we actually are.''

Ray is talking about the well-worn furry ground, where we present as animal-people in furry spaces. This is usually online for furries, but also real-world spaces like a private party or a convention. We choose an identity for ourselves, one which matches our internal perceptions, and we interact as if that identity were true. It is true, in its own way, but it's different from the arbitrary real-life meatbag with which we are burdened.

This duality, our arbitrary meatbag versus our created identity, is at the centre of the furry experience. Our two faces can be very different: different species, gender, sexual preference, size, colour, personality. (Many furries précis their new identity by naming it using the convention [adjective][species].)

We furries may be the first large group to collectively choose to socialize using identities of our own original invention. We're certainly experimenting with the limits of identity in a way that no other group (of comparable size) is. Our genesis came when furries moved away from the first wave of `furry fandom', made up of sci-fi fans who liked anthropomorphic characters, and towards today's second wave of furry as an identity. It's no coincidence that this change coincided with mainstream adoption of the internet: the online world allowed us to form our community.

The rest of the world is following in our footsteps. People are learning that the identity through which they socialize can be a different, truer one than their arbitrary meatbag. Second Life is an obvious example -- there are plenty of furries but there are also plenty of other people keen to explore the freedom of an identity that can reflect their true, internal nature.

Grindr is another environment where people project a different version of themselves. A Grindr identity might be a hypersexual version; maybe a little hornier, a little better-endowed, a little younger. And the social environment in which Grindr users meet is (presumably) a space where those identities are `real', and where mundane aspects of life don't intrude.

It happens on Facebook, too. Users show a version of themselves that reflects their true interests and identity. For example, someone who is a young parent can choose to use their baby as an icon. This allows them to reframe the dependent relationship as one of equals, just at different stages in life. The adult is a former baby, and the baby is a future adult. (Similar things can happen later in life, when the child becomes the carer for a dependent elderly parent.) Something similar can happen with pet owners too, where the human can reframe their own life in the context of the love and luxury they are able to afford their domestic animal. In both cases, the Facebooker is expressing that their care provides a sense of internal wellbeing, something important and worthy.

It also happens in queer communities. Ray Van Fox is genderqueer, and his community has grown on Tumblr. He has found a community where he can express his true identity:

\begin{quote}
  Somehow these folks, without even knowing some supposedly basic things about me, have created a safe space where I can be my most authentic, uncensored, almost fully ungendered self.
\end{quote}

(A quick note on pronouns: I've chosen to use `he' for Ray, because that's how he mostly presents himself out in the real world. It's not perfect but I think it's better than using a gender-neutral neologism, which I find to be jarring. Neither option is perfect, so I've chosen what is least-worst, at least from my perspective.)

Ray's description of his `safe space' sounds a lot like the furry spaces in which I spend much of my time.

Here is his description of his safe space:

\begin{quote}
  Lots of us have names and personas and pronouns that are different from the ones we have in ``real'' life, but we aren't using them in order to deceive anyone.
\end{quote}

And:

\begin{quotation}
  I'm exhausted with all the tiny lies and self-betrayals involved in trying to squeeze myself into an identity that isn't quite mine. Why would I leave the house and deal with that, when I can get online and interact with others without having to package myself in any shape but the one I've got?

  Tumblr provides a level of anonymity in the act of self-creation  --  of constructing my blog persona  --  that gives me freedom from others' preconceived notions based on my body. Because it's all about what you say, not how you look.
\end{quotation}

In all these examples -- on Second Life, on Grindr, on Facebook, on Tumblr -- groups of people are taking advantage of the online world to experiment with identity in the way that furries do, and have been for the last 20 years or so. We're not exactly leaders -- people aren't walking around with WWFD bracelets or consulting the latest advice from furry think tanks -- but we are the first to cross this new ground. And so we can expect that non-furry groups will experience the positive and negative aspects of our furry experience as time flows on.

I think that fellowship is the biggest gift the furry community has given to us. We are able to be ourselves and be treated with respect, in a way that many of us cannot easily find in non-furry spaces. I think that this change is already affecting mainstream culture: as more people learn the value of self-expression, those on the fringe are finding more acceptance. As examples: there has been a seachange in attitudes towards gay people; there are signs that the world is starting to move beyond gender binaries (although there is a long way to go); an inclusive, intelligent third wave of feminism is gaining traction.

People in these three cases (gay people, trans* people, women) are all on the fringe, and are exploring aspects of identity. Members of all three are having to make compromises in the way they present themselves in society, something which they are not required to do in the `safe space' of their respective communities. Some will refuse to compromise (to their own detriment -- they will be given the perjorative label `militant'), and some will not explore their true identity. But the majority will balance two identities, internal and external, and they will have to deal with the challenges this presents.

There has been a lot of talk here on [a][s] recently about how we, as furries, manage our internal animal-person identity with the need to conform to society's expectations. I won't cover that ground again here. Suffice to say that compromise is necessary, and that there is no perfect solution.

The requirement to balance a true internal identity with a curated external identity is challenging. It can require vigilance, especially if we want to keep ourselves googleproof. There are techniques and tools, however they are yet to reach maturity (Google Plus looked promising before they decided that we wouldn't be allowed to socialize under an invented identity). But the tools will improve as the mainstream world catches up with the furry community, as people learn the freedom and happiness that a safe space and self-consistent identity can bring.

As Ray Van Fox puts it:

\begin{quote}
  That space may be made up of a bunch of ``strangers'' who might look different than I imagine, but I can bank on the fact that their reasons for befriending me have nothing to do with my body. And I can't tell you how comforting that is.
\end{quote}

You can read Ray's full article here. He lives at http://www.rayvanfox.com/.
