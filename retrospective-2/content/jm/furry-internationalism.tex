\articlehead{Furry Internationalism}{JM}{2013}

Furry is a genuinely international phenomenon. There was a time when furry might have been accused of being an English-speaking Caucasian phenomenon, but those times are long gone.

Here at [adjective][species] Zik has put together a review of Furry Cons of the World, which remains the best single demonstration of furry's worldwide spread. Despite its length, it's not a comprehensive list, with some oversights and some new cons appearing in the year since Zik's article was published.

One new con is based in the city where I grew up: Perth, Australia. FurWAG will be held at the Rendezvous Studio Hotel on October 4-6, 2013 (www.furwag.com.au/), a con that has some claim to being the first in South-East Asia.

Perhaps that's a bit of a stretch: Australia is hardly most people's idea of an Asian nation. However Perth is geographically closer to Indonesia than any other Australian city, and is in the same time zone as Singapore, Hong Kong, Tokyo, and Beijing. For those living in South-East Asian nations, loosely collected under the AnthroAsia banner (www.anthroasia.com/), I expect that FurWAG will become a regular pilgrimage.

I know quite a few of the AnthroAsia folk, and I can think of no better example of furry's spread into culturally diverse parts of the world. Furs from the region tend to travel to Singapore for local meetups, recently including a visit by Japanese furries.

It's a similar story in South America and in Europe, home of the standard-bearer for furry internationalism, Eurofurence. The 18th Eurofurence, last year, saw people travel from 34 different nations, a record I'm sure they will handily beat in 2013.

Furry is an international community that pays little notice to the borders that divide us. Our community is participating in one of the great upheavals of the 21st century, an upheaval in its early stages: the loss of true national identity.

Country borders are becoming arbitrary. Nobody thinks that there is anything fundamentally different between two people born one mile either side of the USA-Canada border. Few people raised in Alsace will feel either completely French or completely German. The idea that each person `belongs' to a certain country, that it is a fundamental part of their identity, is becoming an antiquated notion. People who identify themselves by their American state or British county already seem old-fashioned or quaint; statements of national identity will become similarly anachronistic as the world becomes more interconnected.

A loss of national identity is, I think, a good thing. We become less prone to broadly stereotyping foreign groups, generalizations that can be insulting, reductive, or racist. Our natural suspicion towards people who are different -- outsiders -- is tempered, and we do a better job at treating our fellow human beings with humanity.

On the downside, some ancient cultures will be lost in this upheaval. Some of those at risk include marginalized racial groups, such as the indigenous cultures of North America and Australia; other cultures at risk include those that are fundamentally isolationist, such as religious absolutists, perhaps most obviously in the United States and the Middle East.

Those people fighting for the safeguarding of indigenous cultures, and those people fighting to protect their religious culture, are often from the opposite sides of the political fence. However their goal in this regard is the same: that some things must be protected, that these cultures represent something important that should not be lost.

It's a worthy fight but probably an unwinnable one: the world is inexorably becoming more internationalist. To borrow a phrase from the Australian culture of my childhood: the pooch may already be rooted. But while some cultures are being lost, new cultures are appearing.

For example: furry culture.

New cultures will appear as our world becomes more internationalist. These new cultures will be less based around physical proximity (and often rejection of outsiders), but based on shared interests. Technological tools such as the internet are making us closer, so we can seek out -- and contribute to -- those cultures we find appealing. Furries, for example, have taken the idea of anthropomorphism -- an idea that has existed for a long time, arguably hundreds of centuries (as explored here in [adjective][species]), and taken it in unpredictable directions.

There are many other new cultures appearing, although not all will mature as quickly as furry. We furries are, after all, a community of early adopters.

The emergence of new cultures is one of the most exciting things about the world at the moment. I'm not going to go quite so far as to call this a Golden Age, but I believe that today's worldwide cultural change is a positive one.

In general, I think that the world is clearly changing for the better. Some will disagree with that assertion. I support my argument with a single quote:

“Global life expectancy at birth, which is estimated to have risen from 47 years in 1950-1955 to 65 years in 2000-2005, is expected to keep on rising to reach 75 years in 2045-2050.” (quote from UN World Population Prospects: The 2004 Revision)

The internationalist drive is, I think, a great positive influence on our world. As the world becomes more internationalist, with furries (perhaps incongruously) leading the way, historical cultural rivalries will slowly fade towards irrelevance. A more connected world is a more peaceful one.

If that sounds like a bit of a stretch, consider a hypothetical war starting tomorrow. There will be furries on both sides of the conflict, and some who become caught in the middle. We will know about this because we're interacting on Twitter, or Fur Affinity, or Weasyl, and wherever else furries socialize.

Our sympathies will be for the affected furries, regardless of whether they are a combatant or innocent victim. War in the mid-20th century was more a matter of choosing sides; this is no longer the case.

Umberto Eco, an Italian author and intellectual, made a related argument in a 1991 essay titled Reflections on War (published in this collection). Eco argues that internationalism has made traditional war `impossible'. He notes that a nation cannot build an absolute consensus inside its own borders, and that the cause-and-effect of any war becomes obfuscated to the point that outcomes are impossible to predict. Reflections on War predicts many of the unforeseen outcomes of America's war with Iraq: a war that, whatever your feelings about it, didn't achieve the intended outcome in the intended fashion.

Furry culture pays little regard to international boundaries, an example of Eco's assertion that nations cannot be considered self-contained units. Furries love and respect one another regardless of nationality or culture. We're not blind to our differences, however these are trumped by our similarities, such as the shared furry culture. It's safe to assume that Eco didn't envisage the furry community as a great example of how `hot' war is impossible, but our internationalist culture makes us exactly that.

A second example (and those averse to sports metaphors should skip ahead two paragraphs): the Indian Premier League, a newish cricket tournament. Before the IPL, cricket was played based on geography: state vs state, or country vs country. The IPL is a club competition that works more like soccer: players are sourced from around the globe.

In the blink of an eye -- where nationality, religion, or culture doesn't define the players of an IPL team -- people have learned the pointlessness of formerly bitter and hateful rivalries. White South Africans cheer black West Indians; Hindus cheer Tamils and, most notably; Indians cheer Pakistanis (although Pakistani players are technically unavailable for the IPL, there are lots of Pakistani expats: Imran Tahir, Owais Shah, Azhar Mahmood, etc).

Internationalism brings people together and makes historical flashpoints look petty and trivial. Furries are coming together in a way unimaginable a generation ago. We are exposed to a range of races and cultural backgrounds, helping us moderate any persisting short-sighted beliefs fostered by parochialism.

It's more than that too: our internationalist furry culture means that we are good rolemodels for new furries, and we provide support to the vulnerable. Consider my favourite generic example of someone unlikely to fit into his local culture: a young gay furry from a rural American community. Our young fur will meet other gay men who are happy and comfortable with their sexuality, and he will take solace that he is not so isolated after all.

The furry community opens up options, to all of us. Our internationalism offers experience we can draw upon and learn from. And we're providing the same service to others when we interact with furries from outside our own culture.

We're spreading our internationalist culture, one that comes with peace and respect as standard.
