\articlehead{My Little MLP Adventure: Prologue}{JM}{2013}

\textit{My Little Pony} has become a visible part of the furry community in the last few years. Since 2010, when the TV series was rebooted by Hasbro and Lauren Faust, ponies are everywhere. It’s not just that they’re easy to draw (although I’m sure that helps), they are popular to the point of ubiquitousness online and at conventions. They have become an important feature of furry’s cultural wallpaper. And they are, of course, anthropomorphic too.

And yet \textit{MLP} is clearly a children’s cartoon.

So why is it loved by so many intelligent and thoughtful furries? Why has \textit{MLP} joined the likes of The Lion King as a furry touchstone?

I’m going to try to find out.

Fortunately I am friends with one of the great furry pony-lovers, a UK furry called Artax. Artax is one of the founders and an administrator of a popular \textit{MLP} forum at www.canterlot.com (4,900 members, 286,000 posts), as well as being a longtime pony geek. I asked him how a pony-sceptic should make a virgin approach to all thing M, L, and equine.

Artax is my Dr Pony, and he has prescribed me an \textit{MLP} marathon: we’re going to sit down and watch as much \textit{MLP} as I can stand, starting with Season One Episode One.

I am filled with curiosity and terror.

The ponies, you see, are a bit personal for me. As a horse furry, I identify with my equine self as a source of quiet personal strength, physical and mental. The horse is the foundation of my identity; it’s what makes me a furry. Popular perception of the ponies undermines the formerly staid horse archetype. People find out that I’m a horse and they don’t think I’m an impressive equine: they ask about my cutie mark.

Now you might find this all to be hilarious. But it’s taken some adjustment on my part. It’s as if I met a dragon furry, and I went on to imagine a pastel Baby Yoshi. This would be at odds with the intimidating expression of outsiderhood presumably intended by my fantastical friend.

Even before the new series of \textit{MLP} starting taking over my world, furry friends would poke fun at me by harking back to the ponies of the 1980s. They did so because they, correctly, sensed that it contradicted my relationship with the horse. And, in the manner of friends sensing a good-natured but genuine weakness, proceeded to satirize me as a pony at every opportunity. I had no choice but to grin and bear it.

My tormentor-in-chief has predictably become a modern-day pony-lover. She was all too happy to draw JM as a pony for this article (you can see more of Rainbird’s pony art here):

\begin{figure}
  \begin{center}
    \includegraphics{content/assets/mlp--ponified-jm}
  \end{center}
  \caption{Pony JM. According to Rainbird, Ponified JM stands for ‘Juicy Mac’, a bizarre moniker inspired either by a type of apple or the fact that I was an early iPod adopter. I do not endorse this name.}
\end{figure}

\textit{MLP} is a children’s TV show, and I rarely choose to consume media created for children.

My cultural interests generally veer towards the highbrow: I subscribe to literary magazines; I read good quality fiction; I sometimes watch ponderous European cinema. And I know that this can make me seem hopelessly pretentious.

But I’m not snobbish about it. I don’t think less of people who prefer their media to be lowbrow, be that \textit{Harry Potter} or \textit{Transformers}. If anything, I’m worried about being subject to a kind of reverse snobbery, where I might be made to feel ashamed for my interests. To quote Thomas Pynchon: ``Except for maybe Brainy Smurf, it’s hard to imagine anybody these days wanting to be called a literary intellectual.''

I’m not going to apologise for choosing to consume media created for adults. I’m equally won’t suggest that there is anything wrong with consumption of media created for children. My preference is personal, and I don’t think that people who love the lowbrow are any lesser in intelligence, or any other supposed measure of the value of a human being.

I don’t have any specific objection to animation or children’s TV, except to say that I often find it, well, childish. There is a scene in \textit{Life Of Pi} (the novel, I haven’t seen the film) where our castaway, in desperate hunger, tries to eat his own faeces. His plight is such that he doesn’t register the taste, he simply learns that it contains no sustenance. I feel much the same way about children’s TV, from \textit{Barney the Dinosaur} through to \textit{Family Guy}.

And so the prospect of a day dedicated to \textit{My Little Pony} fills me with terror. I understand that the show is set in a pony-only universe, and that the characters (who have names like Rainbow Dash) go about and have adventures. Everything is going to be colourful and high-contrast and jolly, which sounds to me like a kind of longform Nyan Cat.

Fear.

But I am curious too, and that curiosity comes from the rather amazing culture that has sprung up around \textit{MLP}. Artax, like many of my pony-loving friends, is an intelligent and grounded guy. I trust that his love for the show must come from something more worthwhile and nuanced than its physical aesthetics.

Perhaps there is an undercurrent of Ghibli-style magic, where an emotional thread lurks below the fantastical creations? Maybe there is a clue in the show’s subtitle, and that the show explores how friendship creates something special, magical about life? (Please, please please please, don’t tell me that \textit{actual} friendship in the pony universe is \textit{actually} magic. Please please.) I’m all for well-told morality tales that reinforce the value of friendship, one of life’s true joys.

There are some obviously positive sides to the show too. It’s rather excellent that the main characters are exclusively female (or close to it), and that they are embraced by a diverse audience. It makes for a refreshing change, especially given that popular cartoon shows among furries are often contemptible bro-fests, where women are treated as if they are an alien species (sometimes literally). \textit{MLP} is a breath of fresh air among the likes of \textit{Adventure Time}, \textit{Regular Show}, \textit{Spongebob Squarepants}, \textit{Aqua Teen Hunger Force}, \textit{Phineas \& Ferb}, et al ad nauseum.

Also, I love the pony fandom’s embrace of the so-bad-it’s-good neologism ‘brony’. Brony is a great term, enthusiastically ludicrous, and neatly co-opts a masculine word base to become at least partly gender-neutral. It’s a miracle born of the unlikely coupling of 4chan-style snarkiness and political correctness.

When I ask my brony friends about \textit{MLP}, I get a range of responses. Some feel that it’s genuinely great TV, some seem to enjoy it as a guilty pleasure, others have referred to it as audio-visual valium. This makes it sound vaguely like the glacial Koyaanisqatsi, a film that—at least in concept—seems like the artistic polar opposite of \textit{MLP} (except perhaps in the ‘stoner classic’ category). Whatever the truth, such descriptions have piqued my curiosity.

So I’m approaching my ponyfest with an open mind, without setting my expectations too high. Artax has suggested that I prepare by bringing a stuffed animal and a very large quantity of vodka. So I should end up with a warm glow one way or the other. I’ve even infused my vodka with beetroot to give it an appropriately pink hue.

Wish me luck. I shall report my results next week.
