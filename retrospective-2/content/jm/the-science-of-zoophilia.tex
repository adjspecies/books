\articlehead{The Science of Zoophilia}{JM}{2013}

Scientific research on human sexuality is a relatively new field. The Kinsey Reports, published in 1948 (men) and 1953 (women) (link), were the first attempt to gather data on human sexual behaviour. These were informally updated by Playboy in the 1970s (link), back when it retained some literary relevance, in an attempt to understand the changes brought about by the sexual revolution, and -- of course -- to provide some salacious reading material.

It took until the early 1980s for researchers to confirm that homosexuality is largely set at birth (ref). This work, controversial at the time, contradicted the prevailing wisdom that male homosexuality came about due to feminization of a male child, caused by an overbearing mother and distant father (the reverse supposedly applied for lesbians). This conclusion was simple enough to make: researchers interviewed a large number of people, asking about their childhood and sexual preference, then looked for correlations. (They found none.) And yet such simple data gathering took more than 30 years after Kinsey to be published.

The science of zoophilia is much less mature. Kinsey asked questions and gathered data (as did Playboy) however the first serious attempt to understand zoophilia was published more than 50 years later, by Dr Hani Milestki in 1999. Miletski's book suggested that zoophilia may be a legitimate sexual preference: one defined by love, not sex.

Miletski's book was followed by research from two long-time specialists in the field—Drs Williams (wikipedia profile) \& Weinberg (wikipedia profile), who started their careers studying homosexuality in the 1960s. (Weinberg, literally, wrote the book—in 1981—that showed that homosexuality was set at birth.) The two are highly respected in their field, and their results agreed with Miletski when they published in 2003. Research into zoophilia has increased since then.

(The works of Miletski -- \textit{Understanding Bestiality and Zoophilia} -- and Williams \& Weinberg -- \textit{Zoophilia in Men} -- are \sout{available in full for free, and are} barnstorming reads. They are both recommended, Miletski in particular, although you'll want to gird your loins for some vivid language.)

The two works are notable for going beyond analysis and discussion of statistics: the authors clearly became sympathetic towards the zoophiles during the course of their research. This sympathy isn't evident in the results, but it is evident in their discussion of the zoophile lifestyle. They note that the zoophiles face unusual personal and ethical challenges as a result of their taboo sexuality. Williams \& Weinberg make a direct comparison with the subjects of their early work, homosexual groups in a less tolerant era:

\begin{quote}
  They reminded us of some of the early gay groups we studied in the 1960s and 1970s, especially when they engaged in banter about sex (in this case, it was not just sex with men).
\end{quote}

Homosexuals in that era were seen as dangerous sexual deviants, similar to the way that zoophiles are seen today. However it is very clear from the results presented in Miletski and Williams \& Weinberg that the relationship between a zoophile and his/her animal partner is based on love, where sex is an expression of that love.

This brings about a special problem faced by zoophiles: if you are in love with a non-human animal, where do you find human contact?

This is clearly a significant personal challenge for the zoophiles, especially given that they must hide their taboo sexuality from most people. Many zoophiles displayed a tendency to anthropomorphize their animals (ref Williams \& Weinberg):

\begin{quote}
  When asked “Is being in love with an animal different than with a human?” approximately three quarters answered positively. The features the men mentioned were anthropomorphic in that they described ideal human love relationships. Ironically, humans were often seen as less able than animals to provide those ideal human characteristics.
\end{quote}

This is a special kind of misanthopy, one where human emotions are projected upon an animal to create an ideal that cannot be met by a real human. This false creation of a perfect, or near-perfect, oxymoronical hyper-human non-human is only going make it more difficult for a zoophile to find real human contact.

Humans are social beings. We have evolved to need one another's company, and we communicate in subtle ways that meet our social needs. A relationship between a human and a non-human will always be one-sided, regardless of the perception of mutual love.

The zoophiles can end up with an unhealthy misanthropic perspective, a perspective I would compare with that felt by depressed people:

\begin{quotation}
  I find the company of animals more pleasing than that of humans – there's less stress, fighting… Love with an animal is how love should be – a lot less complicated with no strings attached. (Williams \& Weinberg)

  I can identify with dogs a lot more than I can identify with humans. I am thinking a lot like dogs, and therefore I can understand dogs better than humans. (Miletski)

  I felt I could only trust animals. They didn't gossip, they didn't laugh at me, they were available most any time. (Miletski)
\end{quotation}

In these comments you can hear the reflected neuroses of the zoophiles. They feel that humans cannot possibly live up to their expectations, or that they themselves will fail to `fit in' with society, so they regress and find reasons to avoid people altogether.

Marcel Proust, as ever, intuited this, framing depressive misanthropy as a reaction to a need to be part of (an untrusted) society. The following quote is from the second volume of \textit{In Search Of Lost Time}:

\begin{quote}
  In a recluse, the most irrevocable, lifelong rejection of the world often has as its basis an uncontrolled passion for the crowd, of such force that, finding when he does go out that he cannot win the admiration of a concierge, passers-by or even the coachman halted at the corner, he prefers to spend his life out of their sight, and gives up all activities which would make it necessary to leave the house.
\end{quote}

The sad irony is that those who have the least social contact are the ones most in need of social contact.

The other issue is, of course, the ethics of sexual contact with a non-human animal. Animal sexual abuse can sometimes be a problem (ref), and such behaviour is commonly assumed to be the act of a zoophile.

According to the researchers, making a connection between zoophilia and animal abuse is wrong. The zoophiles were defined by their love for the animals. Miletski states:

\begin{quote}
  The majority of my subjects love their animal-partner. Some see them as a spouse and will do anything for them. Sexual relations with the animal is an expression of love for them, and if the animal tells them, with its body language, that it is not in the mood for love-making, the majority of my subjects will leave the animal alone. In fact, many of them are members of the Humane Society and other organizations that are taking care of animals.
\end{quote}

The ethical issues associated with zoophilia are important however I don't intend to explore them in detail here. This is difficult ground because of the strong moral reaction people often have to zoophilic acts (very comparable to the strong moral reaction some people have to homosexual acts). In general, researchers and ethicists on the topic (notably Peter Singer, author of \textit{Animal Liberation}, ref) agree that the issue is whether the animal is harmed, and that the sexual aspects are irrelevant. This may be the subject of a future article (although, given my recent article on Why Zoophilia is a Furry Issue I'm a bit concerned about turning [a][s] into \textit{Zoophilia Weekly}).

There is a small zoophile subculture growing on the internet. Zoophiles are expected to continue to congregate online, due to their small numbers and the benefits of anonymity. Many of them, including some who participated in Miletski's study, have engaged with the furry community. This makes sense: furry provides what is important to zoophiles, namely a largely online-based culture with strong social connections, an emphasis on intimate friendships, and a safe environment for people of unusual sexual or gender orientations.

It's no accident that the researchers compare zoophiles today with the GLBT community of 50 years ago. Gay relationships were seen as an exercise in immoral sexual behaviour, however this has changed as homosexual relationships are now largely perceived to be about love. The zoophiles have not reached this stage, but they may find that the furry community provides a social environment where their love is tolerated as unusual but acceptable. If zoophiles can be open within furry, they can provide good role models for the `zoo-curious', helping young people manage and accept an otherwise complex and difficult sexual orientation.

I expect the level of conversation within the furry community to improve over time as the number (although perhaps not the proportion) of zoophiles increases. We will see more intelligent, respected, well-adjusted zoophiles be open about their orientation within furry. Dismissive and offensive language will become marginalized, just like homophobic language has declined in general society in recent years. And conversation topics, online and offline, will move away from the presumption of abuse, and towards the real ethical and emotional challenges of being a zoophile.
