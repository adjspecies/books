\articlehead{Furry Reseaerch: A Look Back at Dr Gerbasi's Landmark 2007 Study}{JM}{2013}

The first notable academic study on furries is six years old. Completed in 2007 (published 2008), Gerbasi et al's Furries from A to Z (Anthropomorphism to Zoomorphism) provides a review of furries based on 246 responses (including 217 furries) to surveys distributed at Anthrocon, plus an ad hoc `control group' of 65 psychology students.

The study had two main goals: to test the validity of the usual furry stereotypes, and to investigate whether furries exhibit signs of personality disorder.

Gerbasi presented data to show that furries are an unusual demographic (anthropomorphic \& zoomorphic interests; male dominated; wide range of sexual orientation), and that the group doesn't exhibit any special tendency for known personality disorders. Beyond that, there was one strong conclusion: that up to 46\% of furries ``may possibly represent a condition we have tentatively dubbed `Species Identity Disorder'''.

The diagnosis of Species Identity Disorder, a term invented by Gerbasi, is defined by her as ``\ldots considering the self as less than 100\% human and wanting to be 0\% human [and] is often accompanied by discomfort with their human body and feeling that they are another species trapped in a human body''. Gerbasi makes a direct comparison to Gender Identity Disorder.

There are some problems with this.

The most obvious problem is the use of the word `disorder'. This implies that there is some sort of problem. Gerbasi seems to be pathologizing furry, or at least a large subset of furry.

Psychologists understand that people have all sorts of different perspectives on the world, and a wide range of personality traits. An unusual trait is not a problem in itself. The word `disorder' generally means that a condition is bad enough to be disabling.

Gerbasi's sample of 217 furries are all people who could manage the cost, transport, and social effort required to attend a large convention like Anthrocon. A large subset of these people cannot be mentally disabled: if they were, they simply wouldn't have been there.

For comparison, the 2011 Furrypoll, which was completed online by over 4000 furries, showed that about 11\% of furries consider themselves either non-human or part-human. This is a long way from Gerbasi's 46\%.

Gerbasi's unreasonably large number is probably an issue related to the slight unreality of a convention environment. This argument is made rather pithily in a paper by Dr Fiona Probyn-Rapsey, who disagrees with Gerbasi: ``There are a myriad of reasons why furry participants at a furry conference might identify as ``less than 100\% human,'' not the least having a hangover from furry drinks the night before.''

Probyn-Rapsey's argument is laid out in her counterpoint, Furries and the Limits of Species Identity Disorder: A Response to Gerbasi et al, published in 2011 in the same journal as Gerbasi's original paper (ref). Dr Probyn-Rapsey challenges Gerbasi's tentative diagnosis of `Species Identity Disorder' directly: ``What might be the “treatment” for such a condition?''

Probyn-Raspey's biggest problem is Gerbasi's link between `Species Identity Disorder' and Gender Identity Disorder. Probyn-Rapsey points out that a diagnosis of Gender Identity Disorder is a controversial and politicized one, and that many people regard it a misrepresentation of people on the transgender spectrum (much in the way that homosexuality was formally considered to be a mental disorder in mainstream psychology up until the late 20th century). Gerbasi avoids any such discussion, simply referring to Gender Identity Disorder as if it were objectively diagnosable.

It's ironic that the mental health of furries is defended Dr Probyn-Rapsey, a feminism theorist. Furry is not a progressive environment for women nor for feminist ideas. We remain significantly informed by moronic (if well-meaning) advocates for `men's rights', probably because of our crossover with the echo chamber of male-dominated online spaces such as Reddit. It's a pity, because feminism and queer theory provides a useful foundation for analysis of our community. However this is all a larger topic, perhaps worthy of a dedicated [adjective][species] article or three.

Gerbasi, for her part, doesn't actually question the mental health of furries or suggest that there a significant subset of us that require treatment. This is a criticism drawn only from her use of the word `disorder' and her link between so-called `Species Identity Disorder' and Gender Identity Disorder.

It feels to me that Gerbasi has chosen to introduce `Species Identity Disorder' because she was hoping to be the first to identify a new psychological phenomenon. It's a professional coup to be a leader in any field, and I suspect that Gerbasi simply over-reached in her language. She is certainly a leading furry researcher and her instinct -- that something special is going on inside our community -- is, I think, spot on.

Her article was the first, and to date only, publication of the International Anthropomorphic Research Project, which Gerbasi heads. The IARP is a grand title for three researchers operating from a small community college. And calling it `International' is bit bullish seeing as it's based on the fact that they have scientists from the United States and Canada (it feels equivalent to a collaboration between people from Brighton and Cardiff). However, ornate naming aside, their research is of great value to the furry community.

The IARP are continually collecting data during regular forays to American furry conventions and online. They are strongly engaged with, and legitimized by, the furry community: their research is touched by the gilded hand of Anthrocon's Sam Conway (he appears as a co-author in their paper), and they include Laurence Parry (Flayrah head honcho and founder of Wikifur) on an advisory board.

Perhaps most significantly, the IARP include a furry in the their research team: Courtney Plante, otherwise known as Nuka. Plante joined their group in 2011 and is presumably on the way to earning the first ever PhD in furry studies. (We are lucky to have another prospective furry PhD here at [adjective][species], Quentin Julien, who joined us as an occasional contributor earlier this year.)

The IARP regularly publishes data from their surveys, some of which I have discussed in previous articles here at [adjective][species] (link). Their methodology is intelligent and elegant. Most recently they have kicked off a longitudinal study, where they will be following furries over a significant period of time. I expect their study will dig up some interesting data, showing how we mature as members of the furry community.

You can visit the IARP homepage, browse their results, and see the full text of their paper at https://sites.google.com/site/anthropomorphicresearch/home.

Gerbasi has tilled the ground upon which a field of furry research is starting to grow. I've spent the past few days at the British Library reading up on the latest furry research and much of it is fascinating. It's difficult to imagine this research existing without Gerbasi's willingness to engage with the attendees of Anthrocon, and her direct exploration of furry psychology and popularly-held stereotypes.

The IARP dataset from 2007 is no longer considered to be particularly large or useful. Of all the available datasets, today's researchers are most likely to use Klisoura's Furrypoll (hosted here on [adjective][species]), for example in this Spanish study from 2013. However the focus of the IARP in recent years is more focussed: geared towards understanding furry psychology, rather than simply furry demographics. I'm fascinated to see what they will learn next.
