\articlehead{Informal Networks}{JM}{2012}

Being furry can help your employment prospects. The connectedness of furries means that we are exposed to new people and new opportunities. This is true for any social network but it's especially true for the furry world.

The business world is not a level playing field. Ideally, people would be employed and promoted based on competence, but this is not always the case. Some people have unfair advantages: the boss's son; the alum from an elite university; the captain of the football team. Other people may unfairly suffer, such as those from disadvantaged backgrounds or anyone perceived to `not fit in'.

Unfair advantages are often examples of the adage ``it's not what you know, it's who you know''. In the management consulting world, these are known as informal networks.

Famously effective informal networks include Skull \& Bones, Yale University's secret undergraduate society, with members including George H.W. and George W. Bush, and 2004 Democratic challenger John Kerry among many others. While the members of Skull \& Bones are an intelligent group, undoubtedly many of their number have been helped by having friends in high places.

The value of informal networks can also be seen in those people who know -- through friends or family -- a professional or tradesman. Legal queries are easy if you can call a lawyer; medical opinions are easy to find if you live with a doctor; computer problems are easy to solve if you're related to a furry.

Companies will usually develop a structure that is intended to define the passage of communication: guidance flows from the CEO down to the workers, and results are reported up the chain. Many companies have a formal organization chart to depict this, along with role descriptions for each of the employees to clearly show appropriate lines of communication. However these structures are only part of the story -- in reality, workers tend to communicate with their friends, or peers, or whoever happens to be sitting near them.

Management consultants try to model these informal networks of communication to see how they affect the business (see here for an introduction). Some communication can be beneficial to the company; some can be harmful. Thanks to this research, it's also possible to understand how informal networks can help (or hinder) an employee.

In management-consultancy jargon, it's good to be a ``mobile'' employee. A mobile employee is one who is more likely to be promoted within the company, or more likely to leave for a better job. These opportunities means that mobile employees are less likely to be stuck, and likely to earn more money.

From a manager's perspective, mobile employees are difficult. They tend to command higher salaries but are often critical to the function of the company. If a manager loses a mobile employee, it can take two or more people to adequately fill the role.

Furries are often mobile employees, for three reasons:

\begin{enumerate}
  \item The furry world provides a broad range of connections. A typical furry group is more heterogeneous then a typical peer group. Accordingly, a furry is more likely to know people in different roles within his industry: perhaps technical, low-level support, management, the self-employed, service providers. Such connections can give access to information that might otherwise be difficult (or impossible) to attain.

  For example: an engineer furry might learn from a friend that a desirable job will soon become available.

  \item The furry world has a lot of people working in similar industries, particularly white-collar technical roles. The sheer number of potential connections increase the likelihood that a fur will learn useful new information that is not available to her colleagues.

  For example: a programmer furry might be able to cast for opinions of an alternative development environment without having to spend time and effort performing research.

  \item There is implied trust between furries. (This is a feature of our community I touched upon in my article titled The Furry Accommodation Network.) This means that furries are more likely to communicate freely and with respect, so the connections tend to be high quality as well as numerous.

  For example: a furry in a management position looking for a new employee may broadcast the opportunity in furry networks, because the implied trust makes hiring a furry a less risky (and more fun) proposition.
\end{enumerate}

These three reason are those that management consultants use to identity mobile employees. I think that there is another, overlooked, reason that can increase someone's exposure to new opportunities: the ability to move interstate or overseas for a new role.

A long-distance move is risky. The risk, at least from an employer's point of view, is that the newly moved employee (and family) may not `settle' in the new environment. Potential employers will usually offer a bonus to someone moving into a new area, often dressed up as ``relocation expenses''. This bonus is designed to retain a new employee who might be otherwise unhappy in their new location, hoping that the money will translate to loyalty.

Relocated employees often struggle because they have lost their social life. This is especially common when the new employee comes with a Significant Other, who doesn't gain the social benefits of the new workplace. For furries, this is an easily surmountable problem: there are furry groups to be found everywhere. (And a lot of our social world is online anyway.)

Research also shows that there is no correlation between the position of someone within a network and their personality (ref). Furries who are introspective and shy are just as likely to be mobile and/or high-value to a company. The loudest people often receive the most attention in an office environment, however these people tend to foster low-quality connections. The quieter introverts tend to forge more valuable friendships.

Finally, online social networks provide a special challenge for employers. There is little research on the effect of such networks owing to their relative newness (the world of research moves slowly), however early results suggest that networks like Facebook provide advantages to employees at the cost of the company. For this reason, many companies decide to block or limit social networking sites even though this is an unpopular decision. This is especially true in industries with few `mobile' employees: companies are concerned about the cost of unwittingly exposing their employees to better opportunities.

These opinions should be taken with a grain of salt. Most of the conclusions on the effect of online social networks are drawn from a landmark 2009 study on Microsoft's employees (link). The social networks investigated, from most common to least common, were LinkedIn, Facebook, Live Spaces, MySpace, Orkut, and Friendster. Twitter was mentioned in a footnote (no word on Fur Affinity). Clearly results drawn from this study are out of date, a conclusion reinforced by tone of the discussion (\textit{``MySpace remains the choice of geeks''}).

One final interesting, if largely irrelevant, point: furry networks also resemble dark networks, as understood in the context of terrorism and terrorist cells. In dark networks, the source of power is obfuscated. It's comparable to furries on Twitter, in that many accounts are private so it can be difficult to follow the progression of an idea. Social networking analysis was used to guess the location of Saddam Hussein's hideout: furries trying to track down someone on Twitter might (informally) use a similar technique, starting with known acquaintances and looking for a key furry who might provide a direct connection (ref).

The value of these informal networks will be felt differently by different furries. They are most likely to be strong in furry-rich industries, such as tech. They are less likely to be useful in industries with few furries, or in so-called ``dead end'' blue-collar jobs where mobility may be difficult or impossible.

In extreme cases, some lucky furries will find new jobs and prosper largely through the strength of the informal furry network; others will see no benefit at all. However it's nice to know that it's not just Skull \& Bones getting an unfair advantage.
