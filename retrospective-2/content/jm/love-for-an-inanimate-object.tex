\articlehead{Love for an Inanimate Object}{JM}{2013}

Some words of unwarning: this article is not about plushophilia, at least not in the sexual sense.

I like to mention sex in the first few sentences of my [adjective][species] articles when I can. I think it provides an engaging hook, something to help keep the reader enthusiastic while they wade through a convoluted premise, or parse paragraphs of statistics. (I sometimes even imply that an article has salacious content when it doesn't.)

Plushophilia, in the sexual sense, does exist within furry but it's marginal, at around 8\% of furries according to the Furry Survey (ref). My guess is that stuffed animals are a true paraphilia (i.e. sexual fetish) for a small subset of this small fraction. In furry's first wave, media coverage would often look to equate furry with plushophilia, in a clumsy attempt to explain our community as an entirely sexual phenomenon. It's safe to say that any conflation of furry with plushophilia is wrong, and that the collective furry groan whenever someone refers to us as `plushies' is thoroughly justified.

A lot of furries, of course, own stuffed animals. It's one of the ways that the furry identity manifests itself in the physical world. And it's normal for furries to have an emotional attachment to their stuffed animals, without the sexual objectification associated with paraphilia.

I'm an example: I'm a competent adult who owns a dozen or so stuffed animals, and I'm emotionally attached to them. I like having them around: I have a stuffed zebra with me at the moment to `help' me write this article. He provides a faint presence, social but without any social requirements. He is a substitute for quiet company.

A different person might like to have the TV on in the background at a low volume. The illusion of motion and life is equally unreal, but it provides enough of a reminder of a real presence to make us feel warm and appreciative, preventing a room from feeling cold and unwelcoming. It's a phantom of a human connection, just enough to work on a subconscious level.

I also like to take a stuffed animal when travelling, especially when travelling solo. They make me feel less isolated.

Happily for me, we live in a world where it is mostly okay for a grown man to discreetly carry a stuffed animal around. They are not a societal norm in most non-furry spaces, so I don't openly display them, but I don't actively hide them either. If one of my stuffed animals were to be revealed in the presence of, say, a work colleague, I wouldn't expect a negative reaction. It would be a minor eccentricity, nothing more.

There are some objects where it is completely socially acceptable to reveal an emotional attachment. Such objects are often said to have `sentimental value', an indication that an emotional attachment makes the object worth more to the owner than to a disinterested third party. Examples might include a childhood teddybear, or a motor vehicle, or sports memorabilia.

On the other hand, some objects are not socially acceptable. One example, one that parallels with furry in some ways, is the `reborn' subculture. This group, largely made up of post-menopausal conservative women in the United States, own and care for ultra-realistic baby dolls.

\begin{wrapfigure}{r}{0.3\textwidth}
  \begin{center}
    \includegraphics[width=0.25\textwidth]{content/assets/inanimate-object--pretenders}
  \end{center}
  \caption{Image from Rebecca Martinez's Pretenders series}
\end{wrapfigure}

Like furry, the reborn subculture is occasionally profiled in the mainstream press. They receive a common reaction, and an interesting one: people find it creepy.

Creepy. Disturbing. Repulsive. Yet no harm is being done. Clearly, these dolls are providing their owners with an emotional need. So why are people -- possibly including you, dear reader -- reacting negatively to something which is unambiguously positive?

Outside of some internet-savvy groups, furries sometimes receive a similar reaction. This reaction is the reason why furries are regularly asked to appear on freakshow/reality TV shows: the producers know that the viewers will have a strong, visceral reaction. (Such TV shows and media often play up the sexual component of furry for the same reason.) Like members of the reborn subculture, we furries are harmless, pursuing an unusual interest because it makes us happy.

So why the negative reaction?

Given that the reaction is automatic, this suggests it is a normal feature of human social behaviour. My guess is that people innately react negatively to people who are `different', and that this has an evolutionary biological explanation. People with unusual emotional needs may be less socially able, perhaps where this is sign of mental dysfunction. Unusual behaviour can weaken, or even destabilize, social groups. The negative group reaction, therefore, may act as a social countermeasure, to `normalize' the outlying individual. Some outlying individuals will successfully moderate their behaviour within the constraints of whatever the social group considers normal, while those that fail to normalize are outcast.

Gay people faced this problem in most parts of the world throughout the 20th century. Society has become more accepting over time, however it is still a problem in some parts of the world. It's a component of racism. It's an ongoing problem for many people with unusual sexual identities.

For we furries, societal pressures are a common consideration, especially in non-furry spaces. Some lucky furries with internet-centric lives, perhaps those with work in the IT sector, might be able to be completely open with no negative consequences: like my travelling stuffed zebra, it might be considered to be a harmless eccentricity. For most of us, furry is something best kept largely private, or perhaps shared only among close friends.

Many people in my living and working world would, I believe, find furry to be creepy. I like to have control over my outwards-facing facade, and so furry is something I wouldn't choose to be a subject of gossip. Conversely, I'm completely open about my homosexuality -- I'm lucky enough to live in a society where anyone reacting negatively would find themselves to be the outcast. So I'm `out' as gay to everyone, but `out' as a furry only to a few very close friends. Such compromises are a normal and necessary part of living in a social world.

I think it's an important skill to give the impression of `fitting in', without compromising those things that are internally important. As Kurt Vonnegut said in \textit{Mother Night}: ``We are what we pretend to be, so we must be careful about what we pretend to be.'' It's not always an easy balance, although sometimes it's just a matter of keeping your stuffed zebra in your luggage.

Just as importantly, we can notice and moderate our own natural negative reaction to outsiders. We can't change our subconscious reaction, but we can control what we say and how we act. It's the first step towards a more tolerant and accepting world.
