\articlehead{Furry as an Alternative to Religion}{JM}{2012}

Furries are a diverse bunch.

Our diversity means that we're often excluded from the mainstream. This is particularly evident in our sexual preferences -- only about a third of us identify as `heterosexual' or `mostly heterosexual' (Ref). Other traits displayed by some furries -- gender dysmorphia, heavy internet usage, or even simple geekiness -- can also play a part in our diversion from society's definition of `normal'.

Not surprisingly, furries do not closely embrace religion, a societal construct that can embody and tacitly enforce the norms of the mainstream. A little more than 50\% of furries are essentially areligious (Ref). This rate is about five times higher than for the wider American population (Ref).

Furry provides some of the benefits of religion -- I identify two in this article, loosely defined as `spirituality' and `community' -- that provide insight into how mainstream society might react to the challenges of our changing world. Furries embody some of the biggest challenges to religion in the twenty-first century: acceptance of diversity, the growing online world and, most importantly, the increasing rejection of religion altogether.

Religion is rightfully a sensitive and important topic. But before I go any further, I want to make two pre-emptive apologies.

Firstly, an apology to the religious, who may reasonably find this article offensive. I'm making a direct comparison between furry and God. To suggest that something as trivial and fleeting as furry can, and should, be compared to a deity would be ludicrous if I wasn't so sincere about it. And possibly even worse, I'm also making an unsaid comparison between this article here on [adjective][species] -- my interpretation of furry's morals -- and a holy book -- a divine interpretation of God's morals.

Secondly, an apology to the atheists, who may reasonably find this article condescending. Religion is an imaginary construct, so it's ridiculous to give it any sort of regard beyond lip service. I'm being respectful towards belief systems that are demonstrably false instead of talking directly about the topic at hand.

These competing paradoxical reactions make writing this article potentially a lose-lose situation. It's also one of the reasons why religion is such a difficult topic outside of conversations with like-minded people. The godly and the godless often see each other as the enemy: they respectively speak in ways that insult the other's philosophy. It's fertile ground for misunderstandings and angry escalation.

This article is intended to explore how our ad hoc furry community provides support to its adherents in much the same way as religious communities. I am not exploring theology.

For starters: furry is not a religion. As far as belief systems go, furry is reasonably comparable to totemism, a broad term covering those who believe they have a connection or kinship with a non-human animal. Totemism has been documented largely in indigenous populations in North America and Oceania, and a modern version of it still exists.

Modern totemists will often identify a `spirit animal', with whom they feel a close personal connection. That spirit animal is usually imbued with superpowers that give strength to the totemist. These powers are often described as a result of the animal's existence in a spirit world, from which they can provide guidance or provide literal physical support to the totemist.

Modern mainstream totemism (sometimes called animism) is considered to be a “new age” philosophy, along with other artifices appropriated from a range of cultures. Your patience for such quasi-spiritual guff will vary: your reaction to the usefulness of dreamcatchers, or perhaps your thoughts on the wisdom (or otherwise) of Chakotay from Star Trek: Voyager, might be a good guide as to whether totemism is for you.

If it sounds like I'm unfairly poking fun at modern totemism, I'm also poking fun at myself. I'm personally inclined towards a lot of this new-agey stuff -- I'm vegetarian, I meditate, I'm a hypnotist, I own a lot of ambient music -- although I would argue that I've appropriated useful aspects of newageism and discarded the dreamcatchers. I've read a fair bit on totemism and I wish I could recommend a good reference -- probably the least worst is Ted Andrews's Animal Speak (link), although there is a lot of nonsense to wade through, such as the author's insistence that his personal eagle totem can disable highway speed cameras. If you can tolerate such intellectual bankruptcy, then the book is otherwise a pretty good reference for furries looking to reflect on their relationship with their species of choice. You could, unfortunately, do worse.

Having said that, totemism and real religions -- and furry -- help us manage our inner world. The totemists and the religious both provide an `other' -- a spirit animal or a God -- that allows us to explore the most difficult aspects of the human condition. At the simplest level, using this `other' as a sounding-board makes it easier to negotiate a route towards happiness, or acceptance of mortality, or manage personal failure. The presence of this `other' means that we do not have to carry the mental load of complete personal responsibility.

For the areligious, furry provides an alternative for managing our internal world.

All human beings carry around an internal critic that thinks and acts in a way that is often contrary to the rational, moral being we imagine ourselves to be. We all hear an internal voice that reminds us of our permanent failure to live up to our own expectations. We all secretly struggle with depression, or lovesickness, or anger, or mortality, or whatever our own inner voice's favourite topic happens to be.

This inner voice is believed to be the cause of auditory hallucinations. People who `hear voices', as is commonly associated with schizophrenia, may simply feel that their inner voice isn't their own. Among the rest of us, our inner voice can still make itself known. We may find ourselves acting on otherwise repressed impulses when we are in a mentally delicate state, perhaps drunk or under stress.

The struggle to manage this conflict between our inner voice and our desire to be a perfect rational being is, for many philosophers, at the core of the human condition. Some people might over-manage their atavistic impulses and become uptight, while others might under-manage and become emotionally unpredictable.

To a religious person, a deity often represents a perfect and unattainable ideal who rewards those who try to improve themselves. This provides a motive force for the internal struggle, providing meaning as one strives towards self-improvement.

Our furry selves may help in a similar fashion. For many furries, the animal-person alter-ego represents an unattainable ideal, mentally and physically. Other furries may imbue their avatar with desirable qualities, and roleplay as a first step towards self-acceptance. The fact that our avatars are not human may be helpful, in that we can never feel like we have reached our destination, much in the way that a man can approach but never attain godliness.

Furry also provides social guidance. We do not have anything as formal as a set of commandments, but we're still subject to unsaid norms that inform the boundaries of appropriate behaviour within the community. For example: furries place great value on tolerance; our friendships are more intimate; we talk freely about sex and sexuality.

These unsaid furry standards are explored regularly here at [adjective][species]. However they are difficult to pin down: I suspect that a non-furry reading these pages wouldn't gain much understanding about what furry `is'. In general, we tend to discuss common experiences (Rabbit on Fursuit Magic) or explore unusual phenomena (Makyo on furry's dearth of women, Eighty-Twenty) but we tend not to try to define `furry'.

It's not through lack of trying, just that we furries aren't easily categorized. I might propose, for example, that all furries have an animal-person alter-ego, that we create and name a furry reflection of ourselves. However this is neither mandatory nor universal -- [adjective][species]'s very own Rabbit, aka Phil Geusz, doesn't interact through an imaginary furry representative. (Having said that, his books are very `furry', particularly so if you are inclined towards bunnies.)

We have also explored apparently simple topics, like species selection. Assuming that, say, furry wolves must have different motivations for species selection from furry foxes, we hoped to find evidence in the Furry Survey data. However several creative data-mining attempts have discovered almost nothing. I can think of exactly one significant correlation related to species: furry women are much more likely to choose a domestic cat for their fursona. (Ideas for future searches are welcome.)

The spiritual aspects of religion are difficult to pin down as well. Taking Christianity as an example, the world has changed to a point where the bible has ceased to be a realistic reference for behaviour. (Atheists sometimes suggest that failure to adhere to the word of the bible is proof that it's at least partly false. I suspect that Christians roll their eyes at this criticism.)

The world is always changing, a process that become very rapid following the industrial revolution some 200 years ago. Huge increases in efficiency and income have led the world's population to increase from about 1 billion to today's 7+ billion, largely away from rural communities and into urban centres.

Religion has had to adapt to this change. Before cities started growing in the nineteenth century, people related to their religion at a community level. The church was at the heart of the community, a role perhaps comparable to that of the government today (as illustrated by the Soviet Union's attempt to enforce universal atheism).

Population growth and the rise of the cities has changed religion. The paradigm of a community church has foundered in the wake of cultural diversity, social diversity, and -- more recently -- the advent of the internet. Furry is less than 30 years old and so has easily adapted to the twenty-first century. Most religions have centuries or millenia of history: they were once described to me as like an oil tanker, in that they take a long time to change course. By that metaphor, furry would be a speedboat.

While the spiritual aspects of religion haven't appreciably changed over this time, the community that once centred around a church has. In diverse cities, church-based community will necessarily be relatively monocultural compared to the greater population. This disconnects you from your citymates, a disenfranchisement from society.

If the inhabitants of a city are not engaged with one another, it can lead to weakening of the social contract. This causes problems on a personal level -- a city can be lonely -- and on a wider level -- illustrated by the 2011 London summer riots. The furry community provides a solution, at least on a personal level.

Our mutual engagement in the furry community brings us closer together. The social contract within furry is strong: we freely offer shelter and company to furry strangers (The Furry Accommodation Network); we offer moral support to the depressed (A Rough Guide to Loneliness); when furry strangers pass away, we are personally affected and provide charity (Death in the Fandom). This sense of community is very similar to that traditionally provided by religion, bringing an entire community together, allowing the strong assist the weak.

The communal furry experience is more tangible than the spiritual side. Unsurprisingly, we at [adjective][species] regularly write about the furry community's actions, and those articles are almost always the most interesting to read. So, if you've read this far, thanks -- I hope you didn't just read to the end so you can comment and berate me for being offensive/condescending [choose one].
