\articlehead{Carroll Ballard's \textit{The Black Stallion}}{JM}{2012}

\textit{The Black Stallion}, Carroll Ballard's 1979 debut feature, is a great film.

It's based on a series of children's books but isn't simplistic or pandering. It's meditative, beautiful, engaging and -- of course -- great for any furs with an affinity for our equine friends.

The movie opens with a young boy, travelling on a foreign ship with his father. The ship is carrying the titular black stallion, a beast with a questionable temperament and unquestionable power.

There is a storm. The boy frees the horse from its restraints before both are thrown overboard. The ship sinks. They find themselves on a deserted island as the only apparent survivors, marking the end of the prologue and the beginning of the movie proper.

The first half of the film is a nearly wordless tale of survival. The weathered pastels of the island and the primary blue of the ocean are stunning. This landscape acts as an ancient canvas for the emerging relationship between boy and horse.

Ballard allows the story to develop naturally, with long scenes showing the boy adjusting to his wild surroundings. Such unhurried minimalism is comparable to the quiet, tense scenes of exploration in \textit{2001: A Space Odyssey} (1968): the boy's quest for survival has parallels in Kubrick's apes discovery of tools, or Floyd's moonwalk to the excavated monolith, or Bowman's slow discovery of Hal's treachery. Like Kubrick, Ballard simultaneously evokes tension and wonder with \textit{The Black Stallion}, although never with the thematic reach or artistic pretension of 2001.

The second half of the film, following the rescue of the boy and the horse, is less effective. But more on that in a moment.

The relationship between the boy and the horse is one of codependence. The boy is saved from probable death twice by the stallion. Firstly, after being thrown overboard from the ship, the boy grabs the horse's restraints and is carried to the safety of the island. Secondly, on the island, the horse tramples an aggressive snake.

The boy saves the stallion from probable death twice. He frees the horse from its restraints on the ship and again on the island, where the horse becomes trapped in a rocky outcrop.

The growing trust between the two leads to \textit{The Black Stallion}‘s best scene, where the boy attempts to feed the horse by hand. If I can anthropomorphize the horse for a moment (and I'm sure readers of [adjective][species] won't mind), I'd argue that this scene shows the greatest acting performance ever by a horse. It is certainly a triumph of animal handling. The horse is clearly nervous as he approaches the boy, pushed and pulled by the competing emotions of anxiety and hunger. His slow approach to the boy's offering is filmed from a distance: a long single shot. The scene is amazing and natural and joyful.

The boy and the horse, doomed to early death alone, combine to thrive on the island. The horse's power provides a blunt instrument against the forces of nature, protecting them both from danger. The boy's resourcefulness helps them survive day-to-day, providing food and shelter. The boy is rightfully fearful of the highly-strung stallion initially but, as the two help one another, respect grows into trust which grows into a tight bond.

The scenes showing the friendship between the boy and the horse are my favourite in the whole film. The two, agents of one another's needs, start to find island life easy. They play: they swim together and the boy (eventually) learns to ride the stallion. These scenes -- the stallion's enthusiasm and the boy's laughter -- wordlessly depict the joyfulness of their bond.

From a less life-affirming perspective, it's possible to interpret the stallion as an agent of death. In the film's chronology, he seems to be the arbiter of who lives and who dies. Ballard's films often starkly depict death, and this is the case in \textit{The Black Stallion}, which opens with the death of the boy's father and presumably the rest of the ship's passengers and crew.

The boy survives the shipwreck because his obsession with the horse draws him to deck to cut the stallion's restraints. As the horse jumps overboard, the boy is tossed over by the storm, saving him from the boat's subsequent explosion. Later, after the boy frees the horse from his tangle in the island's rocks, the horse saves the boy from the snake. In both cases, the boy's survival is directly associated with -- and arguably caused by -- his selflessness towards the stallion. The horse, as Death, shows mercy towards those that show mercy to him.

The same events could, of course, be interpreted as a representation of the power of friendship. However I prefer the horse-as-manifestation-of-Death theory, and I point towards the stallion's black coat as evidence.

It might be a stretch to suggest that \textit{The Black Stallion} is an exercise in karmic vengeance, but the horse is shown to be wild, powerful, and dangerous. In an early scene on the boat, the horse is shown fighting against his handlers as they corral him into his stall. The boy is fascinated by the stallion's power and becomes drawn to him, firstly by supplying illicit sugarcubes, and ultimately cutting him to freedom in the storm.

On the island, the horse is still dangerously flighty. However the boy's obsession means he does not see the horse as a threat, and his persevering kindness is rewarded. Their friendship endures when the boy is eventually rescued: the horse swims out to the boat, convincing the rescuers to bring the horse on board as well.

Back at home, the boy is reunited with his grieving mother, and the movie becomes a different beast. The stallion escapes from their yard; the boy meets possibly the most egregious magical negro in cinematic history (who comes with magical and totally gay horses); the horse is found in the barn of a retired jockey; they enter into a horse race for no obvious reason other than to give the film a convenient, and clichéd, climax.

The retired jockey is played by Mickey Rooney, who is most famous for hamming it up as a cherub-faced child actor in the 1920s and 30s. His brand of ham has aged poorly, and his scenes in the \textit{The Black Stallion} are the worst of the film. (Kelly Reno, as the boy, comfortably out-acts one of the most celebrated child actors of all time.) While researching this article, I was shocked to learn that Rooney was nominated for an Academy Award for \textit{The Black Stallion}. It must have been a sympathy vote. He did not win.

For all the lameness of the second half of The Black Stallion‘s plot, it is still a beautiful film. The small town in which the boy lives is a perfect slice of rural America. And there is a racetrack scene -- a reporter is invited to see the stallion go through his paces -- set in a night-time cloudburst that stands alongside the best moments of the film.

In this way, the cinematography of \textit{The Black Stallion} is comparable to the craptactular films of Michael Bay (\textit{Bad Boys}, \textit{Pearl Harbour}, \textit{Transformers}). Bay's films may be irredeemable nonsense, but they are beautifully shot. A Bay film, randomly paused, will often be composed and striking. (It's a pity Bay and his team don't put as much effort into the plot, direction, continuity, or assessment of his audience's intelligence.) \textit{The Black Stallion}, even in it's lowest Rooney-filled moments, is always pretty.

The climactic race scene of \textit{The Black Stallion} is almost Bay-worthy in its preposterousness. However the horseback scenes, shot largely in close range around the boy and the horse, are vivid and moving in their depiction of the stallion's speed and power. A similar technique is used in \textit{The Club}, a 1980 film that follows an Australian Rules team. By filming close to the players and bringing their footfalls to the front of the sound mix, the viewer gets a visceral sense of the footballer (or horse) testing himself to his thoroughbred limits. These scenes, in \textit{The Black Stallion} and in \textit{The Club}, share the athlete's perspective with the viewer like no other.

Notably, and laudably, \textit{The Black Stallion} is not a coming-of-age story. The boy is shown to be self-reliant from the beginning of the film but is very much a child throughout. His journey, starting with the death of his father and ending in a horse race, is defined by his relationship with the horse.

Both boy and horse are juvenile. They complement one another and help one another survive, thrive, and succeed. The boy is creative and the horse is powerful: they are, each, half a man. Together they are a match for the world.

\textit{The Black Stallion}, then, is a celebration of childhood. One day the boy will grow and become strong and powerful himself, and he will no longer need his other half, the horse. However this is not the subject of the film. In \textit{The Black Stallion}, both boy and horse are free to enjoy and explore their childhood, through their friendship.

\secdiv

This is the first of four posts on the films of Carroll Ballard. The other three articles will come irregularly, as I write them. All four movies are great. Choose your species and join us:

\begin{itemize}
  \item \textit{The Black Stallion} (horse)
  \item \textit{Never Cry Wolf} (wolf): coming soon
  \item \textit{Fly Away Home} (goose): coming soon
  \item \textit{Duma} (cheetah): coming soon
\end{itemize}
