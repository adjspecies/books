\articlehead{The Second Wave of Furry}{JM}{2012}

Furry is an evolving phenomenon. This article is an attempt to capture where our community is now, and how we got here.

Furry's First Wave, its origin and consolidation as a unique phenomenon, lasted up until the turn of the century. Furry is currently in its Second Wave, a fast-growing adolescence.

The First Wave of furry is neatly captured in Retrospective: An Illustrated Chronology of Furry Fandom, hosted on Flayrah\footnote{http://www.flayrah.com/4117/retrospective-illustrated-chronology-furry-fandom-1966-1996}.

The First Wave defined the furry community. Furry began as an offshoot of sci-fi fandom and almost immediately become notable for production of original anthropomorphic content. Furry had less of a focus toward pre-existing art than sci-fi fandom, and in this way furry started to transcend the usual boundaries of a fandom. Over time, furries started to explore the idea of ``being'' a furry, and a struggle developed between those who considered furry a part of their personal identity, and those who saw furry as a fandom. Furries roughly spilt into these two groups: so-called `lifestylers' and `fans'.

Culture wars between the lifestylers and the fans defined the First Wave. The lifestylers openly incorporated sexuality into their identity. The fans were dismayed by the permissiveness shown towards extremes of behaviour, particularly where sex was involved. The furry fans put a premium on quality and family-friendliness, creating Yerf; the furry lifestylers put a premium on acceptance and open sexuality, creating VCL.

The fundamental conflict was simple. For the fans, furry was something you enjoy. For the lifestylers, furry was something you are.

The lifestylers won the culture wars and, in the Second Wave, have become the furry mainstream. There are still furry `fans' however they have typically been around since the First Wave. Furry is a broad church and fans are not excluded: it's simply that new furries tend to take up an animal-person identity with a species and a new name by default.

Furry is still maturing. Second Wave furries are continuing to explore the idea of furry identity, and also starting to consider the community's culture and values.

(A note on terminology: I like `community' as a description of our collective although `fandom' is probably more common, and is used by other writers on this site. I'd argue that `fandom' is deprecated because, while there are many fans within furry -- anime, MLP, Redwall, etc -- we are collectively not fans of anything in particular. This is where furry deviates from fandom: we created and propagate a furry universe, a virtual reality of animal-people that exists parallel to the real world.)

Early expressions of Second Wave furry included some conventions (notably ConFurence, which received a lot of criticism for being overtly sexual) and FurryMUCK. In these spaces, furries presented as if they were their animal-person avatar, a furry cultural norm that is now widely accepted. Most furry spaces are Second Wave although this is not always the case: arguably of the two Australian furry conventions, MiDFur (with occasional non-furry guests) is First Wave, whereas the newer Furry Down Under (with a focus on socializing and fursuiting) is Second Wave.

The maturation of furry is reflected in media coverage. During the First Wave, those willing to publicly discuss furry were often on the fringes of the group, and were largely selected to reinforce the freakshow element. Serious attempts to understand furry, such as a 2001 Vanity Fair article, were largely hijacked by furries who were unwilling or unable to act in a socially appropriate fashion. As I have said before here on [a][s], the most visible members of a minority are rarely the best ambassadors. The result was cringeworthy, and furries ran a mile from the image portrayed in the media.

This is no longer the case. Second Wave furries are collectively comfortable with the idea of furry as an identity. Media outlets, regardless of whether they have honourable intentions, are presented with a community that knows how to present itself. Coverage often tends to focus on the more unusual aspects of furry, or even the range of sexualities on display, but the overall vibe is usually one of disinterested acceptance. The visibility and city-wide acceptance of Anthrocon during its annual residency in Pittsburgh is a good example.

I saw Anthrocon's Sam Conway speak a few years ago, and he went out of his way to talk about furries who held respectable positions in the real world. He mentioned furry aeronautical engineers, medical doctors, and the like. It was a speech from someone who was trying to convince himself -- and his audience -- that the First Wave furry stereotypes no longer apply. He was, like Ophelia, protesting too much, as if he could will such a situation into being. They were the words of someone who had experienced the worst of the First Wave furry culture first-hand, where furry's reputation was repeatedly tarnished in the media by extreme elements of the group.

Conway's concerns are reasonable but out-of-date. Nowadays, the idea that furries might be innately unemployable is all but nonsensical.

However the perceptions of the furry group in the First Wave suffered from the actions of a visible minority. Furries distanced themselves from such behaviour, insisting that real furries are people who simply, ``have an appreciation for anthropomorphic characters''.

Pre-emptively defensive sentiments like Conway's persist on Wikipedia. There are hardworking wiki-guardians who maintain furry's entry, the highest-profile source of information for someone unfamiliar with the community. It opens with:

\begin{quote}
  The furry fandom is a subculture interested in fictional anthropomorphic animal characters with human personalities and characteristics.
\end{quote}

The article provides an alternative definition of furry for ``furry lifestylers'', quoting a line from Usenet that is about 20 years old:

\begin{quote}
  \ldots a person with an important emotional/spiritual connection with an animal or animals, real, fictional or symbolic.
\end{quote}

Wikipedia's portrayal of furry, like Conway's speech, is firmly First Wave. We no longer need to act so defensively: our collective image is no longer shaped by a few outliers.

Fandom, as opposed to furry, is still largely perceived as a collection of social rejects. In many cases it's reasonably applicable: if you are obsessed with Hamtaro (say), it's likely that you are either very young or you have a very limited relationship with the wider world. The stereotype of the narrow-minded geek, that of Comic Book Guy or the stock action-figure-collecting sitcom character, is one of fandom. Furries are still pretty geeky and fandom-oriented -- 61\% of us describe ourselves as `a fan of science fiction' (Ref Furry Survey) -- but it's no longer the driving force of our community.

(I don't want to suggest that fandom geeks are any better or worse than furries. I'm merely trying to describe the progression from furry's First Wave and its fandom origins, to today's Second Wave. I appreciate that my embrace of fandom stereotypes is reductive and possibly a little insulting. I mean to say that fans will be over-represented by Comic Book Guy, not that all fans are like that.)

Furry media is largely Second Wave. Our social sites -- Fur Affinity, Inkbunny, Sofurry -- are Second Wave almost by definition, as furries socialize nearly exclusively as animal-people. Meta furry sites, like Flayrah and [adjective][species], who look at furry with a critical eye, are also Second Wave. There are still echoes of the First Wave culture wars however these are largely marginalized to `below the line' forums, comment threads, and the juvenalia of 4Chan and Encyclopaedia Dramatica.

One of furry's greatest features is that it is decentralized. We do not have a universally-respected figurehead or a formal code of conduct. Our culture and our community are fluid. Our most useful tools are those which allow furries to come together as a loose collective: conventions, social media, art depositories.

As we grow -- and we are growing worldwide, fast -- our culture is consolidating. New furries learn to abide by the unwritten rules of the pre-existing furry culture. This maturation is the furry Second Wave. Keep your whiskers erect and your ears perked for signs of the next step forward.
