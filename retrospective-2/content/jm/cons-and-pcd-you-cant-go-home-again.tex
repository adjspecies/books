\articlehead{Cons \& PCD: You Can't Go Home Again}{JM}{2012}

Chicago's Midwest FurFest took place last weekend. It's our second-biggest convention (larger than California's Further Confusion, smaller than Pittsburg's Anthrocon), with around 3200 attendees in 2012. I've never been to Midwest FurFest but many of my friends have, and by all accounts it's one of the best-organized and most enjoyable conventions.

Last weekend my Twitter feed was filled with those enjoying the convention, easily identifiable by the \#mwff hashtag. From many miles away, I vicariously observed a fursuit parade, mutual friends meeting for the first time, a hotel evacuation, and any number of social antics.

Searching for all those on Twitter using \#mwff, I watched many furries -- largely strangers -- explore the convention. I saw expressions of furriness, geekiness, drunkenness, flirtatiousness, and happiness. It was like peering into an alternate reality, one filled with good-natured animal people.

Furry conventions have a culture of their own. The culture is especially strong within those conventions that are able to monopolize an entire hotel or convention centre. When you pull into the carpark or walk through the front doors of such a convention, you enter a different world. It's a lot like visiting a foreign country.

Arriving at a furry convention can be disorienting. There is a lot of information to assimilate: a different culture, an unfamiliar geography, and new rules. (Where do I check in? Do I need to wear my badge? How do I get to my room? Are my friends here?) It takes some time to adjust to these surroundings, which might be as little as a few minutes (for a seasoned convention-goer) or many hours (for the unsuspecting newbie).

This feeling of disorientation also occurs when you arrive in a foreign country: it's known as information overload. The human brain does a great job of identifying important signals -- human faces, voices, road signs -- amongst the noise of the world. When walking into a new environment, such as a furry convention, it's difficult to determine what is relevant -- and so our brain tries to manage more information. The extra demand on our unconscious brain comes at the cost of conscious brain power, reducing our ability to make decisions or think logically.

Information overload can make us feel disconnected from our surroundings. We become less mindful, and we may feel like we are observing ourselves from a distance. This disconnection combined with reduced cognisance creates confusion. This is why furries tend to aimlessly mill around the front entrance on opening day, and why many retailers think a `greeter' provides a positive focus for a new customer who might otherwise be hesitant.

We, hopefully, adjust fairly quickly. In a particularly unfamiliar environment -- perhaps your first visit to a furry convention or your first time in a new country -- this adjustment can be a slow process. The safety of a hotel room can often be a relief, and courage can be required to open the door and try again.

Once we adjust to the new environment, we tend to accept otherwise novel experiences as a `new normal'. At a convention, the new culture is a mix of the exotic and the familiar.

A furry convention is neither high-culture nor low-culture, although there are elements of both. Avant-garde art sits next to pornography; philosophical discussions compete for time with drinking games; ruminations on sexual politics give way to lists of the sexiest football team mascots. The tone is not exactly lowbrow, but it's not exactly transcendent either.

More tangibly: furry friendships tend to be quite tactile, so there is a lot of interpersonal physical contact, most obviously when fursuiters are around. Friendly (platonic) physical contact at a furry convention might, outside of the convention doors, be perceived as sexual. The physical closeness seen at conventions seems to be tied into a kind of physical exuberance as well, and it's easy to guess that this is because touching and being touched makes us happy.

There is also a kind of collective delusion at furry conventions, where we tend to treat each other as if we were really our animal-person avatar. Our conbadges supply the picture and name of our alter-ego, and we tend to accept these as true. There is even a tendency for convention-goers to organize by species, and there are many versions of a [species]-only room party. It's tempting to regard this as trivial, but I think this reinforcement of our furry identity helps us relax the masks that hide our furriness in day-to-day life.

Finally, the outward traits of furries as a collective are on display, for good or for bad. We are very male-dominated (about 80\%) and we are largely non-heterosexual (about 65\%). We're also techy, fussy, sexy, obstinate, poorly dressed, and unathletic.

This all requires adjustment, and it's not always conducive to relaxation and enjoyment.

The cultural differences are not the only challenge. Conventions are, fundamentally, a social environment. It's important to either have plenty of friends or have the opportunity to meet new people (perhaps by attending a [species]-only room party). Without a large social group, a convention can be a very lonely place. Much like a visit to a foreign country, if you can't engage with the local culture on some level, your only other option is to retreat to your hotel room. And when that door closes, you find yourself wondering why on earth you came here in the first place. It's not nice to feel out of place in a situation you've spent a lot of time and money to put yourself in.

For those that thrive in the convention environment, it can provide an immersive counterpoint to the real world. The convention culture is one in which we can relax and feel liberated from stifling social norms. Like an overseas holiday, we can temporarily disregard our responsibilities and failures in the real world. However, when the convention is over, we must cross the border and readjust. This can be disorienting, a phenomenon known among travellers as `re-entry shock'.

The real world can feel unfamiliar when we return. Compared to a furry convention, the culture can feel restrictive and faintly ludicrous. We may find ourselves feeling slightly disconnected as we leave, just as we did on arrival.

The phrase “you can't go home again” refers to the feeling experienced by someone from a country town, who returns home after living in a city for a while. The person who grew up in the country town is different from the person who returns: the reality of rural life jars with the rose-tinted glow of nostalgia.

If we find comfort in the culture of a furry convention -- the tactile friendships, the connection with our furry self, the acceptance, tolerance, exuberance -- we might be unwilling to readily reintegrate into the real world. We may feel some resentment toward society's norms, even though we had accepted these before the furry convention. It can take time to overcome post-con depression. We have changed. You can't go home again.
