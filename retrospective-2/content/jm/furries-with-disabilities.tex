\articlehead{Furries With Physical Disabilities}{JM}{2012}

For many furries, there are big physical differences between their real-world bodies and their preferred avatar. We often act as if our animal-person representation really exists: we might consider the logistics of tails, we might miaow or bark a greeting, we might assume personality traits that reflect our perception of our species.

Such roleplay is central to the furry experience for many people. Online, furries commonly present as their animal-person avatar and will socialize as if everyone else were their fursona. This behaviour translates, to an extent, to real-world furry spaces, from one-on-one meetings through to conventions.

This `fursona illusion' occurs regardless of how closely our real body matches our avatar. For those furries who feel their real-world body doesn't reflect their self-image, this can be a liberating experience. And for furries who are physically disabled -- perhaps wheelchair bound -- it has the potential to transcend their disability.

People who are physically disabled have a challenging life. Most obviously, they may be physically restricted by a society that is set up for the able-bodied. However restrictive, this is less of a problem from a mental perspective, and more a logistical puzzle that needs to be solved. This is particularly the case for those who are congenitally disabled, and so experience physical restrictions as `normal'.

More subtly, and more importantly from a mental point of view, is that physically disabled people are treated differently by able-bodied members of society. Anyone who physically presents in an unusual fashion -- weight, race, clothing, anything -- will tend to suffer from the same prejudice, where other people become unsure of how to engage in social contact. This is not to place blame: it is simply human nature. (I'll discuss the psychology of this prejudice a little further down.)

Research shows that the well-being of the physically disabled is strongly correlated with their sense of community engagement. Efforts to improve the lives of the physically disabled therefore often focus on social aspects. This is the target of public awareness campaigns with slogans like ``see the person not the disability''.

Furries pay less regard to physical appearance. We are used to treating fellow furs as if they were their animal-person avatar. The furry world, then, may provide a social environment where physically disabilities are less relevant.

I chatted with three physically disabled furries who were happy to share their experiences inside and outside the furry community. These conversations took place over text and are edited for clarity and length.

\textbf{BlooCat} (IBloo on Fur Affinity) is a UK fur with muscular dystrophy and a wheelchair, who might be described as a garden-variety furry:

\begin{quote}
  My fursona is just a cat representation of myself. By that I mean she shares my name, age, personality etc. Having my fursona do things I can't do is fun, but sometimes I don't like it because it feels less me. My disability isn't all I am, but it's also not something I feel that I want to get rid of.
\end{quote}

BlooCat's physical disability is obviously restrictive, and she finds that people are often unsure of how to react when she meets them. This awkwardness is also experienced by \textbf{Shorebuck}, an Australian fur who is very mildly disabled -- he has diplegia, a form of cerebal palsy. He isn't physically impaired in any significant sense and doesn't consider himself to be disabled, however his diplegia affects his gait, ``giving off the appearance of a limp -- or dancing''.

Despite the irrelevance of his condition from a physical point of view, Shorebuck still suffers:

\begin{quote}
  Socially, it affects a lot. Some people have been freaked out by it, and some couldn't give a crap. Some people look away when they see me.
\end{quote}

The story is similar amongst the more severely disabled. \textbf{Nornhound} is wheelchair-bound fur with a rare condition called Fibrodysplasia Ossificans Progressiva (F.O.P.), which causes painful unnecessary bone growth in her muscles.

\begin{quote}
  To a complete stranger, I am certainly known as `the disabled kid/teen/adult woman', and these strangers treat me differently from an able-bodied person. When I was in my early teens, strangers would automatically assume I had an intellectual disability, and treat me as such. They were often hostile, too.
\end{quote}

This initial awkwardness doesn't occur in the online world: physical disabilities become invisible and, from a social point of view, mostly irrelevant. The internet also provides tools that can reduce the logistical challenges posed by an able-bodied society, such as online shopping, and opens different employment opportunities. Largely due to these factors, research shows that the internet improves the wellbeing of the physically disabled (Ref).

It's not all good news. As will be clear to anyone who has spent time reading forums and comment threads, the internet can be corrosively negative. This has a greater than average impact on physically disabled people, because they are more likely to rely on the internet. Furthermore, use of the internet for escapism -- such as online gaming -- also has a negative impact on the wellbeing of the physically disabled (Ref).

From this research, it can be inferred that the online world provides the greatest benefit to the physically disabled when it is social and enjoyable. The furry community may provide this, and more -- the online furry world translates, in part, into the real world, because of the persistent `fursona illusion'. If you are physically disabled, engagement with the furry community may lead to a better offline social experience, because furries will tend to see the animal-person alter ego.

BlooCat:

\begin{quote}
  The physical barrier is more easily overcome with furries. I find a lot of non-furries are a lot less tactile with me than they would be with other people. Something that really struck me at a recent convention was the amount of people (even strangers) that would come up and ask for a hug.
\end{quote}

Shorebuck:

\begin{quote}
  I'd say the furs just accepted me [as an arctic fox named Shorebuck] -- that was the main focal point. `You are this fur, pleased to meet you.'
\end{quote}

The initial awkwardness experienced by many people when meeting someone physically disabled may be less common among furries, but it still exists. This is an unavoidable outcome of our human nature as social beasts.

In the furry and non-furry worlds, social groups tend to act as a kind of meritocracy. We tend to socialize with our peers: people who occupy a comparable position in life. The process of peer group selection is driven by our desire to `fit in' -- we tend to change our own behaviour towards the group's behaviour; outsiders who meet or exceed the group's standards are welcomed; outsides who fail to meet the group's standards are rejected.

This phenomenon of normalization is clearly demonstrated by a 2007 study that tracked the incidence of obesity within social groups over a long period (Ref). The results showed that social norms have a very significant impact on obesity risk: essentially, that fat people tend to have fat friends. The results do not strongly suggest (as was reported in the media) that having fat friends can cause you to be fat; more that fat people are likely to find and keep equally generously-proportioned friends.

Such normalization of peer groups occurs everywhere in human society, with the criteria varying depending on the nature of the group. At its simplest level, people tend to have peers that are a similar age. To choose a few examples from the furry world: skilful artists are likely to hang out together, strong programmers are likely to hang out together, and furries with similar sexual interests are likely to hang out together.

This unsaid enforcement of social norms is a natural process, but it can have negative consequences if you are different. Furry is a broad church -- geekiness, gayness, intelligence, introspectiveness, etc -- and accordingly many furry readers of this article will be familiar with how someone different can fail to `fit in' to a social group. Or, to put it another way: the girl in the wheelchair isn't going to catch the eye of the captain of the football team.

The feeling of awkwardness felt by many when meeting someone physically disabled is normal. It is rooted in the same psychological phenomena that lead to peer group normalization. When we meet someone unusual, we are unable to draw upon on an unconscious `social script' that we use with our regular peer groups. This causes us to engage our conscious mind: we ask ourselves ``what should I say''. This leads us to think about how we are being perceived (psychologically, we become `self aware'), which can make us awkward and anxious. It's the same process that makes teenage boys nervous around teenage girls.

These feelings of anxiety are unpleasant and can provoke someone to withdraw from a conversation. This is frustrating for the physically disabled person, who sees it all the time. Unfortunately there is no easy way to overcome this initial conversational hurdle, no fallback social script.

BlooCat:

\begin{quotation}
  I know it would be a lot easier if there could be a set script, but it's a very personal thing. What I think is okay, someone else may not agree with. Take for example another girl in a wheelchair I know, we had a discussion about how we feel about people touching our wheelchairs when we're in a bar or something. Her view was that it shouldn't be touched as it's your personal space. For me I don't really mind if someone leans a bit on my chair, it's just a chair. A fancy bar stool even.

  The same applies for when you meet a disabled person for the first time. I think people just need to disregard the disability/wheelchair. Think of it as meeting a person rather than a disabled person.
\end{quotation}
