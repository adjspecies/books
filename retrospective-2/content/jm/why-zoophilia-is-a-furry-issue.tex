\articlehead{Why Zoophilia is a Furry Issue}{JM}{2013}

Zoophilia is fairly visible within furry.

Most obviously, so-called `feral' art is ubiquitous, and some animal characters -- the cast of The Lion King comes to mind -- seem to be minor sex symbols in some circles. More personally, furries sometimes actively denote themselves as zoophiles in social media, perhaps on their Fur Affinity page.

Klisoura's Furry Survey, which at its peak received over 9000 annual voluntary responses from furries worldwide, shows that 13-18\% of furries self-identify as zoophiles. This does not mean that all these furries have had sexual contact with a non-human animal; these furries are probably just reporting sexual attraction. However this is significantly higher than the general population.

Little research has been performed on zoophiles. Serious attempts to study the phenomenon are limited to the last ten years or so, at a level that academics compare to analyses of homosexuality in the 1960s (ref). All of these newer studies rely, in part, on Kinsey's landmark 1948 study, Sexual Behaviour in the Human Male (link) for data on the incidence of zoophilia.

Kinsey estimated that 6\% of American men have sexual contact with a non-human animal during early adolescence. The overwhelming majority of these cases were in rural areas. Such contact later in life quickly becomes vanishingly rare.

(There have been other studies, but none of any consequence. Notoriously, Alvarez \& Frienhar—ref—in 1991 reported rates of bestiality within the general population of 10\% to 15\% however this was based on a pitiful sample size of 20, all psychiatric staff.)

Kinsey attributes cases of sexual contact with non-human animals to young men lacking an available human partner. This is known as `situational sexual behaviour': sexual behaviour that takes place because of a dearth of otherwise preferred options. Such situational sexual behaviour is not limited to zoophilia. There are many other examples, gay sex in prison probably being the most obvious.

Like prison, situational homosexuality is common in gender-segregated communities. Gay sex is endemic in the armed forces of countries that have compulsory military service, Iran and Saudi Arabia in particular (ref). (The Singaporeans have a particularly unusual way of managing homosexuals in their National Service: openly gay men are given restricted duties that depend on whether they are `effeminate' or `non-effeminate'. Straight but effeminate men, or otherwise nonconforming men, are also given special duties. I like to imagine that there are whole platoons of drag queens in the Singaporean army, possibly defending Orchard Road from last season's shoe fashions.) Homosexual behaviour between young heterosexual men is also widespread in Muslim East African nations (notably Sudan) and single-sex boarding schools (notably England).

Situational homosexuality occurs inside the male-dominated furry community too. I've written a full article looking at the availability of men (and unavailability of women) within furry -- It's Raining Men -- detailing the plight of the heterosexual male furry. Suffice to say that their options are limited. Furry is an open environment that fosters intimate friendships (regardless of sexual preference), so it's not surprising that many heterosexual young furries will engage in mutually enjoyable sexual contact with male friends. The blunt categories of `gay' and `straight' are not strictly applied in the furry world (like they are in general society), so a heterosexual can engage in same-sex behaviour without risking the ire of his peers, or provoking an unresolvable identity crisis.

Situational bestiality also occurs within furry, partly due to furry zoophiles making their animals available for sexual contact. And, as Kinsey showed, it's common for young men with access to non-human animals to sexually experiment during adolescence. Further: frank depictions of zoophilic activity are easy to find in the furry community, as is frank discussion on the topic. Given this availability, it's inevitable that some young male furries will explore this side of their sexuality.

Kinsey's estimate of the numbers of adolescent men having sexual contact with non-human animals (6\%) was published in 1948. This was widely considered to be out of date by the mid-1970s (ref) and is even less relevant today. The reason for this is simple: we live in better connected and more sexually liberated society. The sexual revolution significantly improved the availability of women (and men) for young men; the proportion of people living in rural areas has declined; the spread of television provided a homogenizing influence on moral behaviour; the internet has significantly improved the connectedness of rural communities. The preferred mode of sexual contact is more available to more people, so situational zoophilia is much less common (ref).

Zoophiles today are different from Kinsey's farm boys. Those people engaging in sexual contact with non-human animals are much more likely to be pursuing it as a sexual preference. The number of zoophiles is small, and so congregation via the internet is common. The influence of the internet is the biggest difference between Kinsey's sample and today's zoophiles: the farm boys may have been influenced by a culture where sex with non-human animals was common, however this did not define the community (ref). Today's zoophiles congregate on the basis that their sexual orientation is an important part of their identity (ref).

And that's fair enough. Recent research strongly suggests that zoosexuality is a legitimate sexual orientation, a conclusion reached for homosexuality only in the late 20th century. I've written an article on this topic here on [adjective][species] -- Zoophilia in the Furry Community -- so I won't repeat myself. In short, studies over the last decade show that zoosexuals meet the requirements for a legitimate orientation in terms of sexual preference, fantasy behaviour, and love and affection (ref).

Zoophilia as a sexual preference will apply to some of the 15\% (or so) of furries that self-identify as zoophiles. As I mentioned earlier, this question was probably interpreted by most responders as relating to sexual attraction only. This isn't the common definition outside of furry: non-furry zoophiles tend to differentiate between `bestialists' -- those who engage with non-human animals for sexual gratification only -- and true zoophiles, who are concerned with welfare, perceived consent, and the sexual gratification of the animal (ref). However some of the furry 15\% will meet the definition of zoophilia as a sexual orientation, and this number will be significantly higher than the general population, optimistically estimated to be 1\% (ref).

Zoophilia is therefore a furry issue because zoophiles are a significant and visible part of our community. Like other unusually prevalent features of our community, as explored here in [adjective][species] -- homosexuality, Asperger's syndrome, fluidity of gender, online relationships -- the presence of zoophiles helps create and inform the wider furry culture.

Researchers into zoophilia have also made a connection: furry is included as a subset of zoophilia under a classification system proposed in 2011 (ref). Furries are specifically included as `Class I Zoosexuals', along with other people who engage in psuedo-zoophilic human-animal roleplay (e.g. pony play). Alarmingly, the author suggests that furries might be `treated' through behaviour modification therapy.

Fortunately, doctors and therapists are very accepting of unusual sexual behaviour. Even in the event that the classification of zoophilia formally includes `members of the furry fandom', it's highly unlikely that any form of treatment would be administered by a halfway-competent doctor or therapist. As things stand today, treatment is not generally recommended for any zoophiles (or other paraphiles). Any furry, or zoophile, seeing a therapist with a different opinion should strongly consider finding a new therapist.

\begin{quote}
  A final disclaimer: I am not a zoophile. I have never had any sexual contact with non-human animals and I've never had any desire to. This disclaimer serves two purposes: (1) I don't want to be subject to the abuse that zoophiles are often subject to; (2) I don't want this article to be seen as self-justifying. Thanks for reading. It's not an easy topic.
\end{quote}
