\articlehead{Are You an Introvert or an Extrovert: The Quiz}{JM}{2013}

Last week I wrote an article titled Are You an Introvert or an Extrovert? It was written partly in response to a new definition of introvert that has cropped up in the last five years or so, where introverts are loosely defined as people who `gain energy' when alone and `expend energy' when around other people.

It's a compelling way of looking at things, and it's helped people shift books with titles like \textit{Quiet: The Power of Introverts in a World That Can't Stop Talking}. The author of that book (Susan Cain) gave a TED talk exploring the idea, and it's been loosely adapted into webcomics and other sharable media. It has been a successful meme.

People find it easy to identify as an ``introvert'' using this new definition. My article was about how such self-diagnosis can be harmful, but I don't want to repeat myself here. I think that labels are important, but that some labels are damaging. (Previously, I tackled another potentially harmful label, which is also subject to rampant self-diagnosis within the furry community, in an articled titled No, You Don't Have Asperger's.)

In my enthusiasm to talk about labels and self-identity, I failed to define what ``introvert'' actually means. This article remedies that oversight, and talks about how introversion ties into the furry condition. And, yes, there is a simple one-question quiz at the end which will help you understand where you sit on the introvert-extrovert spectrum.

In general, introversion is a tendency to be internally focussed, as opposed to externally focussed. So if you are lost, consulting a map would be an introverted act, whereas asking for directions would be an extroverted act. People who are introverted can be shy (and extroverted people can be outgoing) but this is not always the case.

Modern psychology uses a personality model that originated with our good friend Karl Jung. Personalities are measured using a model called the Big Five, which considers there to be five key, measurable personality traits, one of which is Extraversion\footnote{Blame America Dept.: In American English (which is the basis of Big Five jargon), ``extrovert'' and its derivatives are spelt\footnote{Note to Americans: this is what the rest of the English-speaking universe uses instead of your provincial neologism ``spelled''} with an ``a'', as in ``extravert''. I accept that there are spelling differences in American English (and that American English is often more logical) but why oh why change ``extrovert'' but not ``introvert''? It makes no sense.}. People fall somewhere on a spectrum, with ``very introverted'' and ``very extroverted'' at the extremes.

Anyway. Deep breath.

Researchers prefer the Big Five because the measured personality traits (of a single person) don't change much with mood, time of day, or any other factor. People change in personality up to about age 30, and are pretty much fixed beyond that point. (Clinical research on personality is underway with furries as well: the International Anthropomorphic Research Project uses the Big Five.)

The Big Five has replaced Myers-Briggs as the personality model du jour, but the difference is only really important if you're a researcher. Most people are more familiar with Myers-Briggs (that's the one that tells you you're INTP, or whatever), and there are a lot of simple, free, multiple-choice Myers-Briggs quizzes hosted around the internet (like here). These quizzes are reasonably useful: no substitute for an assessment by a professional, but better than, say, a quiz on OkCupid titled \textit{Which Power Ranger Are You?}

None of these personality models make any reference to gaining/expending energy in social/non-social situations. The idea that an introvert, say, expends energy in social situations and then must `recharge' has nothing to do with personality, as least from a scientific point of view.

We humans are social beings. Yet socializing, or even being around people, can be stressful. Non-verbal communication is a huge part of the social experience, and we rely on body language and other subtle social cues, which require mental processing and accordingly a lot of conscious and unconscious effort. It can be exhausting, and it's worse if we're somewhere unfamiliar, or if we're feeling anxious. So meeting new people in a foreign place can be tiring, while watching TV at home with a close family member is usually easy.

It's worth adding that all humans have a need to socialize, to some extent. The amount of social contact required for mental health varies from person to person. Happily, we live in a world where social contact is easy enough to find (online, for example), so it's rarely a problem, at least among the computer literate.

The idea that we expend energy in social situations isn't clinically meaningful, but it is useful as a tool to help us think about ourselves. There is a lot of value in thinking about ourselves and our own behaviour; this is one of the ways we grow and improve. I think that the ``energy model'' of socializing helps us understand our unconscious motivations (although I think that ``introvert'' as a label can be harmful).

We consider ourselves to be furries, which means that (for most of us) we perceive ourselves as animal-people. We create versions of ourselves from scratch, each of us with at least one (virtual) physical body and (virtual) personality. And research from the IARP (link) suggests that our furry selves are significantly different -- indeed, happier and more mature -- than our non-furry selves. I think that furry can be seen as an exploration of who we really are. I think that we are, collectively, doing ourselves a lot of personal and mental good.

A therapist will often use a simple personality test as a tool. This might be a Myers-Briggs test, or a question like ``if you were an animal, what animal would you be?'' The therapist's intent is to get the client thinking about themselves: a follow-up question might be ``what is it that attracts you to wolves?''\footnote{Furry joke answer: ``foxes, duh''}

In a therapeutic environment, there isn't any real value in personality profiling. The therapist doesn't care that you're ENTJ, or that you feel you would be a macro silver wolf centaur with thunderbolts in your fur and teardrops that taste like Irn-Bru. It's just a conversation starter. Yet it's a very useful tool in the therapist's kit: therapy is a lot more than ``just conversation''.

Furry gives us a framework to continually converse with ourselves. We can challenge ourselves with new ideas, we can road-test behaviour, we can think and rethink who (or what) we really are. Furry can be a kind of self-administered therapy. We can think about it ourselves (if we are feeling introverted) or we can chat with others (if we are feeling extroverted). We're a group of very lucky animal-people.

\secdiv

\subsection*{Are You an Introvert? A One-Question Quiz}

Question 1:

Think back to a time where you emotionally reacted to a negative event. This may have been a break-up, or the death of someone close to you, or a sudden health scare. Pretend you are watching a video of yourself during this difficult time.

Watch the video and observe how you cope. Do you spend time on your own, trying to manage your thoughts? Or do you look for support from other people, in person or online?

Undoubtedly you did both. Both are always required, for all people.

If you (mostly) unplugged your internet and refused to answer your phone, you are more introverted. If you (mostly) sought help from others, you are more extroverted.
