\textit{Fly Away Home} (1996) is the third of director Carroll Ballard's four great animal films. I've discussed \textit{The Black Stallion} (1979) in a previous article; \textit{Never Cry Wolf} (1983) and \textit{Duma} (2005) will come later.

\textit{Fly Away Home} is the story of a tween girl who becomes the de facto mother to a gaggle of young geese. Like Ballard's other three films, it's entertaining, great fun, and -- in partnership with cinematographer Caleb Deschanel -- a spectacular exploration of the beauty of his animal stars. It's a grown-up movie that can be enjoyed by children.

But is it a furry film? And what makes a film `furry' anyway?

A quick peek at the Ursa Majors and the regular `Furry Movie Award Watch' features on Flayrah indicates a strong preference for animated children's films in the vein of \textit{Kung Fu Panda}. That's fair enough: anthropomorphic characters have always featured in children's entertainment, including many of the reference points for the early furry fandom (\textit{Robin Hood}, \textit{The Lion King}). Anthro characters are rare in films intended for adults.

The [adjective][species] chronicler of the international spread of the furry community, Zik, asked for furry movie recommendations recently on Fur Affinity (link). He guessed, correctly, that people would recommend Robin Hood and Fantastic Mr Fox, and little else. So where are the rest of the furry films? Many furries, of course, enjoy those movies that are designed for children, but -- as a community -- we don't seem to have a go-to oeuvre of grown-up furry movies in the way we do books or TV.

I'd argue that any film with non-human protagonists might be considered furry. Aside from the Carroll Ballard films I'm writing about here, you could include Avatar or even the Alien films in that group. Certainly, both those films will appeal to some furries on a deeper level. \textit{Fly Away Home} does too: even though our geese are not anthropomorphic in any way, they are presented in a way that invokes the close personal bond that some people have with their pets.

The movie starts, as Ballard's films often do, with death. The 13-year-old main character, Amy, is riding in a car with her mother. The only sound is music, a new version of the old folk standard, 10,000 Miles. As it turns out, 10,000 miles is the approximate distance between Amy's parents. She is with her mother just outside of Auckland, her father (and the rest of the film) is in rural Ontario.

\textit{Fly Away Home} starts in New Zealand because Amy is played by Anna Paquin. Paquin is best known nowadays as Rogue from the X-Men films: in 1995 she was 13 and had just won an Academy Award for her debut film role in The Piano.

Good child actors must be hard to find. Casting Paquin meant that the film needed to be rewritten to explain (excuse?) her New Zealand accent. The choice of 10,000 Miles to accompany the prologue presumably also came after Paquin was cast.

Amy's mother dies in the crash, and Amy flies to join her father in the Canadian countryside. She then discovers some goose eggs abandoned due to land clearing near her new home, and we have ourselves a coming-of-age story, as Amy plays Mother Goose to a gaggle of ludicrously cute goslings.

This all happens in the first few minutes. As a viewer, it's already obvious that this is a film with no filler. Ballard doesn't like to advance his plot through conversation: in interviews he has dismissively referred to spoken exposition as `yak scenes'.

The film also stands out for having a female lead character. It shouldn't be rare to see a coming-of-age story starring a girl, but sadly it is. This was noted by one of our commenters on my \textit{Fantastic Mr Fox article}, who noted ``It is harder to appreciate some of the truly spectacular stories that explore masculinity because at this point it's largely the only option.'' (Thanks Ju.)

Amy's father is a thoroughly irresponsible parent. On Amy's arrival, his first act is to take a dangerous ride on a hang-glider of his own invention, leading to the first of the movie's many crash-landings. He is the one who allows -- and facilitates -- the idea that Amy should fly from Ontario to North Carolina to lead her geese in their first winter migration. It's obvious that this is a terrible idea, but he encourages Amy to risk her physical safety and frames the journey in a way that risks her emotional safety as well.

Fortunately her journey, and various antics leading up to it, are hilarious and beautiful. Like all of Ballard's films, his animal stars and the countryside are spectacular. He is also smart enough to avoid the worst of the potentially treacly story, keeping Amy's adventures engaging and light.

\textit{Fly Away Home} is an apolitical film, but it presents a fascinating moral take on the world. Consider these two lists:

\begin{itemize}
  \item Good guys in \textit{Fly Away Home}: academics, the American military, hippies, TV stations.
  \item Bad guys in \textit{Fly Away Home}: capitalists, park rangers, hunters, the Queen.
\end{itemize}

In the film's morality, these two groups are delineated in a simple way: you're good if you go about your business without negatively affecting others; you're bad if you assert your own desires as more important than someone else's. It's a kind of libertarian philosophy, but one where monolithic government-provided service providers -- local TV news; scientists; the military -- are considered to be essential to the social fabric. Those who blindly apply rules (aka `laws') are not looked upon kindly.

It's refreshing to see a film where the United States is morally superior to Canada, to say nothing of the moral superiority of the US military over the Queen (!).

Having said that, the film is too black-or-white at times. Storytelling efficiency is clearly a higher priority than believable minor characters, some of whom are drawn in broad shorthand: the moustache-twirling evil property developer at the end leaps to mind.

The physical comedy in the film is often too broad as well. There are a couple of slapstick scenes backed by slightly Bennyhillian music, that feel as if they have been accidentally edited in from a bad children's film. It's a pity, because Ballard creates some hilarious moments: not wacky misunderstandings in various states of undress, rather a baby goose falling into a toilet.

If you can forgive these small mis-steps, you have a brilliant film. Ballard's scenes showing tiny goslings running around in Amy's footsteps are some of the best he's ever shot. And the scenes of the adult geese flying alongside Amy's ultralight aircraft are stunning.

\textit{Fly Away Home} is easy to enjoy and packs an awful lot into 103 minutes. It's cheap too: I bought my DVD from Amazon, delivery included, for less than £1.50. You'd be crazy not to.

\secdiv

This is the second of four articles on the films of Carroll Ballard. All four movies are great. Choose your species and join us:

– \textit{The Black Stallion} (horse)
– \textit{Never Cry Wolf} (wolf): coming as soon as I find my DVD following a recent move
– \textit{Fly Away Home} (goose)
– \textit{Duma} (cheetah): coming eventually
