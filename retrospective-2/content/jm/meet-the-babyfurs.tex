\articlehead{Meet The Babyfurs}{JM}{2013}

Babyfurs are a significant part of the furry community, but they tend to exist below the surface. It's common for babyfurs to create two identities: a clean identity for use in the furry community at large, plus a second identity for socializing with the babyfurs. So there isn't much leakage from the babyfurs into the furry mainstream.

The babyfurs that are visible within furry largely fall into one of two categories: the charismatic types who are able to express their babyfur nature without it overwhelming their identity; and the laissez-faire, who are overt and often less-than-subtle. The rest of the babyfurs, the silent majority, are staying hidden.

There is a dilemma for this silent babyfur majority, those who want to express their identity honestly but choose to moderate such expressions in the furry mainstream. On one hand, they would like to be open; on the other, they don't want to be subject to abuse.

And there is a lot of abuse aimed towards babyfurs from the furry mainstream. Most people reading this will be aware of the stereotypical antisocial babyfur, and will probably have heard some second-hand horror story about something that happened at a convention that one time.

Happily, I'm here to report that the stereotypes are wrong. The mainstream treatment of babyfurs is unfair and largely unfounded. This article is about the real babyfurs.

A few weeks ago, [adjective][species] published an article titled How To Be A Babyfur. In this article I investigated some of the challenges facing babyfurs, but the main point was an attached survey. The survey was shared around by babyfurs on the usual social networks, and (at the time of writing) we had 351 responses. My thanks to those who participated, particularly those who took the time to provide some extra comments. I learned a lot.

We've collated the results. While the data isn't statistically significant, it shows some clear trends. I've read through the various extra comments, and I've followed up with a few extra questions for some respondents. I believe I have enough information to write a brief but broad summary of the group. Ladies, gentlemen, in-betweens… meet the babyfurs:

\subsection*{1. Babyfurs Are Indistinguishable From Regular Furries}

(All comparison data is taken from Furrypoll.com.)

The median age of the babyfurs is 24, compared with 22 for all furries.
This difference is insignificant, and easily explained by the fact that [a][s] probably attracts a slightly older audience. The babyfur age distribution looks like the furry age distribution: a group of people around age 20 with a long tail. The youngest babyfur respondent was 14; the oldest 55.

\begin{itemize}
  \item The babyfurs are about 80\% male, the same proportion as all furries.
  \item The babyfurs are gayer: 44\% compared with 22\% for all furries.
\end{itemize}

This is an expected result: men are kinkier than women, and there is a sexual element to the babyfur identity for many (but not all). More on this later.

\begin{itemize}
  \item About two-thirds of babyfurs live in North America. The remainder are spread about the usual furry hotspots: Western Europe, Scandinavia, and Australasia.
\end{itemize}

Those from non-English-speaking nations are undoubtedly under-represented, because the article and survey are in English. Population data from an online survey is always subject to significant error so I won't present any comparison data, but in general this result is similar to what I'd expect for the overall furry group.

\subsection*{2. Babyfurs Are Very Social}

I was pleasantly surprised to discover that babyfurs are social beasts. As expected, a large majority socialize in babyfur communities online (83\%), however there is also a lot of in-person socializing: a full 55\% of babyfurs have attended a real-world babyfur event, and 34\% have attended a non-furry AB/DL (Adult Baby/Diaper Lover) event.

Very clearly the stereotype of babyfurs as being socially-averse is false. The babyfurs that are poorly socialized are simply the easiest ones to spot. (Thus invoking JM's Law: the most visible members of a minority are rarely the best ambassadors.)

\subsection*{3. Babyfurs Like To Wear Diapers}

This may sound rather obvious, but results show essentially universal agreement: babyfurs are wearing diapers. It is fair to say that ``babyfur'' is synonymous with ``furry diaper lover''.

The term ``babyfur'' is actually a bit of a misnomer, because many babyfurs do not engage in ageplay, or anything else that called be called babyish. Which brings me to…

\subsection*{4. There Is A Minor Schism Between Ageplaying Babyfurs and Diapers-Only Babyfurs}

As far as conflict, aka furry drama (TM) goes, this one is very minor. Everyone seems to be happy enough to be lumped together in the broad category titled ``babyfur'', however there are clearly two main subgroups.

Those who like to wear diapers, but don't engage in ageplay, often prefer to be called ``diaperfurs''. This is an important point (thanks to those who pointed it out), although use of the ``diaperfur'' term is not universal among diapers-only babyfurs. Some are happy to be labelled babyfurs; others see a big difference, to the point that some weren't sure whether the article and survey were intended for diaperfurs as well as ageplaying babyfurs. (It was.)

Conversely, some ageplaying babyfurs prefer to be called ``kidfurs'' or ``littlefurs'', which indicates that their babyfur identity is age-regressive. I suspect that the special delineation is largely for convenience, because it helps likeminded ageplayers identify one another more easily. There are also some furries who play as caretakers towards the ageplayers, as either an occasional or permanent preference.

The two groups seem to get along well, and nobody seems to mind being collected under the babyfur banner. Given the important difference separating the two groups, I think that the overall spirit of fellowship is rather generous.

The ageplayers have a special challenge: they are flirting with one of society's great taboos, the sexualization of underage characters. For many babyfurs, ageplay has a sexual component, an interest that (partly) drives demand for cub porn. And this association sees some furs make an easy but completely unfounded leap: they accuse babyfurs of paedophilia.

The vague association of ageplay with paedophilia is one reason why some diaperfurs don't like the babyfur term. It's not because they think that there is any connection, just that they know that some people make that connection, and that they'd rather not be tarred with that particular brush.

The connection between ageplay and paedophilia is wrong. But it's an easy connection to make. I can think of one case where a furry convicted of paedophilia-related crimes turned out to be a babyfur, and I think that it's reasonable to guess that furry paedophiles are fairly likely to be babyfurs. However the correlation only works in one direction: it doesn't mean that an ageplaying babyfur is likely to be paedophile.

Consider that violent criminals are likely to enjoy violent video games. But people who enjoy violent video games are not likely to become violent criminals.

For the doubters: if you are uncomfortable with cub porn, or feel that there must be some correlation between ageplay and paedophilia, please (1) consider that people don't choose their sexual interests, and (2) read my article from last year, In Defence Of Cub Porn.

It's a controversial topic, and not one I want to explore in any detail here. It's only tangentially relevant to the subject at hand, and I think it risks overwhelming the main points. Suffice to say that it is false to suggest that ageplayers are doing something ‘wrong'. Which brings me to…

\subsection*{5. Babyfurs Are Unfairly Demonized}

The babyfur group as a whole -- ageplayers, diaperfurs, and the rest -- are routinely accused of being anti-social or having poor hygiene. The stories are often exaggerated, and usually completely false.

One astute babyfur noted that watersports is a relatively visible fetish within the furry community. While watersports fetishists are subject to a certain degree of kink-shaming, they are far less likely to be demonized in the way that babyfurs are. I can only surmise that diapers suggest age regression (regardless of whether of not ageplay is taking place), giving diapers a faint whiff of the taboo.

I'll add that unfair demonization of babyfurs occurs, to an extent, within the babyfur community itself. Some diaperfurs unfairly dislike ageplayers, much in the same way that some furries unfairly dislike babyfurs as a whole.

There are, of course, some bad eggs. Every group, including the furry community, has some outliers.

One of the (intelligent, moderate) commenters on my article reposted some comments to a Fur Affinity journal\footnote{http://www.furaffinity.net/journal/4877653/}. Here's part of a comment he received, from a user named Bondagepup:

\begin{quote}
  Having to smell someone's stale piss-pants in public is also not the end of the world. Ever been in public? People smell. (Old people especially.) Just hold your nose and move on people.
\end{quote}

Bondagepup argues that he's merely expressing himself, and that he should be free to do so:

\begin{quote}
  Lastly, just a note to anyone who is offended by seeing anything they deem sexual in a public setting, your moral code is not law. Just because you were taught that sex was naughty and needs to be hidden doesn't mean it's true.
\end{quote}

Bondagepup thinks he's being laissez-faire and sex-positive, and he is to a degree. However he is also forcing people around him to engage in his sexual fetish. Sex columnist and ethicist Dan Savage sums up the problem with public fetish play nicely. (I have edited this quote for clarity, you can read the advice in full here):

\begin{quotation}
  Asking people to accept your pastime doesn't give you the right to force other people to take part in it. That's not asking for tolerance, that's demanding participation. And that's not okay.

  Not once in our struggle for social acceptance have gays and lesbians demanded the right to have sex in front of our relatives. We want to be accepted by our families, tolerated by strangers, and treated equally by our government. But people who don't want to watch us have sex aren't compelled to.

  This fetish stuff is, at bottom, about sex.

  Keep the heavy stuff behind closed doors and keep it subtle when you're out in public. That's not oppression, that's common courtesy.
\end{quotation}

Bondagepup is being anti-social. There is nothing wrong with discreetly wearing diapers (or anything else) in public, but there is plenty wrong with being actively smelly. He is reinforcing the negative babyfur stereotype, to the detriment of the babyfur community as a whole.

The overwhelming majority of babyfurs are discreet. They are not noticed by the mainstream because they are respectful of those around them, and because they understand the boundaries of reasonable behaviour.

As ever, the most visible members of a minority are rarely the best ambassadors.

Consider this final statistic: only 35\% of babyfurs have ever taken the simple, reasonable step of displaying a babyfur conbadge. Which means that there are two (or so) stealthy babyfurs for each conbadge you see. Next time you're at a convention, take a headcount.
