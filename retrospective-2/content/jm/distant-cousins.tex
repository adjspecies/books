\articlehead{Distant Cousins}{JM}{2013}

This article is a counterpoint to Rabbit's article published last Friday, Not-So-Distant Cousins.

Rabbit argues that furries and mainstream SF fans have a lot in common, that the two groups are similar enough such that ``we should be treating each other as respected and beloved cousins, if not brothers and sisters.''

For evidence, he cites a common geekiness, a shared private language, a similar culture, and finishes by drawing a parallel between fursuiting and cosplay. He says:

\begin{quote}
  We're all fen together, is what I'm trying to say. Natural allies, not rivals. I mean, how many places can you find people who not only enjoy discussing terraforming over barbeque, but are good at it? Not many, in this sad and intellectually-declining world.
\end{quote}

I disagree.

I need to be careful when I'm talking about SF and related fandoms. I've gotten a bit of grief from fandom insiders in the past about things I've written in these virtual pages, most recently when I delved into the psychology of My Little Pony. I get accused, by geeky fans, of being dismissive towards fandoms, or belittling, or elitist. (I suspect that some people would be unhappy with my use of the term `geeky fans', but I think it's clear enough.)

I don't mean to be negative towards fandoms, SF or otherwise. I've never been involved with a fandom other than the furry community, so my perspective is that of an outsider. The value of a fandom is self-evident: if they didn't have value, nobody would bother. And fandoms are full of great people too, although I wouldn't go so far as to claim they are any better endowed with the good and the great, any more that the rest of the world. I've certainly met, mostly within furry, excellent people who are also fans. Rabbit counts among that group.

There is important history between furry and SF fandom: furry started its life as a distinct phenomenon as an offshoot of SF fandom. But this is furry's history, not furry's present. Nowadays furry is a stand-alone phenomenon, a community of people drawn together not by fandom of pre-existing works of art, but by a common perception of identity. We see ourselves in anthropomorphic animals, we think of ourselves through the lens of atavistic behaviour, and most of us choose to socialize in a half-imaginary world, as if we really were an animal-person.

``Hi I'm JM, I'm a horse'' is very different from ``Hi, I'm Matt and I like Star Trek''. Furry is personal: fandom is social.

Rabbit's article is built around a great anecdote. He shares a terrible meal with a group of furries, talks about geeky topics, and has a whale of a time. He points out that the experience could just as easily been that of group of SF fans.

And perhaps it was a group of SF fans: lots of furries geeky sci-fi lovers, including some paleofurs (a great term I've gleefully stolen from Rabbit) who have been around since there was much less physical and philosophical distinction between the two groups. A full 60\% of furries responding to the Furry Survey (now curated here at [adjective][species] as www.furrypoll.com) categorize themselves as ``a fan of science fiction''.

Geekiness and SF fandom is a big part of the furry experience. But it's only a fraction of furry culture, and it doesn't define who we are. It's the furry identity that binds us together.

There is a reason why we furries are drawn to the community, and it's related to our internal world, not the external world that drives fandom. The furry experience isn't easy to summarize, but I think it's one united by introspective, personal things: our predilection to re-evaluate our sexual preference\footnote{About 60\% of furries will consider themselves heterosexual when they discover furry; that number drops to 30\% after five years. (link)}, our non-mainstream sexual identity\footnote{For example, about 15 to 20\% of us are zoophiles (link). For further evidence, ask your friends about their f-list.}, our non-mainstream gender identity\footnote{About 20\% of us identify as something other than completely male or completely female. (link)}, our connection to the idea of transformation, our animal-person roleplay\footnote{These are all just examples of course, and won't apply to everyone.}.

Yesterday I had an experience comparable to Rabbit's SF-filled meal. I visited an old Tudor house on the outskirts of London with a furry friend, to stroll around the grounds in full spring flower, see an animal-themed sculpture collection, and have our very own terrible meal. Like Rabbit, we had a great time, and it had nothing to do with a potato and leek soup that had clearly been made using powdered `french onion' soup mix.

(The exhibit, `Beastly Hall', runs until 1 September 2013 at Hall Place \& Gardens, Bexley, Kent.)

As we walked, we chatted about the furry experience. We talked about furry's demographics, our collective reaction to death in the fandom, our sexual interconnectedness, the politics of uncommon sexuality, the experience of travelling overseas to meet a love interest, fursuiting, roleplay. We also took non-furry conversational diversions into areas of mutual interest. And I'd argue that's what happened with Rabbit's group: they talked terraforming because the group shared a mutual interest, one that happens to be related to SF fandom.

Rabbit also says that we furries share a private language with SF fans. He cites `fen' (meaning SF and other geeky-type fandom members), `mundane' (anyone else), `SMOF' (secret master of the fandom), `gafiated' (gotten away from it all), and `fafiated' (forced away from it all) as examples.

I've spent a lot of time as a furry and I can honestly say these terms, bar one, are new to me. Of them, I learned fen from a previous Rabbit article, and the other three are completely new, and actually kinda perplexing (what does a fandom master do, and why is it secret? what is the `it all' that people might get away from? is that a good thing?). And the one term known to me—mundane—makes much more sense in a furry context. Compared to an animal-person, regular human beings seem totally mundane. I'm not sure I'd say that about someone compared to a geeky fandom member.

(Lest that final sentence seem too negative, please keep in mind that I'm an outsider to fandom. From what I have learned, within fandoms there often seems to be a wilful rejection of the outside world, a shared belief that being inside the fandom is something special. While I'm sceptical of the value of a group that implicitly rejects the outside world, I'm sure fandoms are spiritually fulfilling places. A bit like an Amish community, perhaps, but with better laptops.)

Rabbit is someone who straddles the furry community and sci-fi fandom. His mundane name is Phil Geusz, probably furry's most successful author, and one of our community's biggest names. His books are in close touch with the introspective furry experience: they dive deeply into what it means to be an animal-person, to be a furry.

The artifices of the various Geusz universe are often sci-fi, with technology such as genetic engineering providing an animal-person minority in a human population. But despite the sci-fi trappings, Rabbit writes quintessentially furry books. His themes are the thoughtful, introspective ones of furry.

Rabbit writes about religion: the spiritual aspects of furry (see The First Book Of Lapism). He writes about how it feels to be lost inside a furry skin, be it the intelligent nuance of his rabbits (see Ship's Boy, which is free on Amazon) or the flighty pride of a cheetah (see Cheetah's Win, collected in Roar \#2). He is using sci-fi as a framework for a furry construction, something maintaining the logic of his worlds but otherwise rather beside the point. And that's how I see SF within furry: it's everywhere but it's not relevant to the true furry experience.

As a final point of contention with Rabbit's article, I also don't agree that we live in a ``sad and intellectually-declining world''. If that were true, nobody would be buying his books.

\secdiv

I came back from my trip to the animal exhibit to find a small group of furries, laptops out, playing Civilization and talking about programming philosophies. We're a geeky group for sure. But it's not what defines us.
