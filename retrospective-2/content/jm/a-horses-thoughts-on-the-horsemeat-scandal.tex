\articlehead{A Horse's Thoughts About the Horsemeat Scandal}{JM}{2013}

Over here in the UK, there's been an extended brouhaha after many cheap TV dinners, known as `ready meals' locally, were found to contain large amounts of horse instead of the promised beef. Some of the meals contained 100\% Pure Horse.

Nobody knows how long the horse has been there. It only came to light because a branch of the Irish Government performed some DNA tests and announced the presence of our equine friends in mid-January. And it's been in the news since then.

I think it's worth discussing here on [adjective][species] because it relates to our relationship with animals. Also, I'm a furry horse, so I get asked how I feel about horses as a source of meat.

The short answer: I feel ambivalent. The longer, more entertaining answer: I'm fascinated how this scandal has come about, been reported, and -- most importantly -- how my furry friends have reacted, often wildly differently depending on their relationship with their species of choice.

It's easy enough to understand how horse ended up labelled as beef: the European Union, which includes the UK, is an open market and goods (including meat) can mostly be traded freely. In the UK, our big supermarkets compete in a desperate race-to-the-bottom to be the cheapest, regardless of quality. They advertise that a pint of milk is 1p or 2p cheaper. A tin of tomatoes will cost me 30p, but they'll be unripe. I can buy a mass-produced chicken for £2, whereas an ethically-raised one the same size will cost me about £15. The drive for a lower price drowns the desire for a higher quality.

It's the same drive that sees high fructose corn syrup (HFCS) everywhere in the USA. HFCS is sweet and cheap. Nevermind that alternatives taste superior, or that HFCS is metabolised in a way that reduces feelings of satiety: people want to pay less for their soda.

Horsemeat is perfectly legal in the UK. It's also about half the cost of beef and, apparently, is indistinguishable in flavour. The meat used in ready meals is largely sourced from outside the UK, because it's cheaper. And so horse has found its way into beef products via suppliers competing for supermarket business, where the strongest criterion for success is price.

The British press, for their part, have been competing to see who can generate the most outrage. The tabloids have dealt in the usual xenophobia, while the broadsheets look for something -- anything -- that allows them to be upset without dabbling in racism. Is there even a problem with horsemeat? After all, it is routinely eaten, if a bit d\'{e}class\'{e}, in France and elsewhere to no apparent ill effect. So the UK newspapers have decided that horsemeat is unsafe because UK horses are sometimes treated with a painkiller that isn't safe for human consumption\ldots eliding over the fact that the ready meal horses are sourced from elsewhere, and that there is no problem with such chemicals elsewhere in the horse-consuming continent.

I'm sure that there are countless other examples of cross-border intrigue and scandal all over the EU, and I'm sure they seem as equally quaint to disinterested observers.

Some will argue that there is an ethical issue with horsemeat, that it's wrong to eat companion animals, or that horses have special capacity for pain, or fear, or some other form of suffering. These arguments are valid—moral arguments always are. However anyone who hesitates at horse yet pounces on pig -- the porcine are at least the equal of equine in intelligence and companionship (if not HP) -- might politely be called self-contradictory.

Yet the fact remains that horses are special for many people, including many furries. The furry identity is usually attached to a specific species (or two), and some horse furs have a special affinity for their pony pals. For rhetorical purposes, I'm going name such hypothetical horse hangers-on as Gullivers, after Swift's eponymous traveller who ultimately shuns human habitats for exclusive equine esteem.

Altivo, one of our favourite commenters here at [adjective][species], is a Gulliver. I think/hope he'll have his say in the comments, so I won't speak for him here. If you're interested, he's written eloquently on the topic in his journal.

Another furry friend of mine is a more vehement kind of Gulliver. His response to the idea of horses as food:

\begin{quote}
  The whole concept fills me with horror and revulsion, and I have to say I felt suddenly sick at the sight of the topic\ldots I think you know of my professional involvement in animal welfare, and I am not a vegetarian. I know some allege this as hypocrisy, and I know issues such as comparative intelligence and whether animals have names or not are not reasons to discriminate what one eats and what one doesn't. There are, however, welfare issues in the transport and handling of slaughter horses which have a direct bearing. These are matter of scientific fact. [\ldots] Even without these important issues, on a personal level I draw no distinction between eating horse and eating dog or cat, or, indeed, human. I would do none of these things (although personally the idea of eating dog, cat or human horrifies me less), and the very idea makes we want to vomit.
\end{quote}

That response is taken from an old journal of mine, where I pondered the idea of eating basashi, a sort of Japanese horse carpaccio, which was offered to me while visiting Tokyo:

\begin{figure}
  \begin{center}
    \includegraphics{content/assets/horsemeat--basashi}
  \end{center}
  \caption{Basashi}
\end{figure}

(I didn't eat the basashi. I'm vegetarian, contributing my part to the predictable phenomenon that sees furries twice as likely as the general population to avoid meat altogether, as discussed in an [a][s] article from last year.)

Some furries have the opposite reaction from the Gullivers, and actively consume their own species, sometimes as an expression of their furry identity. (Most common, in my experience, among furry deer and bulls.) I haven't come across any horse furs who look to devour horseflesh, however those in UK looking to express themselves in such a way have more options nowadays: the scandal has seen horse openly introduced to menus across the country, as pubs and restaurants cater to the curious.

There is nothing wrong with being horse-curious, no more than there is being vegetarian, or being a Gulliver. For those that think about it at all, meat is a moral issue, by which I mean that it's unreasonable to apply universal definitions of right and wrong. There are cultural norms and politics at play here: imagine the hypothetical reactions among people you know to eating dog, or guinea pig, or scorpions. The consumption of animals -- living, breathing, tasty things -- provokes strong responses in many people. The righteous might keep that in mind before they start telling the rest of us how to think.

It's the thinking that's important. And I'm interested to hear your thoughts below.
