There is a display of religious pamphlets outside Liverpool Street station, which I pass on my stroll into work each morning. A recent pamphlet title: Pornography: Harmless or Toxic?.

The pamphlets are being peddled by Jehovah's Witnesses, a well-funded American-based group that attempts to practise Christianity as it was 2000 years ago. They are probably best known for refusing all blood transfusions, including those that might be life-saving, because ``the Bible prohibits health treatments or procedures that include occult practices'' (ref jw.org).

I, like most people who don't subscribe to the JW's very special brand of stupidity, am pro-pornography. So I think to myself ``pornography is harmless''. But I'm wrong, because I can immediately think of examples where pornography is harmful. And so I wonder if the JW's might be on to something. (Spoiler: they are not.)

I've been caught into a logical bind because I've tacitly accepted the premise of their question. They have cleverly phrased their title, drawing on a trick used by salesmen and interviewers everywhere: by offering up two competing categories, people are drawn towards one or the other.

And so it is with the title of this article: Are You An Introvert or an Extrovert? You, dear reader, almost definitely chose ``introvert''. You did that because I wanted you to. In reality, the label of ``introvert'' can be a harmful one, and it is probably a label you should reject. Let me explain why.

Labels are useful things because they help us understand ourselves, and help us explain ourselves to other people. On the downside, they do not always allow for nuance or change.

We furries like to label ourselves. We often do so in an online profile, perhaps in a Twitter bio or Fur Affinity userpage. I encourage you to take a look at your own labels before you continue.

Here's my profile, which I wrote, from the [a][s] About page:

\begin{quote}
  JM is a horse-of-all-trades who was introduced to furry in his native Australia by the excellent group known collectively as the Perthfurs. JM now helps run [adjective][species] from London, where he is most commonly spotted holding a pint and talking nonsense.
\end{quote}

I've labelled myself three times. I am a horse-of-all-trades, commonly spotted holding a pint, and commonly spotted … talking nonsense.

I know that labels are important, and so I've refrained from being too direct. ``A horse of all-trades'' is pretty vague, and my other two labels are qualified with ``commonly spotted''; they are things that I do, not things that I am.

Now let's look at Kyell Gold's [a][s] profile:

\begin{quote}
  Kyell is a fox, a writer, and a California resident. He likes to write stories of varying lengths, often (but not always) dealing with gay relationships and foxes.
\end{quote}

Kyell is much more direct. He has applied three strong labels to himself: fox, writer, and California resident. I suspect that these terms are internalized, which means that Kyell considers them to be part of his identity.

A ``fox'' is a good label, because Kyell is free to make and remake himself in that image. A few weeks ago, Makyo and Klisoura did some datamining and published the results here on [a][s], exploring the words that people use when describing their fursona. As you might expect, they vary considerably, although there are some trends. When foxes describe themselves, the most common terms include cunning, sly, and cute. And so we can guess that Kyell might use such terms to describe himself, but in the end he will have a unique relationship with his foxly self.

I'm not sure that ``writer'' is a good label for Kyell. It's certainly accurate, but this might change in the future. If Kyell were to, say, experience an extended bout of writer's block, he might find this label—this identity—to be problematic. How often does Kyell have to write for him to identify as a writer?

The same goes for ``California resident''. Again, it's mostly accurate, but what if circumstance sees Kyell spend an extended period of time out-of-state? This label may be a mere statement of fact rather than important to Kyell's identity, although I wonder if Kyell the Oregonian would feel quite right.

When a label becomes part of your identity, it can be limiting. Kyell, for example, might be inclined to turn down an otherwise positive relocation to Oregon, because it could force him to rethink his own identity. A bad label can be self-limiting, and it can provoke an identity crisis.

To use an example that isn't Kyell, consider a brand new furry who considers himself to be straight. Let's call him Straightfox. Straightfox finds furry to be an environment that doesn't have society's stigma on homosexuality, and he—like so many of us before—is interested. But Straightfox, because of his identity as heterosexual, has a problem. He can either:

\begin{enumerate}
  \item Refuse to participate in any homosexual activity, or;
  \item Rethink his identity.
\end{enumerate}

Neither of these options are easy for Straightfox. Those many, many furries who re-evaluated their sexual preference after discovering furry (a group which includes me) know how difficult it can be. Straightfox, like all before him, would have been better off if he never considered his sexual orientation to be important to his identity.

There are similar problems if you identify as an ``introvert''. It's an attractive label, but it's self-limiting.

``Introvert'' is an attractive label because it's in opposition to the unattractive label ``extrovert''. If asked to conjure a mental image of an extrovert, most people will think of someone acting like a Dallas Cowboy in the 1990s: hyper-social, overbearing, and lacking any sort of introspection or internal narrative.

Furries are especially prone to this because we tend to be analytical, with lively inner lives. Furries are thoughtful, creative, and often a touch depressive. It's easy to look at other people, especially other people in a social environment, and wonder if they have any personal doubts and fears. It's easy to conclude ``I'm not an extrovert like all these people''.

Extroversion, then is about actions, especially social actions. And introversion becomes a label about inner thoughts. We, each of us, know that social actions make us anxious and uncomfortable and scared. Everyone else, even a coked-up Dallas Cowboy in the 1990s, is also anxious and uncomfortable and scared. But we aren't privy to anyone inner world except our own.

(As an aside, I think that there is a clue to the furry condition here. We are a group of individuals who are prone to feeling alienated from society. This doesn't mean that we are necessarily rejected by the world, it means that we are made to feel as if we are different from those around us; as if we were a different species.)

Someone who identifies as an introvert is tacitly accepting the premise that they derive limited enjoyment from social activity. They may decide that the stress of socializing always overwhelms the positive aspects, or that they simply do not have the social knack. Both of these may be true, but such an identity doesn't allow for nuance or personal growth.

In reality, social skills improve with practice. Nobody enjoys small talk; nobody finds small talk natural. But we engage in it because it provides a non-aggressive entry to conversation, and we get better at it with time. Someone who thinks they are introverted might assume that they will always fail at small talk, and so they stop trying, and never learn the skill.

The marketing world has picked up the popularity of ``introvert'' as a label. It's now a sales pitch, along the lines of ``if you are introverted then you must read these three tips on how to improve relationships with your workmates''. It's the same marketing premise as diet books, except that it's aimed to the socially anxious rather than the body-conscious.

Here are a few examples, all books marketed towards people who label themselves as an introvert. Notice how the titles encourage you to identify as an introvert, by suggesting that ``everyone else'' is an extrovert:

\begin{itemize}
  \item Quiet: The power of introverts in a world that can't stop talking
  \item Introvert Power: Why Your Inner Life is Your Hidden Strength
  \item The Introvert Advantage (How To Thrive In An Extrovert World)
  \item Introvert's Way: Living a Quiet Life in a Noisy World
  \item Quiet Influence: The Introvert's Guide to Making a Difference
  \item Energized: An Introvert's Guide to Effective Communication
\end{itemize}

And the books marketed towards extroverts? There aren't any. Nobody identifies as an extrovert. Not even a Dallas Cowboy in the 1990s.

The supposed dichotomy between introversion and extroversion is false. They are not mutually exclusive; you do not need to ``choose one''. In my Jehovah's Witness example, pornography is not always harmful or always toxic; there are elements of both. Similarly w≠e are all introspective to some degree; we are all social beings to some degree.

Labels are important, but ``introvert'' is a bad one. You can be introspective without undermining your ability to socialize.
