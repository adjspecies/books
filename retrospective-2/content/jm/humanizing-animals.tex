\articlehead{Furry Research: Humanizing Animals}{JM}{2013}

Furries play a starring role in a 2006 paper that explores `animal geography', an emerging field of cultural research related to human-animal interaction. The paper's author believes that furry phenomenon is on the leading-edge of changes affecting society as a whole: the replacement of human-human social contact with human-animal social contact.

The paper, written by Dr Heidi J. Nast and published in ACME, is titled ``Loving\ldots  Whatever: Alienation, Neoliberalism and Pet-Love in the Twenty-First Century'' (link to full text). If that sounds like tortured prose, then, well, you should read the article itself. It's not easy going. But hidden under the unwelcoming academic language is a fascinating perspective on the furry phenomenon.

Nast's point mirrors one I've made in a previous article, Furry As An Alternative To Religion. She notes that traditional community structures -- archetypically the rural village church -- have broken down in the modern world. People have moved into cities, lost connection with the people around us, and this has left us feeling alienated and alone. It's a sad irony that many people feel lonely, while simultaneously being surrounded by other human beings.

I argued that furry provides that missing sense of community, and Nast makes a similar argument although she sees furry as one example of a wider cultural shift. She thinks that people are projecting human characteristics onto animals, as compensation for a lack of real human contact.

Nast sees this happening most obviously in the first world's growing trend for pet ownership. She argues that domestic animals are much less likely to be working animals, and much more likely to be a humanized `member of the family'. Pets are de facto children to many people, offering a big advantage over real, human children: pets are less inconvenient. She writes:

\begin{quote}
  \ldots pets (especially dogs) today supersede children as ideal love objects; they are more easily mobilized, require less investment, and to some degree can be shaped into whatever you want them to be
\end{quote}

Nast points to a growing marketplace for inessential pet `care' as evidence. If she were writing her article today, she might also point towards the tendency for people to create a social media presence, like a Twitter feed, on their pet's behalf. And she argues that people are spending time and money on animals, instead of spending that time and money on humans.

The time and money being spent on non-humans is also institutional, including scientific research and charity. Cats can be cloned (for a price); you can take your pooch to a `dog psychologist' (for a price); urban animal welfare is increasingly focussed on minimizing euthanasia (at a cost to human taxpayers). Nast suggests that this time and money would be better spent on minimizing human suffering.

Nast feels that, by humanzing and infantalizing animals, we become less connected to other humans. She goes further to suggest that this is linked to consumerism, where animals are a convenient replacement for human beings because the relationship is uneven. We can, essentially, spend money on our non-human family without having to worry about whether it's useful in any way. As Nast puts it:

\begin{quote}
  \ldots the hypercommodification of pet-lives [and our]\ldots  post-industrial lives and places\ldots  [are] tied firmly to neoliberal processes of capital accumulation more generally and the attendant growing gap between rich and poor.
\end{quote}

Which sounds a bit like something you might read on an Occupy Pet Warehouse flyer.

To put it in a less tortured fashion: Nast sees our human-like engagement with non-human animals as evidence for the inhumanity of a capitalist world.

The furries fit into her argument because our human-to-human contact takes place through an animalistic lens. We are humanizing (virtual) wild animals and using them for our own ends. As she puts it:

\begin{quote}
  In the case of furry fandom, humans [present themselves as animals], this transmogrification apparently being needed in order to facilitate human contact, sociality, and love.
\end{quote}

Like the people who humanize their pet dogs, we furries are focussed away from human society. We focus on ourselves, or on the part-human versions of our fellow furries, or on non-humans altogether.

Furry, in Nast's eyes, is a product of our dehumanized capitalist world. We socialize through the guise of animal-people because our world doesn't allow us to (easily) directly socialize with human beings.

\secdiv

Now that all sounds like Nast has gone off the deep end. But plenty of evidence from the furry world supports her ideas.

Firstly, ever notice how much easier it is to interact with a fursuiter than the person inside? Most of us (and many non-furries) find it more natural to initiate social contact with the animal-person.

Secondly, furry's spread throughout the world broadly correlates with deregulated capitalism. First in the USA in the 1980s, then other modernized western nations such as the UK, Australia and Germany in the 1990s, then the remainder of Europe and South America in the 2000s, and more recently capitalist Asian nations such as Singapore, Malaysia and Japan.

Thirdly, we furries are relatively alienated from greater society. That's because, as a group, we often don't meet society's norms: perhaps it's because of unusual sexuality, or geekiness, or distaste for mainstream culture. This alienation reduces our engagement with fellow human beings.

That's not to say that Nast gets everything right. She lumps furries into three broad categories:

\begin{quote}
  \begin{itemize}
    \item egg-heads with more or less intellectual interests in how and why a society or group anthropomorphizes animals
    \item furries [who] assert a particular animal identity, either playfully or believing that they were animals in a former life, or that they are an animal trapped in a human body
    \item persons erotically and/or sexually invested in their animal-identity
  \end{itemize}
\end{quote}

It's not hard to poke holes in her categorization, an exercise I leave to the reader.

She also asserts that furry ``involves largely `white' adult populations''. While mostly true, this misses the point: furry is not a monoracial phenomenon, as evidenced by its spread across the world. However I can see how she could draw this conclusion from her happily unscientific data collection method: looking at ``photographs of furries reproduced on various websites''.

\secdiv

The biggest flaw in Nast's ideas is, I think, her willingness to tie everything back to capitalism and consumerism. She presents it as a fait accompli, which I suspect is normal for academics performing research in the field of cultural geography. I don't want to explore the validity of this point of view -- I'm sure that readers will hold a range of strong opinions -- but suffice to say that I don't think Nast makes a compelling link.

To be fair, her focus may be geared toward the sensibilities of the journal that published the paper: ACME. ACME has the following mission statement, which you read at your peril:

\begin{quote}
  The journal's purpose is to provide a forum for the publication of critical work about space in the social sciences  --  including anarchist, anti-racist, environmentalist, feminist, Marxist, non-representational, postcolonial, poststructuralist, queer, situationist and socialist perspectives.
\end{quote}

So ACME is not exactly aiming for political moderation.

As an aside, check out ACME`s unintentionally ironic guidance for prospective authors: ``The style that ACME advocates emphasizes clarity, accessibility, and care in writing.''

Happily, Nast's article is written to a higher standard than that. However it's not an easy read by any means. So I can't really recommend it, despite its worthwhile and unfamiliar approach to the furry phenomenon.

Dr Nast is writing a book on the topic: Petifilia: Volume 1. Presumably furries will make another significant appearance. I'll read it with interest.
