\articlehead{How To Be A Babyfur}{JM}{2013}

So, you're a babyfur.

I know, I know: you're not one of \textit{those} babyfurs.

You probably like wearing diapers. You probably find that you can unwind and relax when you're doing childish things. You have probably found that, as time has gone on, you've started incorporating ``adult baby'' elements into parts of your life -- clothing, accoutrements, roleplay -- to add to your enjoyment of diapers.

Or maybe you just find the art cute, and the characters easily relatable. Or maybe it's more of a sex thing. Or maybe you like to watch cartoons and talk in baby talk. Or maybe, just maybe, you have a professional `adult' who looks after you in a nursery once in a while.

In any event, you're probably aware of how other furries react when they hear about babyfurs. They find babyfurs distasteful. And so you probably have a babyfur-only identity that is separate from your `normal' furry identity. Or maybe you just keep it to yourself.

As a babyfur, you probably feel like you can't be open and honest with your local furry group. I think there are more people in your situation than you realize.

I think that there are a lot of babyfurs in the furry community. I don't know exactly how many, because no large survey has ever asked. But I think it's a lot, perhaps comparable in size to the other large minorities we have within furry: the genderqueer, the zoophiles, and the women.

We here at [adjective][species] would like to hear from the babyfurs. We've created a short survey -- which is anonymous and confidential -- and we'd like you to respond. But more on that in a moment.

I have only anecdotal evidence that suggests, to me, that there are a lot of babyfurs out there:

\begin{itemize}
  \item Babyfur events, usually room parties, occur at every convention. Some of these are G-rated exercises in icecream and Power Rangers; some are explicitly sexual; many are a bit of both. These events occur despite being organized via word-of-mouth, and occur despite the perception that they are taboo within the wider furry community.
  \item Real-world AB/DL (Adult Baby / Diaper Lover) events, which occur in some cities, are often full of furries.
  \item Furries who are open about being a babyfur and are also socially presentable often find themselves approached -- in private -- by friends. These equally presentable friends are either curious about what baby-furriness entails, or they are already clandestine babyfurs.
\end{itemize}

I know that all this is true because I have spoken to lots of babyfurs.

I think that babyfurs are suffering from something that plagues many marginalized groups: that the most visible members are not the best ambassadors.

By way of explanation, consider the following thought experiment. For each of the minority groups I'm about to list, imagine a stereotypical member: (1) gay, (2) feminist, (3) Fox News viewer. (I have tried to select three categories with little overlap.)

Chances are that you thought of a pretty normal person for those groups of which you're a member, and that you thought of a grotesque caricature for those groups you tend to avoid. This is a normal response for a couple of reasons:

\begin{itemize}
  \item Humans are naturally distrustful of the unknown. This instinct is the root cause of racism and homophobia, and it takes a bit of mental effort to overcome.
  \item If you're not a member of the minority in question, and nor is anyone in your social circles, you're more likely to have been exposed to the extreme elements of the group; the bad ambassadors. So, outrageous pride costumes inform perspectives on gay people, feminists are seen as angry and intolerant, and Fox News viewers are mindless gun-toting yahoos.
\end{itemize}

This totally instinctual human reaction can be seen in attitudes towards gay marriage in the United States. If you don't know someone gay, you are much less likely to support gay marriage.

From Slate (link):

\begin{quote}
  Research shows that knowing a gay person makes you 65 percent more likely to support same-sex marriage, and having a conversation with that gay person about marriage raises the figure to 80 percent.
\end{quote}

(I should note that this pattern is certainly not restricted to the US, just that it's a been political football, and the Americans love collecting polling data.)

There's some science that suggests that babyfurs, like gay people (and like zoophiles), are more likely to generate a negative reaction. It's a linguistic problem: the sexual practises of each of these groups is suggested in the group's name.

A study published in 2011 sums up the issue in its title: \textit{Disgusting Smells Cause Decreased Liking of Gay Men} (full text, pdf). In brief, the study showed that people felt less warmly towards gay men when they were in a smelly environment. The effect wasn't seen towards other minority groups. Essentially, the smell of poo makes gay men seem kinda gross, because they engage in anal sex.

As the study author commented in Scientific American (link):

\begin{quote}
  I think what's happening is that the social category of ``gay men'' (and to a lesser extent, gay women) is one that is defined by the sexual act… I tell my class to imagine if the first thing they learned about a person is that he or she frequently masturbated to pregnant women. The sexual disgust response would likely eclipse every other aspect of the person, such as their also being a fireman, a pharmacist, or Irish.
\end{quote}

The problem is similar for babyfurs: those people who don't know any babyfurs aren't easily able to create a mental image that goes much beyond the diaper, and the (imagined) smelly contents thereof. And so babyfurs tend to keep quiet about it, because they know to expect an initially negative reaction.

The urban myths that circulate about babyfurs always focus on disgusting behaviour. The stories are inevitably exaggerations, speculations, or outright false. Furry is not awash with people soiling themselves in public or leaving dirty diapers in convention hallways. There is a large minority of babyfurs (perhaps including you, dear reader), and they are being respectful of those around them, and keeping quiet.

This is the point in the article where I say that I am not a babyfur. It shouldn't matter whether I am or not, but I know from experience that it does. My first article about zoophilia for [a][s] (I've written three) was criticized for being self-serving, that I was just trying to justify my own proclivities. I'm concerned that this article will lead people to draw a similar conclusion. There is nothing wrong with being a babyfur, and it's a bit sad that I feel the need to distance myself personally, but unfortunately I think it's the best (or least-worst) course of action.

Which brings me to my slightly hypocritical advice: I think you should tell furries that you are a babyfur. There are a few reasons:

\begin{itemize}
  \item For your non-babyfur friends, you'll be a good example. You will be disproving the kneejerk babyfur stereotype simply by being yourself.
  \item For your closeted babyfur friends -- and you almost definitely have some -- you'll be a rolemodel.
  \item For yourself, because you'll be able to be open and honest with your friends. And that's good for the soul.
\end{itemize}

I met a furry named Karis a few years ago, at a convention. He's charismatic, well-liked, and a generally great guy. He's one of those furries who seems to be forever surrounded by friends. And he's a completely open babyfur.

His baby-furriness was gossiped about when he wasn't present. People were surprised that he could possibly be a babyfur.

Karis was comfortable and happy to answer any questions. He directed people to his website, Karis' Playground (http://www.karisplayground.com/), which features webcomics like World of Wetcraft. He changed the mind of a few people simply by being open, and I'm guessing that there were some closeted babyfurs present. I'm guessing that they felt a surge of joy at seeing Karis being treated respectfully.

And so I recommend that babyfurs be open, or at least relatively open, because I think that there are lots of you out there. You'll have plenty of support.

I also hope that this article helps, because it's never easy to hide a part of your personality. It's mentally stressful, and it's easy to start seeing yourself in the grotesque artchetype: it's easy to be self-hating. It's never healthy to deny a safe sexual urge: it can lead to stretches of self-hatred and denial, interspersed with bouts of sexual mania. (I've written about this before.) Far better, if you can, to accept yourself, respect yourself, and love yourself.

Starting next year, the Furrypoll will have a question asking ``Are you a babyfur?''. In the meantime, we have a Babyfur minipoll, which is anonymous and will not be shared beyond  me, an (anonymous) babyfur helper, and Makyo. The responses will be used in future [a][s] articles. Please participate, and help us get an idea of what the babyfur community looks like. Alternatively you can email me directly at jm@furrynet.com, or just leave a comment below.

Please also share this article within babyfur circles. I'd like to hear from as many people as possible. It's about time that babyfurs were recognized by the furry mainstream.
