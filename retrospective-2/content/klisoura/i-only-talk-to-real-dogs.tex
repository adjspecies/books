\articlehead{Excuse Me, I Only Talk To Real Dogs}{Klisoura}{2013}

``Welcome to the Internet. Where the men are men, the women are men, and —''

Wait, what?

Hang out in the chatrooms that dot the furry landscape, and you’ll find this sentiment expressed not infrequently. Boiled down, it encapsulates the belief that you can’t trust what you see, which is simple enough -- but I’ll suggest that this line of thinking is both inaccurate and also slightly troublesome.

If you’re not a roleplayer, this line of discussion is all somewhat irrelevant to you. But according to the 2012 Furry Survey, more than half of furries \textit{do} engage in roleplaying to some degree, and at some time. This probably isn’t surprising; roleplaying offers a safe space to explore our identities, and it probably goes without saying that furries would gravitate towards this exploration.

It seems to be self-evident that people are willing to accept interacting with people who present themselves as a different species than they really are, and in my experience it’s generally accepted that one’s online sexual orientation can legitimately differ from one’s real-world orientation. So why is gender so problematic?

Well, first of all, what do I mean by ``problematic''?

Quantitatively, we notice a strong aversion to changing one’s sex online: 82\% of people say that they do not do so, with a strong majority (58.5\%) saying they would not do so. Even amongst active roleplayers, 74\% hew strictly to the \textit{biological sex} they were born with -- that is, the remaining 18\% (26\% amongst roleplayers) also encapsulates the (admittedly small) number of transgendered persons who are electing to accurately represent their gender.

Qualitatively, we see statements like, ``I’m not a fan of people who are [girls online but] guys in real life'' -- the backronymic pejorative ``GIRL'' (Guy In Real Life) applies here -- and it is here that we start to see one of the interesting dimensions of the issue, which is that it is expressly gendered and generally heteronormative: far fewer people seem as troubled by the idea that the male winged magic-using bipedal talking sapient fox-wolf mix they’re talking to is actually being operated by a female puppeteer.

We understand, at least to some degree, that furry chatrooms are not accurate representations of reality, as my last description indicates. In my sojourns through the fandom I’ve seen people who claimed to be Russian when they were really American, people who claimed to be lawyers, people who claimed to be thin, people who claimed to have master’s degrees in esoteric subjects…

It’s pretty much par for the course.

So why’s it gender that sets people off? Why not other areas of body image? Why wouldn’t you put in your profile, ``I only want to talk to people who are physically fit in real life''? Possibly because it would seem shallow, and slightly irrelevant for the purposes of light conversation, nondirected roleplay, and typefucking?

Let’s examine some possible answers.

The first is that it’s an inherent dishonesty that is fair to judge people on. That is: if I can’t trust that you’re honest about such a fundamental aspect of your personality, then what can I trust you on? Is it supposed to not matter because we’re talking as two avatars? If we’re only interacting mask-on-mask, then what does anything really matter, anyway?

This seems like a logical statement, until you unpack it a bit. After all, someone’s real-world physical attributes are only actually relevant if you enter every conversation expecting the possibility that your interaction on FurryMUCK could logically lead to a real-world romantic or sexual encounter. Not to put too fine a point on it, but this is a weird, overbearing, and even slightly offputting mindset to start from.

We are, after all, expressly entering into an abstracted, idealized world when we engage with avatars. Even chatroom sexuality is transgressive: we gain the ability to interact free of many of the restrictions and repercussions imposed by the real world. Make the phrasing honest: ``I would like to pretend to be a dog, and for you to pretend to be a red panda-lynx hybrid, and I would like to put some of my pretend bipedal clothes-wearing ambient-music-appreciating dog parts inside your pretend red-panda lynx body but \textit{only if I’d be cool doing that in real life, too.}''

As pickup lines go, it’s a little awkward.

A more interesting objection, though it’s not often phrased explicitly, is the one that boils it down to the unseemliness of straight men pretending to be women so that they can have straight sex, or to otherwise benefit from the attention they would otherwise lack.

So, then. \textit{Fetch me the numbers, Igor!}

On the Furry Survey, I ask about presenting yourself in the fandom as a gender different from your biological sex. Five options are presented:

\begin{itemize}
  \item No, and I would not do so
  \item No, but I might do so
  \item Yes, sometimes
  \item Yes, often
  \item My primary furry avatar fits this description
\end{itemize}

As said, 58.5\% of respondents gave the first answer -- that is, that they ``would not'' do so. When we limit the response to only straight men, that number jumps to 71.6\%. A further 21.4\% of straight men say they don’t, but they might consider it. Straight men are a third as likely to say they do it ``often'' (<1\% compared to 3\% in the general population), and around a quarter as likely to say their primary avatar differs from their own biological sex (1.5\% compared to 5.6\% in the general population).

It is here that we pause to note a couple more things about the prevalence of gender fluidity. Firstly, in a proportional sense it’s substantially more common amongst women; women are 2.5 times as likely to have a male primary avatar than men are to have a female one, and 2.7 times as likely to say they ``often'' represent themselves as a different gender. Only 37.2\% of women say they ``would not'' use a male avatar; 64.3\% of men say they ``would not'' use a female one.

Secondly, it would seem that since straight people are substantially less likely to do, then the slack is made up by those in other portions of the sexuality spectrum. It was suggested that partly this might be because changing genders allows you to explore your own notional homo- or bi-sexuality in interesting -- and safe -- new ways.

But this is an interesting concept, and we’re going to come back to it in a bit.

If we compare those who say they would not and those who say they always present themselves as a different gender, it’s true that there are certain evident differences. For one, as stated, people who always do so are less likely to be straight (22\% vs 43\%), and far more likely to be pansexual (24\% vs 4\%). They’re also three times as likely to be asexual, though -- 11.3\% vs 3.7\%. In real number terms, they make up 5.6\% of the fandom, but 22\% of the fandom’s asexual people and more than a third of the pansexual members.

Outside of sexual orientation terms, they are also, as stated, more likely to be female. They are older, though by less than a year, and have a higher degree of education. They are 19\% less likely to be single and 45\% more likely to be in a long-term relationship.

Their positions on an attitudinal survey tend to be more extreme. People with gender-transgressive primary identities are 46\% more likely to strongly disagree that what other people think of them is important (14.2\% to 9.7\%). They are 50\% more likely to strongly disagree with the statement ``creativity is one of my strongest attributes'' (43.4\% to 28.4\%). They are 88\% more likely to ``strongly agree'' that they are more talented than most of their peers (10.9\% to 5.8\%) -- but also 55\% more likely to ``strongly disagree'' with that statement (18\% to 11.6\%).

They are not appreciably likely to say that sex is more important to their furry identity (average score on 10-point scale is 4.6 vs 4.3), which circles us back to an earlier point. It may seem like I am, to a degree, harping on this, but I think it’s important to note that, from the evidence, people who change their gender online aren’t doing so for sexual reasons.

So what does it tell us if we think they are?

What first drew me to this topic was how closely the discussion recalls classic and unfortunate interactions transgendered individuals are familiar with. As I said to start with, because the question discusses presenting an avatar different from your biological sex, a small number of those people are transgendered persons -- but most of them are not, and I am certainly not going to suggest that gender dysphoria is the primary motivation.

But, in furry chatrooms and roleplaying environments, you see the same classic scripts playing out. You see the same troubling, parochial belief in ``traps'' -- people who are disingenuously trying to mislead straight men into a life of… well, certainly a life of something, anyway, and evidently something more problematic than simply pretending to be a tiger. You see the same stigma attached to gender transgressiveness, particularly in the notion that people make the choices they do because they would be relationship-unsuccessful otherwise (a statement that is demonstrably incorrect).

You even see hints of ``trans panic,'' with people discovering ``the truth'' about their conversational partners attacking them, belittling them, and engaging in other behaviors that are designed to reinforce a gender-normative worldview. I ran a roleplaying chatroom for nine years, and I cannot count the number of times, as a moderator, I had someone breathlessly ``out'' someone to me.

``Oh, bloody hell,'' you are sighing into your scotch. You wave the waiter over to bring you your check, shaking your head and muttering: ``Here they go on about transphobia again.''

Well.

\textit{Yeah}.

I’m willing to call this out because, as I said, it seems to be equally parts silly and troubling. I have yet to see a clear articulation of why it should be acceptable to change your species but not your sex that doesn’t boil down to balky circumlocutions around the fundamental issue that people still see gender as immutable and transgenderism as the slightly skeevy hallmark of second-class persons.

That is to say, I don’t see a clear articulation that doesn’t either hem and haw around that issue or reveal a hell of a lot more about the speaker than you’d initially suspect. As I said, your conversational partner’s real-world gender is dubiously crucial if you enter into conversations expecting the possibility that you intend to engage with them in real-life sexual contexts, but that’s a can of worms all on its own.

As the New Yorker‘s Peter Steiner once famously quipped: ``on the Internet, nobody knows you’re a dog.'' Anonymous communication involves striking a careful balance between respecting the freedom that comes from constructed identity, and being aware of the assumptions we make in our interactions with others.

It’s clearly something that we’re uncomfortable with: anonymity invites its own destruction, and the Internet takes a singular pride in denying of others the right to be anonymous, or to choose on their own terms what they present. And when gender roles come into play, we run headlong into traditional discomfort with people who don’t play by the rules. Hence the invention of new stereotypes, irrespective of whether they are actually accurate -- and I have no doubt that some of you who have gotten this far are thinking: yeah, but I know people like that.

Perhaps.

But these seem to be edge cases, and the thing that strikes me about the dim eye turned on those with gender-transgressive identities is that casual chauvinism is still chauvinism, and bears reflection. The fandom has an established and positive legacy of being supportive of all types of self-exploration. How peculiar -- and slightly sad -- it would be if this is one of the last to enjoy the legitimacy of existing unexamined and uncriticized.

Because in all probability, insisting you will only talk to real dogs is a losing game, of dubious reward.
