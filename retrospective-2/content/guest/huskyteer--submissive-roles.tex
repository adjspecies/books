\articlehead{Submissive Roles: Writing for Furry Anthologies}{Huskyteer}{2013}

Writing in the furry fandom is seen as the poor relation to art, and there are plenty of articles discussing why that is so. Here's where we writers score over the artists, though: the pool of furry authors is tiny by comparison, and you could count those operating at a professional level on the pads of one paw. The downside: the market, and the rewards, are correspondingly small. However, most creative furries, whatever form their art takes, get started out of love for the craft, and because they have stories to tell, rather than in the hope of fame and financial reward.

Many furry fiction writers cut their teeth on fanfiction (\textit{The Lion King}, in my case) before moving on to tales of original characters posted on Fur Affinity or SoFurry. Although the internet provides a way to get your work in front of a huge number of people, from all over the world, in a matter of moments, some of us still crave an appearance in traditional print publication.

One way to make the leap from screen to page is to contribute a short story to a furry anthology. Based on my own experience, I'll describe about how to go about it, how the process works, and the pros and cons of anthology writing.

An anthology is a collection of short works -- fiction, non-fiction, poetry, or a mixture -- by a number of authors. It can be a one-off special or a regular annual publication, like \textit{Heat} and \textit{New Fables}, both published by Sofawolf Press. Anthologies may have a generalised theme (\textit{Heat} is for romantic or erotic works of all kinds) or a more specific one, like Sofawolf's \textit{X}, whose stories are each based on one of the Ten Commandments.

Anthology editors solicit contributions by putting out a call for submissions, which will include guidelines for the type of story required and a submission deadline. These days, you will usually be asked to submit by email rather than posting a physical manuscript.

Editors are as keen to receive material as writers are to get it published, but it can still be difficult to find open calls for submissions. Keep an eye on the websites of furry publishers -- Sofawolf, Furplanet and Rabbit Valley -- as well as the Paying Venues page of the Furry Writers' Guild. Following writers and editors on Twitter or keeping up with their blogs will also give you an idea of what's happening in the world of publishing. Established writers may even be contacted by the editor and asked to contribute a story. That's a pretty big compliment. (Still waiting, guys!)

The good news is that a small pool of active writers means a greater chance of acceptance. But it's likely there will still be many more submissions than slots, so how do you improve your chances of getting in?

First and foremost, follow the guidelines. That means a final word count somewhere between the minimum and maximum, if specified; obeying any requests for particular file types, margin sizes, or fonts; keeping within any specified themes or restrictions; and submitting before the deadline. This all sounds like common sense, and it is. But you will be putting yourself ahead of the pack straight away, simply by reading and following the rules.

Writing something truly original is a good way to get your work noticed. Granted, this is a bit of a tall order. But if an anthology themed around fairy tales receives twelve stories based on Red Riding Hood, only one or two are going to make it in no matter how brilliant the rest. Avoid plot clichés unless you can give them a really good twist, and read as widely as you can in the genre so you don't accidentally come up with something too similar to an existing work.

Before submitting, get a friend to read through your story. Whether or not they're a writer, they will be able to tell you what does and doesn't work for them, and spot the kind of typing and grammatical errors that slip past computer spellcheckers. Fresh eyes will also be better than yours at spotting plot holes, or character names that change halfway through.

When you're ready to submit, include a short covering letter with your story. Sometimes the call for submissions will tell you what information is required; if not, give your name, a brief synopsis of the piece, and any other writing credits. Usually you'll receive a brief acknowledgement to let you know your email has arrived safely.

What happens next? A lot of waiting -- during which you keep yourself busy writing the next thing and refrain from bothering the nice editor -- followed by either acceptance or rejection. (There is a third option, in which the anthology simply never materialises for one reason or another. It's a shame, but it happens.)

Rejection is disappointing but doesn't have to be a disaster. If the editor has taken time to include critique, then treasure it, even if it makes painful reading, and think about putting any suggestions into practice next time.

Sometimes the story just isn't right for the publication, or is too similar to another submission. Try submitting elsewhere. Stories that didn't quite make the grade can be rewritten or revised before dispatch to a different publication. Still not quite there? Post to your SoFurry or FA account as a freebie.

Even if you decide your story isn't fit for public consumption after all, chances are that something can be salvaged from the wreckage and recycled for use elsewhere – whether it's a plot point, a character, or a particularly good piece of dialogue. At the very least, words exist now that didn't before you wrote them. With every word, and every rejection, you're practising and improving your skills, both in writing and in \textit{being a writer}.

When your story is accepted, it's a wonderful feeling. Allow yourself to bask and gloat; you've earned it. But what happens when the glow has worn off a little?

A few weeks or months later, you will probably be asked to revise your story in a few small ways suggested by the editor. You may approve all changes straight away or you may want to argue for your original version. The process can be as quick as approving placement of an apostrophe, as complicated as writing a whole new scene, or as heartbreaking as deleting one. Once both you and the editor are happy with the final version, you'll sign a contract, either electronically or on paper to be posted back.

More waiting follows, this time for publication day. You should receive a contributor copy or two ahead of release to the general public. When the anthology launches, it's time for you to promote it -- on your blog or website, Facebook, Twitter, and perhaps by posting an extract to your favoured furry art site (best to check with the editor first).

There are plenty of advantages to writing for anthologies. If you're the kind of writer who has difficulty actually sitting down and writing -- like most of us -- a cold hard deadline can be a big help. Brainstorming ideas to suit a theme can be inspirational, too, taking your muse in new directions.

During the publishing process you'll gain experience of the editorial system, and start to build relationships with editors – which will stand you in good stead when it's time to pitch your novel.

Many writers find publicising their work much harder than the act of writing itself, since we're a bunch of self-doubting introverts. Appearing in an anthology immediately increases the potential audience, with all the contributors plugging the book to their friends, families and blog readers. If a big-name author is involved, their fans will probably pick the anthology up as a matter of course. All this means an anthology credit is a great way to get your work, and your name, in front of new readers.

Finally, one entirely self-serving advantage: contributor copies mean you get to read other writers' work for free.

There is a downside to anthology writing, too. The pay tends to be small, whether it's a flat or per-word rate, and may even be nonexistent, with contributor copies the only reward. Themes, deadlines and word counts can be restrictive as well as inspirational. You may find yourself disagreeing with the editor over requested changes.

To sum up the disadvantages, an anthology represents an editor's vision, not yours. It's not your baby and you have no or little control over the layout, price, font, or, where applicable, the artist chosen to illustrate your work. But if your ultimate ambition is to publish a full-length work or story collection of your very own, writing for anthologies can help to build invaluable experience, contacts, and even a fanbase.
