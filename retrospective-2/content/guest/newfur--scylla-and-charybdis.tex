\articlehead{Scylla and Charybdis (or, the Art of the Lie)}{Newfur}{2013}

Reading through JM's and Makyo's posts here got me to thinking about the nature of presentation and the public face, and from there to one simple tenet I have lived by. Having lived in many ways and in many places, from my often rocky relationship with my family and cloistered visits to famously conformist South Korea, to my time in the furry fandom and in other such gloriously tolerant places, at some points I have had to live by this tenet by force, but thankfully more often, by choice:

The face we show one person need not be the face we show all people.

I know, I know, plenty of ink's been spilled over ``authenticity'', and we're told from childhood that ``honesty is the best policy''. In the end, though, only the demands of others to know all they desire about you and childhood conditioning drawing from this motivate these feelings. It is, in the final estimation, relatively easy to stay true to yourself while, in a calculated manner, making sure that some people don't see what you don't intend for them to see. At first blush, this may seem like a suspect notion, even one indicative of sociopathy. Think for a moment, though: does anyone really want to hear about what you had yesterday for lunch?

The telling of such lubricative lies and stories along with the careful holding-back of information is integral to this bewildering phenomenon called society, regardless of how we feel. Naturally, this sort of lie is crucial because it allows people to interact with each other without the abrasion that would result if we were as true and open as many purport to seek to be. We build up, consciously or otherwise, a public face: a mask that we show others in place of our true face. Those that fail to do so, whether through a lack of capability or desire, generally end up marginalized: out of work, shunned by others, or, occasionally, dead.

As furries, then, we find a tactic used time and again in the animal world: camouflage. Camouflage of dress, of speech, and of action. These aspects of disguise are the tools and the material with which we build the edifice that we intend others to see. They form the public face, as furries or even just as people. For if we must craft a mask just to function in society, we must craft it as best we can, rather than just letting the cards fall where they may. Hoping blindly that we'll be accepted, no matter how awful it feels to choose to deliberately manipulate the emotional reactions of those we meet, will do us no good.

Balance is required, though. We find in ancient Greek mythology an apt picture of our dilemma: that of Scylla and Charybdis, the famed great hazards of the Strait of Messina. Scylla, it was told, was a terrible sea monster with six heads, and Charybdis, a terrifyingly deep and powerful whirlpool. And so too for us: to one side, the Scylla of too little -- indiscriminately wearing and doing what we please anywhere and at any time, the wearing of collars at all times or barking at strangers -- threatens us with ruin. More dangerous to my mind, though, is the Charybdis of too much, of selling your soul and losing what makes you delightfully unique, of wearing business clothes everywhere because you've forgotten what anything else feels like. Judiciousness is required, too. A three-piece suit would be almost as out of place at a furmeet as a collar would be at a job interview.

We draw inspiration, then, from the delicate art and subtle science of the lie. To put it poetically, a well-crafted lie is a story told about an alternate world which is easily confused with one about the real world. So too for our camouflage of word and deed : a story told in speech and movement of someone who is an ordinary member of society, about whom it cannot be said that something suspicious or untoward is going on. We must be careful, though, of Charybdis again: the lie bites both ways, and it isn't unheard of to wind up living the lie you have accidentally convinced yourself you believe.

So what's to gain from all this trouble taken? It seems to me that the presentation of a false face to the world is the only way to balance your true self with the demands of society, and in so doing guard yourself from society's assimilative pull. Even better, it allows you to hide your true preferences from someone who might seek to use them against you.

It can attract friends, true ones, too: people will enjoy knowing you more if you still have the decency to treat them properly as a fellow human being even if you don't really feel personally invested in the way you're interacting with them. And oftentimes, those you meet out on the fringes as they sail carefully between Scylla and Charybdis, or even find some small success taking Odysseus's famed daredevil route, will be some of the best friends you'll ever make.

It's hardly as bleak or as do-or-die as I've said here -- at least, not most of the time. Society is often forgiving: in practice, people have a Somebody Else's Problem filter for the strange and unusual, which those like us can gleefully exploit. A good example of this is the noted example of the residents of every college town whose residents have become inured to strange happenings, have seen it all, and now don't even blink at the strangest happenings. The important thing is to keep the transgressive and the truly strange out of view, which seems to be the rather successful tack the furry community as a whole is consciously taking, and that many other subcultures have taken with varying degrees of consciousness and success in the past.

And thus in the end, it's important not only to know the rules of the world, but also to know when you can bend or shatter them. A life lived placidly within imagined and self-imposed boundaries is barely a life lived at all. It is, after all, only in the lighting of a little non-conformist lantern that you can signal to those other interesting strangers that you're someone worth talking to. Just don't let it turn into a ship-consuming bonfire, of course, and you should be fine.
