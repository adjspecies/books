\articlehead{Evidence that Furry is Following the Rest of the World}{Amethyst Bassilisk}{2013}

One of the best benefits from participating in a creatively chaotic community such as furry is the ability to be whoever you'd like to be. It's an important outlet for many of us -- our expressions tend to come out in terms of being who we feel we're not necessarily allowed to be from a greater cultural perspective. Most of us didn't fit in, wherever we came from. Most of us were too geeky -- too awkward, even. Too loud and boisterous, too strange or too tweaked. As a result, we've fled to and cultivated ourselves a safe haven from emotional treachery. The only explicit laws against fantasy in most cultures are typically put in place to prevent fraud and violence. However, there also exists a social hierarchy which takes every opportunity to reinforce one's alleged place in its expansive machinery. A plethora of societal and financial pressures as well as generalized threats on survival are applied in order to enforce this order whenever possible. A lack of order, as the host species appears to feel, is a formula for destruction.

Furries balk at this thought. Furries pretend to be whoever the heck they want to be, regardless of what others may think of them -- at least non-furries, anyway. The community attempts to shed the societal pressures; intellectual disdain; and hatred toward experimentation; to craft crafts of provocative proportions. As a result, furries are outcasts for the things they enjoy doing, allegedly hated by the rest of the world for what's perceived as anywhere from fun to enlightenment.

One of the sad ironies of cleverly crafted utopias are their abilities to mimic and even amplify the societal sundries they're attempting to flee. Furry is Schr\"{o}dinger's Island.

Schr\"{o}dinger's Island is a neologism piggybacking on the quantum parable of Schr\"{o}dinger's cat. Its use attempts to describe a social phenomenon wherein participants of the furry community simultaneously attempt to receive recognition from the greater culture that has allegedly rejected them and equally boast-- sometimes to arrogant degrees -- about how separated they are from the culture as a whole (e.g., aggressively boasting about furry pride to a group who is perhaps ignorant or just doesn't particularly care). Essentially, on one hand, actively identifying as a furry is an emotional protest toward the arbitrary boundaries enforced on them in a culture that rejects such sense of freedom to self-identify. Without the element of societal rejection from a puritanical society, it's very difficult to argue that furry would be what it is today. And yet on the other hand there's this intense craving to be accepted for who we are -- hence the hand-to-forehead magnetism induced by the more stereotypically vocal among us.

This psychological schism doesn't have a tendency to exist in one person: it's a social superposition which causes emotional projections of what brought us here in the first place, leading to bizarre circular arguments, self-fulfilling prophecies, and a naively malevolent darkness that allows for all sorts of horrible abuses to happen. To summarize: what is a hero without a villain? And what happens when the hero yearns for villains to confirm their heroic existence? Will they find villains they didn't know of before, or will molds be formed to redefine their villainy? Let's go with a common (and easy) villain of the puritanical mindset, as well as the reputation of the furry community as a whole: sex.

Statistics regarding the opinions on sexual psychology in the furry community make this quantum ideology stand out like a sore thumb. Frequently, when furries are polled, they feel the rest of the furry community is way more sexualized than they are personally -- which is to say not very much, of course. A cynic would argue blatant hypocrisy, whereas an easy counter-argument is citing statistics regarding artistic production: there are always, on average, much more general audience productions than erotic productions on a given furry site, at a given furry art show or in a given furry artist alley. Yet, as furries, we know a dead horse when we see one, so let's shirk the stud-shank. There's an interesting curiosity in having sex panic in the first place. Let's talk about sin.

Sin isn't simply biblical. All cultures have sins. Sins are a means of social control enforced by those who feel emboldened and uplifted by their followers. What the leader deigns, the followers enforce, creating exiles in their wake and loyalists for the cause. There are many cultural sins -- sex is just one of them. But sex has been used as a means of social control back before the bible, so to declare sex as a weapon of the puritan would make Ghengis Khan cackle.

Sex is used in the furry community in a similar way -- though mostly to control the perception of what others think of us. Attempting to portray an opposing opinion on the sexuality of furries…does not usually end well. Trying to enforce it as a good thing usually creates strong opinions one way or the other regarding what others will think of them. It's rather difficult to have an intelligent discussion about sex in mixed company -- and this isn't even really a furry problem! But the push-back in talking honestly about sex touches on one of the more cardinal sins of the furry community.

Picture a parade. Fursuits and other costumes cruise through. Everything is normal. Try and picture one of the costumers striding past you while performing in character, utilizing the physical mannerisms of their costumed persona to really bring that outfit to life. Then imagine them locking eyes on you with excitement, pulling off their fursuit head and striking up a conversation with you while the parade is going on behind them. What is your perception of the crowd around you?

The first perception that comes immediately to mind is annoyance, irritation and fingered, judgmental murmurings from the rest of the crowd. This fursuiter has sinned. They ruined the fantasy.

But this sin goes beyond fiction. This sin in particular -- the ruination of fantasy -- has deep, deep roots that go to the core of our emotional utopia and reach further into the puritanical society we feel exiled by, even through attempts to exhume the very thing many feel furry provides as a shelter from attack. To present furry as anything potentially adult to others is a sin: that ruins the fantasy of furry being acceptable by society, and its members as acceptable by proxy. To state that public furry groupings could potentially be an unsafe place is a sin: that ruins the fantasy of furry as a safe-haven from the greater villainy we ran from. To make claim of potential malevolence, be it sexual, violent or psychologically manipulative by another furry, is a sin: that ruins the fantasy that furries are friendly and affectionate. Analogies apply equally to the puritanical society furry attempts to escape.

Yet this inability to even question the fantasies that exist within the furry community itself -- combined with the Internet's infinite appetite for being proven permanently right -- allows the very abusive villainy furries fled from to flourish. As is standard in the society furry left behind, the survival response to prevent one's self from becoming either insane or isolated becomes apathy. A cascade of anti-intellectualism occurs due to a rising desire to become the most apathetic of a given group, and from there the cycle of psychological (and, though thankfully relatively rarely, physical) abuse continues to tumble along in the darkness.

The movie Inception provides a fascinating metaphor for the horrifying phenomenon caused when one becomes consumed by the fantasy world. As the protagonists dig deeper into the dreams of their victim, they go further and further into the many layers of dreams, with limbo being the final layer. But limbo is infinity. Limbo is simultaneously everything and nothing. Limbo is exactly what you make of it. And its perfection is as much beautiful as it is horrifying. Beautiful in its ability to create exactly what you want and how you want it; horrifying from the existential stress imposed from over-pressing the dopamine depressor.

Inasmuch as we may be leaders by actively experimenting with our newfound freedom to abstractly identify ourselves, there exist strengthened and well-enforced cultural patterns from this identicraft that caused us to disobey standard social order in the first place. Yet instead of attempting to address these patterns, they are culturally shoved under the nearest rugs and dismissively declared as drama. In that regard, the furry community unfortunately follows the beat of the same oppressive drum this escapism attempts to shut out: protect the fantasy of order at all costs.

The social order of furry is founded primarily on the quality of fantasy. It is our culturally accepted vice. But what makes us followers instead of leaders in this regard is our inability to give our fantastical society a strong foundation by questioning and maintaining its fantastic structure. Leaders fight for their followers with bold and courageous tactics -- not propaganda.
