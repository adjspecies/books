\articlehead{Furry Research: A Response from the IARP}{Nuka}{2013}

Hi there! I read through (and quite enjoyed the insight in) your recent article and felt compelled to provide my take on things, (keeping in mind that Dr. Gerabsi's article pre-dates my involvement with the furry research). Given that I'm in the lucky position of being at the forefront of our team's research, I may be able to provide another perspective on this issue.

I'll structure my comments in a point-by-point fashion.

\begin{quote}
  The most obvious problem is the use of the word `disorder'. This implies that there is some sort of problem. Gerbasi seems to be pathologizing furry, or at least a large subset of furry.
\end{quote}

I both agree and disagree with your statement. I do agree that the use of the term `disorder' is (and has always been) problematic in this regard. I disagree, however, that the term has pathologized furries, for reasons I'll get into in a moment. But to start, I'll agree that `disorder' is problematic in its use here.

The stance of the IARP is that furries are neither inherently pathological nor is furry, in and of itself, a disorder. Just like a hobby, a religious belief, or an aesthetic preference, there is nothing ``wrong'' with furry. Where furry, like anything else, becomes problematic is if, and only if, it begins to cause clinically significant distress for the people who engage in it.

From Dr. Gerbasi's perspective, she did the right thing for a scientist: she simply asked furries to indicate whether, as a result of the belief that they were not entirely human (for the ones who believe this), there was a persistent feeling of distress or discomfort with one's physical body. Such items were drawn from a Gender Identity Disorder (GID) scale because this research had been asking an analogous question, albeit in a different domain, on the subject of felt conflict between one's mental representation of self and their physical body. Asking the question ``could there be something analogous to GID for furries?'' is not the same thing as claiming that Species Identity Disorder (SID) exists, that it's something that needs to be ``cured'', or that all furries have it -- it is a necessary part of testing the hypothesis that some furries may experience discomfort or distress analogous to what those with GID feel -- even if only at subclinical levels.

The data reveal that, for at least some furries, they do report significant levels of distress regarding the apparent disparity between how they see themselves in their mind and their physical bodies. Dr. Gerbasi did not claim that furry itself therefore pathological, nor did she go on to claim that furries who do experience this distress are therefore diagnosable. Instead, the intention was always to merely raise awareness that within this fandom there may be some who experience this distress to a significant degree, and were this the case it would be worth taking this person's distress seriously (for the most part, any clinician hearing about furries might otherwise dismiss it as so much nonsense). The majority of furries do not experience a discrepancy between their ``felt species'' and their ``actual species'', and the majority of those who do feel this discrepancy do not experience distress at anything resembling a clinical level. The point was only to illuminate the possibility that there may be some (and, in fact, there have been cases) who do experience significant enough distress because of this discrepancy to seek clinical assistance.

To summarize the point, Dr. Gerbasi was not intending to claim that all -- or even some -- furries ``had'' any condition; it was merely to entertain the notion that, given the content of the fandom, there may exist some who experience a discrepancy between mental representation of self and physical self that is comparable to that seen in GID, and that for some this discrepancy may cause clinically significant distress.

\begin{quote}
  \ldots the 2011 Furrypoll, which was completed online by over 4000 furries, showed that about 11\% of furries consider themselves either non-human or part-human. This is a long way from Gerbasi's 46\%.
\end{quote}

More recent numbers across five different samples over two years (with numbers as large as 4,500+ furries in some of the samples) suggests that between 25-45\% of furries consider themselves to be ``less than 100\% human''. These numbers are not incompatible with those of Furrypoll, as the questions asked were different: Furrypoll asked whether people considered themselves ``non-human'' or ``part-human'', while we asked if a person felt ``less than 100\% human''. A person who felt ``95\% human'' may answer ``no'' Furrypoll's question while still answering ``yes'' to our question. Neither question is ``more right'', they're simply placing the threshold for ``human/non-human'' in different places. Our measure is more sensitive to any feelings of being non-human, whereas the Furrypoll question seems more sensitive to the distinction between feeling non-human to a significant enough extent that it changes the label you apply to yourself. Put another way, Furrypoll's numbers may be a better measure of ``being a Therian'', whereas the IARP's question is a better measure of the presence or absence of any feeling of being non-human.

\begin{quote}
  Dr Fiona Probyn-Rapsey, who disagrees with Gerbasi: ``There are a myriad of reasons why furry participants at a furry conference might identify as ``less than 100\% human,'' not the least having a hangover from furry drinks the night before.''
\end{quote}

I feel that the argument I made above better accounts for the discrepancies between Furrypoll's numbers and our numbers than the argument that a convention environment is somehow meaningfully different. We have done a number of studies where the same survey was administered both at a convention and online, and have found only minor differences (primarily having to do with age of sample or available financial resources), and almost no differences with regard to how ``furry'' a person was or their inherent ``human-ness''. Frankly, I felt Dr. Probyn-Rapsey's statement about ``furry drinks'' was insulting and trivializing to those who genuinely feel not entirely human. This insult was further compounded by the fact that Dr. Probyn-Rapsey's research was based on relatively little contact with furries (in comparison to Dr. Gerbasi's continued treks to Anthrocon and non-stop dialogue with hundreds of furries in any given year), and Dr. Probyn-Rapsey seemed more focused on attacking GID than on actual concern for furries.

\begin{quote}
  Dr Probyn-Rapsey challenges Gerbasi's tentative diagnosis of `Species Identity Disorder' directly: ```What might be the ``treatment'' for such a condition?''
\end{quote}

It should be pointed out that Dr. Gerbasi never claimed that ``furry'' was something to be ``treated'' -- this is a straw man built up by Dr. Probyn-Rapsey. Dr. Gerbasi only claims that for those furries for whom the feeling of being discomfort with one's physical body is causing significant distress, it might be worth considering it somewhat analogous to the way one would address a person who felt discomfort with the gender of their physical body. There is no attempt to ``normalize'' non-furry or to ``pathologize'' furries, only to state the very obvious: if a person's particularly troubled by not feeling human and being trapped in a human body, it's worth taking seriously.

What makes Dr. Probyn-Rapsey's point so ironic is the fact that the IARP has recently had an article accepted for publication in a clinical psychological journal where we appeal to psychologists to avoid pathologizing the ``furry'' in furry clients, who are often seeking a clinician for completely unrelated reasons (e.g. depression, anxiety issues, etc…). Our team's stance is that ``furry'' is not pathological, not unless an individual furry feels that being furry is causing them distress.

\begin{quote}
  Probyn-Raspey's biggest problem is Gerbasi's link between `Species Identity Disorder' and Gender Identity Disorder.
\end{quote}

There really is no link intended between GID and ``SID'' beyond the fact that it served as a convenient analogue for comparison and a source of some existing questions to ask. Dr. Gerbasi had no intent of validating or discrediting the diagnosis of GID, or of debating the merits or worthwhile of it as a diagnosis. Instead, she was merely observing that it was a condition that clinical psychologists recognize, and that a comparison might be made between folks experiencing distress over their felt gender and the gender of their body and a furry who was experiencing distress over their felt species and the species of their body.

To avoid beating a dead horse here, I'll point out that a full rebuttal was published to Dr. Probyn-Rapsey's article called ``Why so FURious?'' (ref); those wishing to obtain a copy can contact our team (furry.research@uwaterloo.ca).

\begin{quote}
  It feels to me that Gerbasi has chosen to introduce `Species Identity Disorder' because she was hoping to be the first to identify a new psychological phenomenon.
\end{quote}

Actually, I think the answer is simpler and less self-aggrandizing than that. When publishing within a field, it is necessary to tie your research to what exists within the field. If you fail to do so, you run the risk of nobody in the field caring about what you've said, even if it's of potential relevance to them, because they don't see the connection.

For example: Dr. Gerbasi initially conducted the Anthrocon study in response to a clinical psychologist who approached her with a furry client who was experiencing distress over being furry (if I recall correctly, the person did not want to be furry anymore, but could not escape ``furry'' feelings). If Dr. Gerbasi had gone ahead and simply published an article dispelling stereotypes about furries, no one in the field would have cared: they would have said ``what the hell's a furry and why should I care?'' However, by mentioning GID, anyone with an interest in self-body discrepancies, or body image issues, or clinical psychologists more generally, now has a reason to read the article, because its potential relevance to them is made apparent.

It's not necessarily for the sake of being ``the first'' (those who've met her in person know that she's not attention-seeking or self-aggrandizing), but instead was likely a device to ensure that the people who might have found it the most relevant would be drawn to it, and to more firmly establish it within an existing body of psychological research (rather than some orphaned article on a topic no one really knows).

\begin{quote}
  Her article was the first, and to date only, publication of the International Anthropomorphic Research Project, which Gerbasi heads.
\end{quote}

Actually, to date the IARP has three confirmed publications in peer-reviewed academic journals (four if you count the rebuttal to Dr. Probyn-Rapsey), one currently in a ``revise-and-resubmit'' status, two currently under review, and two currently in the process of being written up. This is in addition to presentations and posters given at eight different academic conferences. The topics of the other published / reviewed articles focus on the use of furry community as a coping resource for members of a stigmatized recreational group, a qualitative study of furries presented to the clinical psychological community urging them to avoid trying to ``cure the furry'' in furry clients, and articles investigating fan group involvement and global activism and socio-structural characteristics within stigmatized minority groups. So we've been keeping quite busy!

\begin{quote}
  The IARP is a grand title for three researchers operating from a small community college. And calling it `International' is bit bullish seeing as it's based on the fact that they have scientists from the United States and Canada
\end{quote}

The choice of the term ``International'' was meant to reflect the more than 70 countries from which respondents have come (including all six continents -- ignoring poor Antarctica), not to reflect the ``international'' nature of the researchers themselves, who all reside in North America. Additionally, we are currently in the process of establishing a translated version of our surveys for use in Japan for the purpose of a cross-cultural comparison of ``Furry'' culture and ``Kemono'' culture.

I'll also mention that, to date, the IARP consists of four ``regular'' members (myself, at the University of Waterloo, Dr. Stephen Reysen at Texas A&M University – Commerce, Dr. Sharon Roberts at Renison University College and Dr. Gerbasi at Niagara County Community College), and has worked with / is continuing to work with more than a half-dozen other collaborators from fields as diverse as social psychology, anthropology, sociology, clinical psychology and English.

\begin{quote}
  Plante joined their group in 2011 and is presumably on the way to earning the first ever PhD in furry studies.
\end{quote}

*laughs* Actually, my PhD is in social psychology, as I'm an experimental psychologist (who, when not doing research on furries, also studies video game violence and fantasy engagement). I don't suspect a degree in ``furry studies'' would get me very far! It's also important to recognize that while I may be a furry, and while I'm passionate about researching furries, I am a social psychologist first. This is important to understanding the questions that drive my interest in the fandom and my own particular bias.

\begin{quote}
  Most recently they have kicked off a longitudinal study, where they will be following furries over a significant period of time. I expect their study will dig up some interesting data, showing how we mature as members of the furry community.
\end{quote}

Indeed! More specifically, we are hoping to track what happens to furries as they get into the fandom, spend time in the fandom, and (for some) choose to leave the fandom. Changes in attitudes, beliefs, and identity are just some of the many topics we're hoping to watch unfold over time!

\begin{quote}
  The IARP dataset from 2007 is no longer considered to be particularly large or useful. Of all the available datasets, today's researchers are most likely to use Klisoura's Furrypoll
\end{quote}

Indeed, the original 2007 study is quite small by comparison, though we conduct studies two or three times a year which regularly have more than a thousand furry respondents! I'll point out here that Furrypoll is a fantastic resource, and represents a wonderful complement our own research, given its larger sample size (owing likely to its running year-round as opposed to our surveys which are only open for a few weeks at a time) and its ability to ask questions of under-18 furries (which, unfortunately, we are prohibited from doing due to the nature of ethics boards). Comparing our research with Furrypoll is like comparing the work of biologists and organic chemists: both are valid approaches that solve different questions and, when combined, provide an even richer understanding of the areas where they overlap!

*phew* That was a long response! Thanks very much for the thoughtful criticism and for taking the time to comment on and link to our research! I'll also encourage you to check out our latest results (Furry Fiesta 2013), which include some brand new issues in the furry fandom (e.g. pornography, employment and living arrangements, relationships, fantasy engagement, etc…). If you're interested in participating in one of our future surveys (or in signing up for our longitudinal study), feel free to check out our website: https://sites.google.com/site/anthropomorphicresearch/
