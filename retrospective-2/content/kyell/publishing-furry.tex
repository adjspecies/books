\articlehead{Publishing Furry}{Kyell}{2013}

In his recent article Of Rabbits and Rayguns, Phil Geusz described his fight to get the mainstream publishing industry to accept furry fiction. Phil is an accomplished writer, and he and I share a number of views and goals. He has chosen a different path than I have to attain those goals, but I think both paths are valid. In response, or perhaps complement, to his article, I'd like to offer my own thoughts.

I have for the past decade been becoming more and more familiar with the furry publishing industry. I have good friends at Sofawolf and FurPlanet and we talk business on a number of occasions. As a fairly high-profile author, I am always trying to figure out ways I can make our joint business more successful. At the same time, over the past two or three years, I have been talking with and listening to more people in the larger SF publishing world; mostly authors, but also some editors.

It's probably instructive to define a couple things before we get into the state of the industry and its attitude toward the fandom. First, I want to draw a distinction between furry fiction and fiction with furries. Furry fiction, by my definition, is fiction in which the furries are the prominent, if not exclusive, characters, and in which their existence is accepted as a given. Fiction with furries is fiction in which furries are an explained part of the world, possibly protagonists, but not necessarily.

To give examples: my \textit{Out of Position} books are furry fiction. Kevin Frane's books are furry fiction. Watts Martin's \textit{Ranea} stories are furry fiction (but closer to the line). The \textit{San Iadras} stories of foozzzball are fiction with furries (but again, close to the line). Most of Phil's stories that I'm aware of would be in that same category, as would Cordwainer Smith's \textit{Norstrilia} and other underpeople stories, C.J. Cherryh's \textit{Chanur} books, Larry Niven's \textit{Kzin} books, Alan Dean Foster's \textit{Spellsinger} books, H. Beam Piper's \textit{Fuzzy} books and John Scalzi's recent reboot\ldots Fred Patten could probably add another hundred names to that list. Jonathan Lethem's \textit{Gun, With Occasional Music} is also fiction with furries, but much further from the dividing line: the kangaroo is a secondary character, and while his existence is never really explained, he's also not that important to the story.

The reason to draw this distinction is that we in the furry fandom are used to furries. We are so used to them that now, stories that include the Origin Of Furries as a major part are very rare. We know all the origins: space aliens, created by humans (for war or sex, usually), or humans changed by magic or a virus (sometimes a magical virus). That is no longer as important to us as what the author does with the characters.

Non-furries, by contrast, often need the origin story, and so there is a gap between what is publishable in the furry fandom and what is publishable in the larger F/SF world (there is also a gap because of the general quality of furry fiction and the barriers to publication, and that gap is actually larger than the gap attributable to world-building, but that's another post). And that leads me to the second thing I want to define.

The publishing industry is a business. It is a tricky business, because they are taking creative properties which people have a personal investment in and attempting to gauge how many other people will relate to those properties enough to spend money on them. Publishers tend to the conservative, because if they take too many gambles, and nobody buys the books they publish, they won't be publishing for very long.

In general, publishers lump book submissions into four categories:

\begin{enumerate}
  \item This is awesome and everyone will buy it;
  \item This is pretty good and we know we can sell it;
  \item This is good but we don't think we can sell it;
  \item No.
\end{enumerate}

(It will not surprise you to learn that the vast, vast majority of submissions fall into category 4.)

Right now, there is no established market for furry fiction, not in the F/SF mainstream, not in the way that there is for space opera or Tolkienesque fantasies or wizard school stories. I'll come back to this in a moment, but what that means is that a furry book, or even a ``fiction with furries'' that is close to the line, pretty much has to fall into that first category to get serious consideration from a mainstream publisher. And I don't think that book has been written yet.

(By contrast, you will see from the above examples that there is plenty of ``fiction with furries'' in the mainstream market, and has been for decades. There's no publisher I know of that would reject a book they otherwise would have bought just because some characters in it are furries.)

But the good news is that there's another way to convince publishers that furry fiction is a winning proposition, and that's by showing them. This is what Phil is doing with his books; this is what short story sales to F/SF venues (Renee Carter Hall and Mary Lowd have both had that success this year) do; this is what strong sales by furry publishers do.

Here's where some of the burden falls on the fandom. The furry fandom is still small by comparison to worldwide SF fandom. In the fandoms' respective awards, the Ursa Majors, which are free to all, do not get as many votes as SF fandom's Hugos, which require a minimum of \$50 to vote. Furry fandom is, I think, more active, more vibrant, and more creative—but furries just do not read as much as science fiction fans, whose fandom is centered around books.

Fantasy and science fiction books regularly make the New York Times bestseller lists; the industry supports thousands of midlist authors who sell thousands of books. I have friends in the fandom who buy hundreds—plural—of books a year. In many ways, furry fandom is stronger than SF fandom; books, for the moment, are not one of those ways.

And this brings me back to the statement about there being no established market for furry fiction. There is, clearly, a market. Sofawolf and FurPlanet and Rabbit Valley manage to not only survive, but prosper, selling furry books, but they operate on a smaller scale than Tor, Daw, Baen, etc. The good news is that that market is growing. Phil has found success outside the fandom with his \textit{David Birkenhead} series. I've found success outside the fandom with my \textit{Out of Position books}. With every sale outside the fandom, we are showing people that they can enjoy stories with furry characters, and opening opportunities for more furry books.

It's a slow process, but that's the way these things happen. Are we ever going to convince everyone in the world that they should like furries? No. I occasionally see the same comments Phil talked about, from people who just have to tell the world how much they don't get furries, and you know what? That's fine. There are people who hate vampires, people who hate boy wizards, people who hate science fiction. In fact, I think the appearance of these protestors is actually a good sign. The fact that they feel the need to loudly state their objection means that they feel like they're in the minority. They look around and see other people appreciating these stories, and they don't get it, and because they don't feel like part of the ``in'' crowd, they have to justify their stance.

But I think it's a mistake to assume that mainstream F/SF publishers feel that way, simply because they say ``we don't publish talking animal stories.'' I had a conversation with an editor from a mainstream F/SF publisher in which she said, ``Why does there have to be a distinction between furry fiction and `mainstream' fiction? Why can't it just be about good stories?'' Amen. Remember, publishers are a business. If they find a story with furries that they just can't ignore, they will publish it.

Furry fandom is growing, and the market for furry books is growing. I've mentioned this before, but I think this past year was the first time all five Ursa Major Novel candidates were written by furries. I remember a time when you could walk through a furry con's dealer's room without seeing a single novel; now you can choose from probably about fifty of them.

Are they selling at a mainstream F/SF level? Not yet. But readership within the fandom is growing, and the fandom itself is growing. I'm pretty happy selling my books to the people who are passionate about them. The people who don't want to read them don't have to. I'm just glad that there is a big, growing community of people who do.
