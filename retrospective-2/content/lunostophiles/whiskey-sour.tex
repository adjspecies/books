\articlehead{Whiskey Sour}{Lunostophiles}{2013}

Emotion lives out its life in poetry. It might summer in prose, it might vacation in speeches, and it may even spend a nice weekend wrapped around a pithy quip. But, in the end, emotion's country of origin is poetry. Even before we wrote stories on paper, far before we recorded everything we created in a fashion archivists scratch their heads at, there was poetry and verse.

The fandom has been slow to adopt poetry, and it's not without its reasons; too often these days culture equates verse with self-absorbed and self-diagnosed loners who attempt to pour their sadness onto the page in recursive stanzas. Are they wrong in choosing this course of release? Of course not, but these ‘angry emo journal poets' have eclipsed the multitudinous and varied styles of poetry there are out there.

(There is, to be fair, a lot of blame to be laid on the poetry curriculum in schools, but that is a conversation for another day.)

With growing sub-communities devoted to writing verse, I'm confident there is a place for poetry in the fandom in the same way there is a place for prose, art, and fursuiting. There is no end to what poetry can accomplish, both within the constraints of meter and rhyme and without. If prose is the way by which we show others how we view the world, then poetry is the way by which we glean meaning from the world we view. A sunset is just a sunset until you can describe it as something else. Then it is much more.

\subsection*{Whiskey Sour}

\begin{verse}
  We cup our claws,\\
  Our talons,\\
  Our nubby, rum-soaked fingers round flimsy cups\\
  Thrust high in praise of the bacchanal;\\
  Of deities borne through chants whispered into bottle caps,\\
  And gods reincarnated with too-loud laughter.\par

  And we, members of a growing cult\\
  That malingers like a skulking formaldehyde dream;\\
  The clan of eternal headaches,\\
  Of moist and sloppy lip-locks in bathrooms,\\
  A brotherhood we did not know we had joined --\\
  All hidden behind locked hotel room doors\\
  Dangling signs to ward away housekeeping just one more day.\par

  The tingling fingers of siren cocktails draw shadows on our eyes,\\
  Their clarion songs promising personality,\\
  Conviviality,\\
  New and absent friends cast in the fires of a molotov.\par

  The party floors reek of high-proof happiness by Thursday's end;\\
  A massive, sharp-toothed plague that grips us\\
  Like beef bourguignon with the red overflowing,\\
  And in its powerful jaws\\
  Forces from us a vomit of glee.\\

  \secdiv

  In my naivete, my swollen days of Massachusetts autumn,\\
  When life was a marbled haze upon my eyes,\\
  New to the north, new to adulthood in its bleak daylight;\\
  It is here I was first thrust headlong into the convention scene.\par

  The smiles of the rogues,\\
  The shade-beings,\\
  Frothing like the head of a fresh-poured Guinness,\\
  With arms outstretched as great bows with no arrows.\par

  ``You're here!'' they cried, they shouted!\par

  ``You've made it!''\par

  ``No more are you doomed to a life\\
  Where what you know of us are pixel silhouettes,\\
  Spectres and creations of fervent, bored imaginations\\
  Illuminated to life upon LCD screens.\\
  No more will you play the most dangerous game\\
  With mouse cursor and hyperlink,\\
  A man on wild safari for a beast no one has caught!''\par

  The lobby was Kublai Khan's pleasure dome,\\
  Husky and dense with delights:\\
  Shrieks of absences making hearts grow fonder\\
  And the soft hum of happy chatter.\\
  This was the soundtrack of a grin.\par

  And this Morphean utopia,\\
  All swathed in furs and memetic shirts,\\
  Laid itself before me prostrate like a lover waiting.\\
  And somehow, despite having never charted these waters,\\
  I spread my fingers wide, the rays of a distant star\\
  Upon the china white body of this vast world made flesh,\\
  Feeling blind corners and sharp elevation changes.\par

  And in my mind, this monolithic and precise relief\\
  Fit jigsaw-snug into the jagged-edged,\\
  Razor-toothed pockets of the conspace --\\
  Just like I knew it would.\\

  \secdiv

  The size of the party means you're having more fun!\\
  Kiss the elbow of the man next to you\\
  (Though you aimed for his lips\\
  And your trajectory erred),\\
  Caress the obliques of a stranger --\\
  Any stranger! --\\
  They know you in spirit.\par

  We pack ourselves tighter into a four-person cubicle,\\
  Sardines with no oil or water,\\
  Just marinating for the main course.\par

  We keep laughing, we writhe our bodies;\\
  We roll our heads, unattached, through the marathon hallways,\\
  Down the stairwells and across the pool chairs,\\
  Colossal sound extricating itself from our maws thrown wide with venom;\\
  Venom and veracity.\par

  Keep laughing, you fools! This is of import! --\\
  Don't let's talk, don't let's converse.\\
  Imbibe, my comrades.\par

  Imbibe!\\

  \secdiv

  Acquaintences met, acquaintences made,\\
  And now a believer in the throes of transubstantiation\\
  I rose from the fairgrounds,\\
  Making careful, tiptoe steps into the elevator\\
  As if wary of nightengale floors.\par

  Rising, rising! like the wind through a flue,\\
  Then left in the dim hallway of an upper floor;\\
  A babe in the clasp of some darkened bosom.\par

  A friendly face?\\
  There, past the ionic columns of pizza boxes,\\
  The tenuous styrofoam skycrapers\\
  And sunken pagodas erected in the conquest of General Tso;\\
  There, through the chalky dark mist, I wandered,\\
  Unaware that this was the land of the forgotten;\\
  This was the desert Moses lost himself in for forty years,\\
  Or a world Euclid would have wept at the sight of.\par

  Hand-scrawled signs on the closed doors,\\
  Effegies of animal-men in cartoon hysterics,\\
  Voiced by a backmask reveille --\\
  Were they speaking?\\
  No, they were barking; mad creatures\\
  All scraping claws on cage bars,\\
  Aching for an exit of this perverted zoo.\par

  A smile across the hall --\\
  My brethren!\\
  They ushered me from the dark and dreary path\\
  And into their light-filled embraces,\\
  All hearth and home.\par

  On the desk, a lanyard graveyard,\\
  Piles of forgeries laid waste in private\\
  To mingle in a flat-ironed spiderweb;\\
  And looming over us all was the altar,\\
  The godless instrument for impassioned debauchery;\\
  A boozy glass harmonica.\par

  I was handed a cup.\par

  In downcast gaze, I saw myself in the milky mirror,\\
  An endless pit just below the surface film.\\
  Its jaws gaped, a chasm, an abyss,\\
  A lion awaiting the head of its master\\
  (And I with no whip or chair).\\
  The drink plumed personality from its depths,\\
  Swarthy and succulent,\\
  Sugar and spice\ldots \\
  \ldots And the hooch was quite nice.\par

  As if I had exchanged lives with a desperate man\\
  Lost in the Sahara, carrying a dry canteen,\\
  Upon seeing the liquid I erupted with need\\
  And the drink disappeared in a fit of magic.\\
  The cup hung as a red flag upon my body,\\
  Too obvious to notice,\\
  Waving defeat in the cold October air.\par

  My thoughts grew hairline fractures, fit to burst at the seams;\\
  The cup was refilled;\\
  And I'd've rather rinsed than repeated\\
  But is it not unkind to turn down one's host?\\
  The steps to a new and baffling dance snuck on through,\\
  A sway and a hop I had hidden,\\
  Shoved under blankets;\\
  Sandwiched between floorboards.\par

  I guzzled, I glutted,\\
  I quaffed and I chugged and I drank.\\

  \secdiv

  Deaddog, deaddog,\\
  Come out to play.\\
  The boy's in the meadow,\\
  The girl's in the hay.\par

  The boy's at the toilet,\\
  The girl's at the sink.\\
  Deaddog, deaddog!\\
  Just one more drink?\\

  \secdiv

  A name, a curse,\\
  Scratched, tattooed in dismantled English,\\
  Tight gypsy glyphs in thick-line Sharpie on cheap red plastic\\
  As if this chalice of consumption,\\
  This cup of infinite holding was mine forever.\par

  But it's never quite ours forever, though;\\
  Never just quite.\\
  When all the rum, all the gin, all the mixers run dry\\
  And down to the floor we descend in a daze;\\
  When corpses of bottles are strewn on the desktops,\\
  Under beds,\\
  Across suitcases unpacked;\\
  When we have constructed mass graves and catacombs to coquetting\\
  which overflow the trash bins;\\
  Tremendous and terrific mountains to excess\\
  Unfit for us to scale --\\
  More appropriate, as knackered as it is,\\
  To set it aflame like a phantasmagoric funeral pyre,\\
  And let acrid smoke curl through the room and asphyxiates us.\par

  When this death waltz has begun,\\
  We stare from the valley of drunken stupor,\\
  Cross-eyed and infantile,\\
  And we gurgle out our sorrows, intoning our distates,\\
  And the once-bright laughter falls pallid and flat;\\
  Fetal fallen angels neck-deep in Hell's detritus.\par

  It is possible to reverse transubstantiation --\\
  In those moments, it is possible to eat your own halo.\par

  The spark of newness rubs away quick,\\
  Like the silver ink on a fresh credit card.\\
  Deep in the cavities of the room parties,\\
  Shadowed under the awnings and eaves of hedonism\\
  (May Dionysus his name be praised into the porcelain shrines!),\\
  And the towering she-wolves we suckle from --\\
  Romulus and Remus ad infinitum --\\
  Inside these wounds we lose the virgin edges,\\
  We claw our way into the light of day\\
  And hiss at the sun.\par

  I do not want to become a parody of intelligence.\\
  I do not want this to be our brave new world\\
  Filled with the vapor trails left by regret,\\
  Bitterness smothered in cold flame.\\
  I will not be baptized into the Church of the Dead Soldier:\\
  Not by mother vodka.\\
  Not by father whiskey.\par

  Yet still, I raise a toast --\\
  In a smaller, finer glass --\\
  To friendships forged in the fandom's smithy;\\
  A fandom sought out by outliers and outcasts --\\
  Those without names and those with too many.\\
  I will laugh a real laugh,\\
  A room-filling sound that is never too loud,\\
  Fringed with the fragile lace of mirth.\par

  And high above us, the dirty angels of the rooms\\
  Pray to their patron saints to let them see the afternoon.\par

  For unlike we folk awake and alive,\\
  They have not learned how to hide their halo\\
  Just behind their backs --\\
  Just out of reach from the cold and clammy hands\\
  That still crush the plastic party cups into cadavers.\par

  No, they have no place for their goodness,\\
  And hide their glow in the bottoms of cocktails;\\
  Just around the far side of the martini olives\\
  That gaze upon them and despair.\par

  And in that moment,\\
  With the very eyes of their consumption cast outward?\\
  Just smile back, take a sip,\\
  And make it the last.\par

  At least for the night.
\end{verse}
