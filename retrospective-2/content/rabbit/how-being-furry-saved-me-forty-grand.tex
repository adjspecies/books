\articlehead{How Being a Furry Saved Me Forty Grand}{Rabbit}{2012}

Tonight I test-drove a \$40,000 pickup truck. Don't get me wrong—I never had the slightest intention of buying the thing. As I made sure the salesman knew before I ever climbed in and turned the key, I was actually maybe, possibly interested in a baseline truck that costs about half that. My current plain-jane 4×4 is seventeen years old and has nearly 100,000 miles on it, you see, and the auto manufacturer I work for is currently offering large rebates to the general public and even larger ones to their employees to move the things more quickly, which sparked my interest. But the dealership had nothing but top-end super-fancy (read that ``high margin, high profit'') stuff on their lot, so if I wanted to take a test drive it was a \$40,000 truck or nothing.

The trip around the test loop was routine, including the salesman establishing who I worked for and thereby learning how much he might be able to bleed me for. I told him, of course—it's not polite not to. So when we returned to the dealership I was treated to the predictable chorus-greeting from the rest of the otherwise-unoccupied sales staff. ``Oh, isn't that a gorgeous truck?'' ``I've never seen such a wonderful shade of blue!'' ``I hear (insert local celebrity's name here) drives one just like that!''

At this point of course I sighed, explained that I was still doing research, and left a terribly disappointed group of middle-aged men behind me. But in much the same way that I'm certain the salesmen, being salesmen, are still reassuring each other that I'll be back even as I type this, as a writer I find myself analyzing both their and my behavior over and over again.

(Please, give me a little more time. This article will become relevant to furry before it's over, I promise!)

The salesmen were doing their best to apply social pressure to me, to make it clear that buying a \$40,000 truck is a behavior smiled upon by society and sure to make me more popular and celebrity-like. They flattered its new-for-2013 color—I'm sure they could care less, if forced to be honest—and extolled the virtues of a product they knew for fact was a far more expensive vehicle than I really wanted or needed. Some of them—the sales manager and my own salesman—stood to profit financially if I succumbed to the pressure. But most of those doing the cajoling would gain nothing more than a smile of approval from their boss and atta-boys from their coworkers if I'd bought the silly thing.

Sadly, I'm no longer either much saddened or shocked when people treat each other as mere cash cows while conducting business. Greed explains much, and the social status associated with ``success'' most of the rest. What I can't get over is that simpleminded, transparent tactics like this continue to work and work and work, not just year after year but century after century. I mean\ldots  It wasn't just obvious to me what was going on, it was sickeningly obvious. Yet the staff wouldn't continue to behave in such a manner if it didn't sell \$40,000 trucks, and as further evidence I'm forced to acknowledge that an awful lot of my co-workers do in fact park very similar vehicles right alongside my far-cheaper one every day of the week even though they can't afford them any more than I can. In fact, they often buy them from that specific dealer and tell me afterwards what a nice bunch of guys they are!

People are so stupid, I muttered sadly to myself as I drove away from the dealership in my seventeen-year-old, still perfectly serviceable truck that not one but many salesmen have done their level best to part me from. So primitive and easily led. Then, strictly as a mental exercise, I listed those friends of mine who I reckoned might see through the sales pitch as easily as I did.

Almost every one of them was a furry, I realized with a bit of a shock.

And that's the realization that led me to pen this column. I've long contended that the furry fandom (along with the SF fandom and some others) really is different from the bulk of society in some basic, fundamental ways. Part of it is clearly intelligence—statistically speaking I believe we're well offset towards the high end of the curve—and one aspect of seeing through traps like the one at the pickup dealer is indeed intelligence. Another, I would contend, is imagination. In order to see the hidden poison behind all the happy-faced affirmation, one must first be able to imagine the possibility that such nastiness actually exists.

I believe, however, that another factor counts for more than either of these: our sense of ``outsidership''. The application of social pressure is an ineffective lever at best when applied to a dedicated non-conformist, and is often actively counter-productive. But ``normal'' people are different. From the outside looking in, at least, their need to ``fit in'' appears to be one of the most powerful forces if not the most powerful force in the lives of most non-fen. Not only does the vast bulk of the population share precisely the same short list of ``acceptable'' interests and hobbies with each other, they often grow acutely uncomfortable at so much as the idea of, say, owning an unusual species of pet or driving a car that looks substantially different than everyone else's in an unapproved way. ``Why would anyone want one of those?'' they ask. ``No one else has one.'' And at that point the discussion is pretty much over—no matter how many advantages you cite, they shake their heads and grow increasingly uncomfortable at being confronted with the horror of Being Different or thinking about a New Idea. My own father, for example, was utterly bewildered for years by the fact that I write books about half-human animals. Or at least he was until the recent network series based loosely on Grimm's Fairly Tales started, which features a lot of anthro characters. ``Oh!'' he said after it began to air. ``I guess it's okay after all, if it's on TV.'' That's an exact, word-for-word quote. And while I love my Dad very much, it explains a lot about many things. Including how so many people can so easily end up behind the wheels of \$40,000 trucks they don't even remotely need. Because those are on TV too, you see. So it must be okay.

Sure, there are furs among us who are ``easy marks'' for skilled salespeople. But they don't make up nearly as a high a percentage of our population as in the general public, I don't think. Why? Because not only are we more intelligent and creative than the general population, we also dance to an entirely different drummer. We embrace the new and different while regarding the conventional with at best distrust. Many if not most of us spend much of our free time at conventions (where we happily wear things most people would rather die than be caught in), looking at furry web pages (where we smile at images that most ordinary folks would find confusing or possibly even repulsive) and chatting online with people that Joe Average Truck Salesman would never, ever willingly admit to spending time with.

And that, I think, is the answer to the question of why I instantly thought of furs as likely being able to see through the little sales-charade I was subjected to tonight. At core we furs don't care what our non-furry peers think, or at least not nearly so much as everyone else does. It's more a pragmatic sort of caring, in other words, as apposed to the vital life-or-death emotional lifeline that social approval seems to constitute for everyone else. This makes us a lot tougher to manipulate; throwing the conventional levers gets you nowhere or worse. So when the sales staff performed their little number I laughed inside instead of reflexively getting out my checkbook in the hope of making myself more like everyone else in my community. Indeed, their actions made me less likely to ever buy anything from them, ever.

I'm not certain, mind you. So don't hold me to this. But I suspect that this is the first time ever that being a fur has actually saved me money\ldots
