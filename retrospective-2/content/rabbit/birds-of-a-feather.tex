\articlehead{Birds of a Feather}{Rabbit}{2013}

It's so trite that it's an eyeroller. But that doesn't mean it's not true. Birds of a feather do flock together, even when the ``birds'' in question lack bills and plumage. I'd guess this is especially evident in my native United States because our population is among the most culturally diverse on the planet. Take my original hometown of St. Louis, for example. During my childhood Italian-Americans lived in a neighborhood called ``The Hill'', while German-Americans (commonly referred to as ``scrubby Dutch'', and half of my own ethnic heritage) mostly tended to live on the South Side and in the inner southern suburbs. The city's small but prosperous Lebanese population also lived towards the south end of town, though not so far in that direction as the Germans. Old-money rich folks lived mostly in a now-decrepit area not far from Forest Park, while more recently prosperous families bought homes in the suburbs of Ladue or Brentwood. Artsy types were mostly clustered in a couple little zones near either Washington University or Webster College, in Webster Groves. The gay community was very small in those days, but even so Tower Grove Park and the neighborhoods immediately surrounding it were closely associated with what little activity there was. The North Side of St. Louis, as well as the adjoining suburbs, have been home to most of St. Louis's African-American population for as long as I can remember and probably much longer.

St. Louis isn't at all unusual in how the various social and ethnic groups are geographically clustered; choose any American city and you'll see pretty much the same pattern repeat itself over and over again. In fact, it doesn't even take a city-sized population for the trend to manifest itself—not far from my current home in Middle Tennessee a group of artists are trying to start an ``art village'' in what was once a cluster of vacation homes overlooking Kentucky Lake, pretty much out in the middle of nowhere. It seems, in other words, that ``flocking'' of this sort is a basic human behavior pattern. Sometimes the clustering is involuntary and/or due to economic discrimination and racism, as in the old-style Jewish ghettos of Europe and arguably in the case of many of America's worst urban nightmare-neighborhoods of today. For the most part, however, the decision to live together in culturally-distinct groups is clearly a voluntary choice by groups of individuals who share much in common and wish to live a lifestyle together that's optimized towards their own specialized wants and needs.

We furry authors have been taking advantage of this tendency almost since there have been furry authors. The ``Blind Pig'' storyverse is just one example among many; it's adventures are centered on what amounts in social-function terms to a gay-type bar located in a distinctly furry neighborhood. (Yes, Blind Pig characters sometimes are not furry and suffer from unrelated issues that have nothing to do with anthropomorphism. But let's keep things simple for the purpose of this essay.) I've seen this pattern repeated in more furry stories than I can remember, and even in stories in which there are no normal humans I often note references to ``the canine district'', ``Vulpine Alley'', ``The Warrens'', etc. ``Neighborhooding'' is such a basic, familiar phenomenon that it's hard for human authors to imagine a world in which it does not take place.

Now, let's take it one step further. Recall the St. Louis neighborhoods I cited. Most were based on ethnicity. But\ldots

\ldots one was the gay district. And another was the ``artsy parts of town''.

After attending as many furry conventions as I have and watching the fandom interact with itself, I've often wondered. Will we, someday, become culturally distinct enough from mainstream society to form a neighborhood or our own somewhere? Real-life, I mean -— not fictionally.

I mean, think about it. Ears and tails, when we choose to wear them, mark us as visibly different from society at large in very much the same way as a Hassidic Jew is visible. Our tastes in recreation and art are also notably different -— how many non-furs so love to wear whimsical costumes in public? In a furry neighborhood, game-shops would flourish and suit-makers would operate little boutiques in the most expensive parts of town. People would hug each other openly on street-corners, while perhaps on the busiest street corners professional suiters would ``busk'' for tips and photo opportunities. Tourists would come from all over to see this, which would in turn mean souvenir shops and nice hotels for them to stay in. Policemen and street repair gangs would be free to wear -— or not -— ears and tails on the job, while art would be everywhere.

It's doable, I think. Unlikely, yes -— at least in my lifetime. But eminently doable, and probably economically viable as well. A large furcon is a lot like a temporary neighborhood, in that it provides structure and security for a large population for a few days. So, in a sense, a furry neighborhood would just sort of be like a large con that simply never ends. And isn't that a wonderful thought?

I once wrote a story based on this premise entitled ``Pelton''. In it, several thousand furs backed by a multi-millionaire buy large parts of a failing downstate Illinois town in order to create a furry ``homeland''. (It's available at http://tsat.transform.to/stories/pelton.html for anyone who wishes to read it, but I'll offer advance warning that it's one of my earliest works and therefore needs re-editing badly. Someday I'll get around to that\ldots ) Between the furs, the poor befuddled Old People who predated the anthropomorphic invasion and the eternal fandom-type drama that was always waiting to rip and rend everything that'd been so carefully built, well\ldots  It was quite a challenge for the man unfortunate enough to be Mayor. Yet in the end, like our fandom itself the grand experiment proved capable of giving birth to something\ldots  Beautiful.

Do I really expect to live to see it happen? Not really, unless we're lucky enough to find the kind of deep-pocketed backer I specify in my story. But\ldots

\ldots just on the off-chance, I vote for someplace with a warm climate and affordable housing like Florida or Texas. And\ldots  Please call me early, because I want to buy a house with a balcony overlooking the town square!
