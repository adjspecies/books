\articlehead{Blood, Toil, Tears, and Fur}{Rabbit}{2012}

\textit{This is a lightly edited reprint of a column from Anthro Magazine \#20}

I've always admired Winston Churchill, perhaps more than anyone else who ever lived. Somehow he managed to cram not one but a whole succession of lives into the span of one. He rode in the last cavalry charge of the British Army; wrote more books than most full-time authors (winning the Nobel Prize in literature along the way); became arguably the most successful columnist and reporter of his day; was a noted watercolorist; coined terms like `seaplane' and `iron curtain'; arguably invented the tank; not only prepared the Royal Navy for World War I but also led it during the early and most crucial parts of that conflict; and sponsored key social legislation that few associate him with today. He was present on Wall Street just after the Crash of 1929, in Cuba during the insurrection against Spain, and personally fought in desperate, bloody actions in India. He remains the only person ever voted an honorary citizen of the United States, by special Act of Congress. Oh, and by the way, he also led Britain during the proudest and toughest period of her history, when she stood alone for freedom against Adolph Hitler and all of occupied Europe. Mustn't forget that part!

He was also without question a furry, long before there was a name for such a thing.

Shocked? So was I, when I first came upon the truth while reading The Last Lion, a biography of the man by William Manchester. Unlike all the other biographies I'd read, this one was up close and personal-more about the man himself than his accomplishments. In it I learned of the troubled, attention-starved youth with a wild and vivid imagination, who couldn't ever quite fit in and all but failed out of school because he couldn't deal with the regimentation of rote learning. I cried with the adolescent who refused to abandon his nurse despite the fact that he was mocked for it by his peers -- his parents had cast her off to live on nothing, and young Winston helped her with money from his own allowance and kept in close touch until the day she died. Later, I grew to know the brilliant young man whose keen intellect eventually became apparent to everyone, but whose poor social skills kept him an outcast. And, I have to admit, everything seemed to be fitting a familiar sort of pattern.

But I couldn't quite put a name on it until I ran into his fursuit.

Yes, it's true: Winston Churchill owned a fursuit. More than that, he owned a whole closet full of costumes, though apparently this was his favorite. He wore it quite frequently, it seems, playing and roughhousing with his grandchildren. As difficult as it might be to picture, according to Mr. Manchester Winston Churchill loved to dress up as a gorilla.

I blinked when I read that part, as little bells and whistles began to ring in my mind. Churchill also kept odd hours, sleeping twice a day instead of once, and did his best work late at night. In a nation noted for its eccentrics, he was an oddball. Winston loved animals deeply -- his home was supposed to be a working farm, but he could never bring himself to slaughter any of the livestock and even worried for days once over a sick goldfish. More and more alarms went off\ldots

\ldots until finally I hit the hard, definitive paydirt, the letters between he and his wife.

Here's a quote from Manchester\ldots

\begin{quote}
  Like other lovers, they invented pet names for each other. Clementine was ``Cat'', or ``Kat'', Winston was ``Pug'', then ``Amber Pug'', then ``Pig''. Drawings of these animals decorated the margins of their letters to each other, and at dinner parties Winston would reach across the table, squeeze her hand, and murmur ``Dear Cat''.
\end{quote}

Or, at a later date\ldots

\begin{quote}
  ``We are going to bathe in the lake this evening,'' he told her in a typical note. ``No cats allowed! Your Pug in clover, W.'' And she would assure him that while he was gone ``your lazy Kat sits purring and lapping cream and stroking her kittens.''
\end{quote}

These were not one-offs, taken out of context. Due to Churchill's odd schedule and frequent travels, he and his beloved Kat didn't see much of each other, and even while living in the same house they wrote each other frequent letters. Practically all of them are full of love -- and they're equally full of what we today would recognize in a heartbeat as typical anthropomorphic on-line role-play.

Here's another example, among many. In closing a long letter in which Churchill's political enemies are clawed to pieces, Clementine wrote her husband:

``Good-Bye, my Darling. I love you very much. Your Radical Bristling-'' here she drew an indignant cat.

It goes on and on and on in this vein. A modern-day fur, looking at this body of correspondence, cannot help but feel right at home. Indeed, he might even envy the easy and natural way that these two very-much-in-love individuals unselfconsciously communicated using the anthropomorphic symbols and language that clearly meant so much to them. Matters continued in this vein to the very end, as did their love. If any part of Churchill's life can be described as filled with joy, this was probably it.

A lot of people seem to enjoy bashing furs. These same sorts of people seemed to enjoy bashing Churchill as well until he grew into such a historical giant that no one dared any longer. He started out life as an awkward, troubled, sickly and accident-prone youth that no one understood and who seemingly couldn't get ahead. But he grew tall and strong, perhaps taller and stronger than any other man of his time. There's not the slightest doubt in my mind that, were he alive today, we'd find him attending furcons and hanging around in furry chatrooms.

I'd submit that Winston Churchill's furriness, along with the intelligence, creativity and sensitivity that so often accompany it, was an essential component of his colossal strength. Certainly, it was a major part of who he was, and how he saw the world.

Which apparently wasn't, if you're reading this, so very different from the way that you and I see it.
