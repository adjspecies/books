\articlehead{Food Stuff}{Rabbit}{2013}

I probably shouldn't even attempt to write this article. I lack any real background knowledge on the subject, and have no academic credentials of any kind whatsoever. While I enjoy writing and contributing the occasional essay, I really ought to be devoting the time to my fiction-writing career instead just now; it's at a crucial point, and not in a good way. Nor will many people will read this compared to my works of fiction. I'm beginning late at night after a long, hard shift at work knowing full well that I've got to get up extra-early tomorrow, and this after going short on sleep last night for other reasons entirely. In short I'm a damned fool to be sitting here typing this. And yet here I am, pounding out the words into my iPad at the local 24-hour eatery.

Why? Because I feel a strange compulsion to do so. A compulsion, I feel, that has much to do with not only who I am as a person, but also why I'm a fur. And it was inspired, of all things, by a coincidence in timing. Just days after finishing the final draft of a novel that in large part deals with the taboo of human cannibalism, I watched with wonder as the British media erupted in outrage over the discovery of horsemeat in the human food chain.

Now, after writing hundreds of pages dealing with the ethics, morality and psychological aftereffects of consuming human meat under survival conditions, well\ldots I have to admit that at first I found the controversy more amusing than anything else. A late Canadian friend of mine was a big fan of horsemeat as a delicious, healthy and cheap alternative to beef. (And yes, as a matter of fact he was an equine-type furry. Whyever do you ask?) This same friend was always mildly amused that horse was banned as a foodstuff in so many countries. For my own part, I spent a couple years working for a fast food chain that was up until that time perhaps best known for having one of their Australian suppliers caught substituting kangaroo meat for beef patties. The tainted shipment never made it to the public, but a good decade or so after the story broke people were still joking about ``bouncyburgers''.

Both then and now, I have a great deal of trouble understanding why people get so worked up about this sort of thing. Dead flesh is dead flesh, and even if we limit ourselves to the Western tradition of cuisine we consume a staggering variety of the stuff. Within the past year I personally have eaten dead pig, dead cow, dead chicken, dead sheep, dead shrimp, dead fish of more varieties than I can name, dead clam, dead scallops, dead crab, dead lobster, dead frogs and even dead alligator. (Sorry, I've never tried dead snails. Though I'd be willing!) This list covers all five orders of vertebrates plus several invertebrates. And you know what? It all tasted pretty much the same. The invertebrates less so, granted. But if cooked in unfamiliar recipes, I doubt that I could name any of these meats solely by taste. They're all pretty much pure protein; if you want variety in taste, the vegetable kingdom is without a doubt the place to look.

Obviously, then, our preferences in meat-animals are driven largely by factors other than flavor. Most Europeans salivate over hot roast pork, a Bedouin licks his lips at a platter heaped with roasted camel, some Indian gourmets reportedly roll their eyes over monkey-brains, and in certain parts of Asia nothing makes a diner happier than well marbled dog steaks. While I've never tried most of these\ldots

\ldots how much do you want to bet they all taste pretty much the same?

Ethically, I have to admit, I consider them all pretty much the same as well. While I'm very much a meat eater, it's not because I think that routinely killing self-aware beings is a good idea. The fact is that I am a creature of very little to no dietary self-control. I'm grossly overweight as well as being a carnivore; being so fat is merely another manifestation of my own lack of self-discipline. As a six-year-old I once wept at the death of about a half-dozen trees that'd stood for years on my grandparent's property; they'd grown too large to be safe, and so had to be cut down before they fell on the house and crushed it. Part of me has never quit weeping at the uncounted thousands of creatures I've consumed or otherwise killed since. And yet\ldots

I also recognize that I'm a born carnivore as well. I've taken a couple-three classes in anthropology and read numerous books on the subject. This is more than enough to make me fully aware of the massive behavioral, cultural, and even physical effects that the act of hunting has had on the development of mankind. A major change of diet is a radical thing in terms of evolutionary pressures, a massive driver of change. When our ancestors were insectivores, we were tree shrews. When we were (mostly) vegetarians, we were apes. When we became hunters, we became men. That's an oversimplification, yes. But it's not all that much of one, which underlines the importance of diet to who and what we are as a species. I'm the son and great-great-great to the nth power grandson of the finest, deadliest hunters this planet has ever produced. My genes, my very identity and the manner in which I view the world necessarily reflects this truth. Should I be ashamed to eat meat? More ashamed, say, than a Bengal tiger who's not half as capable and versatile a hunter as I am?

There are a thousand million arguments in regard to the ethics of meat-eating, and I don't intend to even begin to deal with them all here and now. Suffice it to say that I, as a dedicated technologist, believe that the single greatest techno-ethical advance in human history lies not far in our future. Soon—within my lifetime, I very much hope, though I don't have all that long left— we will finally perfect the ``vat-grown'' meats and meat products that science fiction has been predicting for fifty or more years now. Meat that's grown with no brain, and comes packaged with no conscious mind that must be snuffed out prior to consumption, in other words. On that great day, perhaps the seven-year-old in me will finally take a day off from weeping at the tragedy of it all and enjoy a nice guilt-free t-bone steak. It'll be the finest one I've ever had, I assure you. Make it medium-well, please!

Some meat-linked eating traditions are easier to understand than others. Despite the fact that thousands of protein-starved adolescent midshipmen of who knows how many navies thrived on them during the Age of Fighting Sail, one can appreciate why most cultures frown on eating rats. My own Ozark-mountain ancestors, perhaps up to and including my grandparents, almost certainly relished a well-cooked opossum. Yet today only a small fraction of American households would even consider eating one; the more one learns of the dining and personal habits of the common `possum, the more understandable this viewpoint becomes. I know of no culture that eats much in the way of voles and mice; they're not worth the effort of catching and cleaning for at best a forkful of meat. But the world's religious prohibitions on meat, well\ldots. They're pretty much beyond me. From where I stand, they look almost\ldots Random.

One other thing I've noticed about food animals over the years is that they seem to get very little respect in our fandom. While many ancient Amerindian cultures are totally centered on the corn plant and it's essential role at the root of everyone's lives\ldots Well, let's put it this way. How many fursuiters do you see at the average convention dressed as a cow or a chicken or a pig? Rabbits excepted, you see almost no food animals of any sort (and you can be pretty sure that the bunnies don't as a rule hold the nutritional benefits of their species-of-choice foremost in their minds). How odd, that the animals we benefit from most of all are the ones that most commonly serve as the butts of our jokes and get the least respect!

Culture, evolution, food\ldots All are intertwined so perfectly and so thoroughly that the closer we examine them the more the threads merge and become one. What a wonderfully complex and mysterious world we live in!
