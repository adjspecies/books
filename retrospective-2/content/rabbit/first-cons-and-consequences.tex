\articlehead{First Cons and Consequences}{Rabbit}{2013}

It's often said that the worst day fishing is better than the best day working. In my life, the same can generally be said of fur-cons. While I haven't kept actual count I've probably been to fifty or seventy of the things, and am fortunate enough to financially be in a position where, if I have a weekend off and there's one within driving range, I can usually go. I consider Mephit to be my ``home'' con, and that probably says a lot about my taste in conventions. I prefer them small, intimate and inexpensive. While I've been to and enjoyed Anthrocon several times and will probably eventually be back, well\ldots I don't know. Compared to the small cons the large ones feel impersonal. Commercialized. More about flash and gee whiz and Big Name Furs than ordinary people sitting around and making new friends.

Back when my columns were posted in other venues I sometimes wrote reviews of this and that con. Then I ceased doing so when I realized that these columns were garnering me a sort of ``VIP treatment'' that I didn't seek. (About the third time the chair of a con looked me up to ``make sure I was having a good time'' I sort of figured it out. And sure enough, when I quit writing these reviews the con-chair visits ceased.) All I want at a con are reasonable prices, ideally between twenty and three hundred furs to interact with (preferably a few of which I already know), a couple-three fursuiters running around from time to time, and some good (preferably writing-related!) panels to go to. The rest sort of takes care of itself, and most cons are well advised not to mess too much with this winning formula.

I've always been especially interested in attending ``first-time'' cons. There's an extra-special sort of magic at these, as a rule -- even the things that go wrong usually just add more ``flavor''. I suspect this is because first-time cons have first-time con staffs, for the most part, who are fresh and unjaded and eager to make ``their'' convention work well. (I was at the first Rain Furrest, the first Furry Fiesta and I think the first MFF, among several others.) Everything is new and exciting to everyone, including many of the attendees, and a sort of magic fills the air.

Usually.

I have seen first-time cons crash and burn, however. It takes work, but I've seen it managed. So I'll complete this column by telling a few woeful tales and offering advice.

\subsection*{1) Registration}

The very first thing congoers experience of a con is Registration. Organizing all those badge sales is difficult, low-profile work performed by people who are in most cases going to miss large slices of the con as a result. (I try to make it a point to recognize whoever registers me for the thankless role they've volunteered to accept.) Sometimes poor Registration experiences are inevitable -- computers crash, printers fail, etc. But, what's not inevitable are gross social and customer-service errors. I recently had the worst Registration experience of my life, when I was left standing at the desk for at least three and perhaps as long as five minutes while the Registration staff totally and completely ignored me even though I was the only one there. The staff spent the time chatting and working on some sort of craft-type project even though I was standing less than two feet away looking at them. Finally, at long last, the person sitting opposite me asked what I wanted. ``To buy a registration,'' I responded.

Their eyes went wide. ``Oh!'' they said. ``I thought you were with the (tool-related) convention! You're wearing a work shirt!'' Then they proceeded to ignore me again for at least two more minutes.

And so, because I was wearing a work shirt (and probably because I'm a good bit older than most con-goers) I started this con off on a totally bad foot. So bad, in fact, that for the first time ever I resolved to inform the con chair about how badly things were being run in Registration.

Then, a little later, I learned that the con chair was the person who'd ignored me.

Which leads well into Point Two\ldots

\subsection*{2) The con is about the attendees, it's not about the staff or the Guests of Honor}

I attended both opening and closing ceremonies at this same con. I usually attend neither, as I generally find them boring. But this time I attended Opening Ceremonies in order to learn what the Con Chair looked like, and then Closing for the same reason that one gawks at a car wreck. At both events the speakers attempted to improvise instead of working from set notes, and the resulting chaos was all too predictable. In the end little to no useful information was transmitted. The staff spent most of the time referring to and congratulating each other instead of interacting with the attendees. They spend some time tossing candy/whatever into the crowd, but the products were thrown hard enough (and some were heavy and sharp-edged enough) to cause potential injury. Several of us attendees -- total strangers -- met each other's eyes and shook our heads at each other; it wasn't just me who disapproved. Apparently, pretty much everyone understands that blinding your guests is a poor way to begin a con.

Another thing I've sometime seen at cons, though not this specific one I've been citing, is a GOH who does their best to sabotage the proceedings. I've seen GOH's do truly awful things, like get so drunk that I've personally had to give multiple panels for them. But worst of all is when a GOH gets the idea in his or her head that the attendees are there for them instead of the other way around. A GOH, in my opinion, owes an even greater debt to the con and its attendees than any staff member save perhaps the Chair him or herself. They're being honored in a unique and what should be humbling way. GOH's shouldn't just be willing to provide art/stories/whatever. They should actively make an effort to circulate, shake hands, and for heaven's sake show the unwashed masses that they're pleased to be honored! Good GOH's can make even a mediocre con memorable. Poor ones can make a wonderful con disgusting. Therefore, it's essential they be selected carefully and have a clear understanding of their vital role.

\subsection*{3) The Hotel Employees}

It's natural that the hotel employees, especially for a Year One con, should stand with wide eyes and be amazed at the wonderful weirdness of it all. They're part of the con too, so why should they not enjoy a little of it? Indeed, I try to take the time to speak with them in a friendly way and explain what I can, when I can. Con Staff should absolutely do the same at every opportunity. Perhaps it's because I'm blue-collar myself and therefore I'm extra-sensitive to such things, but I don't often see Staffers interacting with the hotel workers in a fraternal way. People may not be aware of this, but when they give snippy, hurried instructions to someone they assume ipso facto must be stupid or they wouldn't work at a hotel, well\ldots It's insulting as hell. This doesn't so much cause problems for a Year One con as it does down the road after repeated exposure, but I mention it here anyway because it needs to be dealt from the very beginning.

Hotel workers may be low-paid, but they're intelligent, sensitive fellow human beings asked to keep a straight face at some pretty outlandish stuff. They've got full, rich lives and interests of their own. At one con, for example, I met a waiter originally from New Orleans who had personally met most of the biggest names in Jazz, was probably a bonafide expert on the subject (I'm not qualified to say) and kept a huge private music library. Such individuals deserve as much respect as any congoer. Again, as a blue-collar guy myself I'm uniquely positioned to note that it doesn't help in the least when the person being snippy is half their age. And I'm also uniquely positioned to inform you that payback can be hell.

Trust me. You'll never regret making friends in low places. Especially at con hotels.

\subsection*{4) Programming}

It's incredibly tough to set up programming at a first-year con. Usually there are few rooms available, and often even fewer credible Subject Matter Experts. I always prefer more panels at a con rather than less, on the grounds that then I always have something interesting to do. Therefore that's what I suggest to the first-year programmer. A poor panelist, so long as they're polite and civil and smile a lot, is generally better than no panelist.

On the same note\ldots don't ever ask a panelist to share a room with another panel, or give a panel in a place like the Hospitality Suite. You're asking the impossible in such a chaotic environment.

\subsection*{5) Atmosphere}
This one's tough, but I'll give it a shot anyway. I don't know about others, but I can walk into a room full of people and in a matter of seconds know if they're bored, happy, hostile or whatever. If you think about it, a very large part of the con experience takes place in the meeting rooms and other public gathering places. It doesn't take a con staffer, especially the chair, two minutes to physically go to these places, ``sniff the air'', and then if necessary do something to improve the situation. He might ask a GOH fursuiter, for example, to swing by the gaming room if it's ``dead''. Or, if the ``social'' area looks slow, he might sit down and chat with a few individuals, smiling frequently. Atmosphere is an elusive thing, yes. But you don't have to be passive and accept whatever comes. Go out there and do something about it!

And that's pretty much it, I suppose -- Phil's take on How Not to Totally Screw up a First Con. I hope someone, somewhere has a better time for it having been written.
