\articlehead{Of Rabbits and Rayguns}{Rabbit}{2012}

I've been a semi-professional writer for many years now. I published my first short story in about 1999, and since then have written and published about twenty-two or twenty-three vaguely book-length thingies (I tend to write a much higher percentage of novellas than most authors, frequently sold to the public as shorter than standard books) plus I've lost count of how many shorts. While I mostly write furry and furry SF, I also do horror, non-furry SF, essays, fantasy and even good old conventional literary fiction. So I know the publishing industry a little, or at least I think I do.

Over the past year, I've achieved my first real landmark success in the David Birkenhead books, in which a young slavebunny is manumitted and finds success in a future navy despite all the social roadblocks that lie in his way. Each book in the series is titled after the rank that David achieves during the course of the story, from Ship's Boy to Admiral. Currently my publisher and I are ecstatically selling well over a thousand copies a week. So far so good and yay for me, right? Maybe.

As I said, I've been dealing with the publishing industry—and through it, in theory at least, the wants and needs of book-buyers—for some time. Over and over again the ``big guys'' have rejected my stuff even though in my own opinion many of my previous works have been better than the Birkenheads. More often than not I got a rejection letter saying something like ``Good work—you're writing at the professional level. But we don't publish talking animal stories.'' Then they'd add nitpicky comments about my personal style that usually were in direct contradiction to the nitpicks I received the time before. It was obvious that the ``talking animals'' were the real problem, especially for my serious furry SF stuff, even though the science involved wasn't just clearly justified in the work in question but was often the root of the theme and the driver of the plot. Without the talking animals, which in my case were usually rabbits, the book was nothing. And because my own best ideas most often are equally rooted in furry imagery, well… It was clear that something would have to give. I could either compromise my art to meet the unspoken ``no furries allowed'' rule at the big publishing houses—which their other comments made clear was what they really wanted—or I could be true to my muse and accept that I'd never do business with the mainstream book industry.

So I set out to beat the big publishers at their own game and sell furry fiction despite everything the closed-minded bastards could do to stop me. At this point I had written three multi-book series that I truly believed were of professional quality and could sell well if given the chance. (One of these three was the Birkenheads.) I sent one each to three small publishers (two of them furry-fandom based) and commenced hostilities against the mainstream. As of today, I can almost claim victory. The Birkenheads, as mentioned above, are currently selling at a rate well in excess of a thousand a week and, as I type this, all seven volumes are in the Amazon Kindle Science Fiction Top 100. Sure enough the other two series are now selling well too, though not at the same blistering rate. I can claim complete vindication and declare victory.

Almost.

There's still one major thing wrong, which I didn't anticipate and perhaps should've. While most of the reviews of the Birkenhead books are embarrassingly positive, a very significant minority of readers apparently are driven to the point of rending their clothing and gnashing their teeth when, about five pages or so into Ship's Boy, they discover (having failed to read the book's ``blurb'', which would've warned them) that they've been reading a story told from the viewpoint of an uplifted rabbit (capital-R Rabbit, in the nomenclature of the series). Something about this is apparently hideously repulsive to them at the core identity level, to the point that in their reviews they do things like claim my parents must've been alcoholics. (No, I didn't make that one up.) Others state that this is proof that I know nothing of science—apparently ray guns and FTL travel are far more ``scientific'' than genetic engineering and species uplift. Some go so far as to call the work well-written, but say it's ``impossible'' for them to get into the character. One even suggested it'd be a good book if the protagonist had been an alien instead.

It's easier to relate to a complete alien than a Rabbit?

Don't get me wrong here. At this point I can gaze serenely upon such reviews and laugh all the way to the bank. I'm being well compensated for acting as a public target, and by far the majority of the comments are supportive and positive. I'm not after sympathy here. Rather, I'm seeking to understand something.

Why exactly is it that some readers react so strongly, even violently, to the idea of a serious story being told from the viewpoint of a Rabbit?

Usually in these columns I try to offer some sort of insight—possible answers to the questions I raise. This time, however, I find myself pretty much at a loss. All long- time furs recall an era when our fandom was seen as something sick and repulsive, and indeed this period may well not yet be over. For when I read the public Amazon reviews on Ship's Boy and to a lesser extent the later volumes of the series, I see the public's reaction to furry in microcosm. Mostly these reviews reflect open- minded support and people who like to have fun, yes. And bless them for it! But I also see a large minority who for reasons I literally cannot imagine instantly become totally unhinged at the idea of seeing the world through anthropomorphic eyes. If you read their comments, it's like they felt the need to ritually purify themselves by expressing immoderate outrage and indignation after exposure to such an unclean influence. Theirs is an attitude that cannot be reasoned with—it goes far deeper than that. The subject fascinates me; I think there's real knowledge to be gained about basic human nature if a way can be found to research the matter systematically.

And I hope this research someday is indeed actually done. Because only now, after declaring war on the mainstream publishers and then beating them in terms of sales do I realize that the real battle is only beginning. The one the publishers knew about all along, and were rejecting me over. And why shouldn't they? After all, it's not their fight. Why risk having their products plastered with negative reviews that resonate with John Q. Public despite their inherent irrationality?

The real fight, I've learned from all this, is the one that will someday allow everyone everywhere to understand that it's okay to experience the universe through whatever set of eyes they choose. Everything else is merely a subset of this larger battle, and it'll be more a matter of freeing people from themselves than anything else. Combat is only just beginning in earnest, and I'm not a young man anymore. I probably won't live to see the end of it. But by golly I mean to get my licks in regardless!
