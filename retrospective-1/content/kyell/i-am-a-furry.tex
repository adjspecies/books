\articlehead{I Am a Furry}{Kyell Gold}{2012}

I wrote a blog post recently about how we shouldn't be afraid to tell our friends that we're furries, and I got a thoughtful question on FA. Namely, why bother? It's just a hobby, right? Do we ``come out'' as a stamp collector, or a Man United fan, or a Jane Austen fan?

I said in the original post that I didn't necessarily want to compare coming out as gay with coming out as furry. The first is a preference coded into us at birth which dictates many aspects of how we live if we choose to live with a partner. The second is a not-fully-understood aesthetic appreciation for animal-people that can range in degree from a guy who likes to talk about Looney Tunes cartoons with his friends to a woman who makes a living designing fursuits and wears her own every chance she gets. But it's telling that when people talk about telling their friends and family that they're furries, that the phrase coming out is more and more commonly used.

It's understandable. It means ``revealing a part of ourselves that was hidden,'' and because gay people were the ones most commonly hiding important parts of their lives well into adulthood, it's been associated with revealing one's sexuality. I think that its use in talking about furry is not so much connected to the ``hidden'' part as it is to the ``important'' part.

For a lot of people, furry is more than just a hobby; it's a home. Some people don't have any other homes; some people are perfectly happy with their family in one setting, with their office ``home'' in another, with furry in their spare time. What I mean by ``home'' is a place where you feel safe, where you feel sad to be away from, where some of the closest people in your life reside.

When I was first getting into the furry fandom, I had a friend who came out to his parents and was kicked out of his family. To a lot of guys in their early 20s, that would be devastating, and he was pretty broken up about it. But he had a boyfriend, and he had the furry fandom, a great support network that made sure he always had a friend around and an ear to listen to his troubles. That's what I mean by a home.

Right now, I have a family who aren't furries. But most of my closest friends are furries, and when Kit and I got married, the furry stuff was pretty much all over our wedding (because our wedding planner, a non-furry, fell in love with it). I have a furry image of myself as the lock screen on my phone, a furry pic of me and Kit as my phone background, so literally a day doesn't go by that I don't see some furry art, and now that I'm making my living from writing -- largely in the furry fandom -- most days I end up talking to other furries or talking about furries.

If your life is like that, if you have a group of close furry friends, and yet you're not sharing that part of your life with other people close to you, then you're hiding something from them. You're not sharing all of who you are. And that's fine, honestly; if anything, people these days tend to overshare. But if you want to tell them, and are simply not telling them out of fear that they'll jump to conclusions, then you're doing them a disservice. More than that, you are hurting yourself. When you make choices in your life, such as to continue to be part of the furry fandom, and then hide those choices from other people who are important to you, you are telling yourself that you doubt your choices. You are telling yourself that those people would be right if they mocked you for being a furry. That's not a healthy way to live.

(And yes, there is adult stuff in the fandom. You don't have to talk about that. What do you do when you go to conventions? What do you talk about online with your furry friends? Are adult pictures and stories really the reason you continue to be part of this community? Or is it the people, the ones you feel you can really open up to, the ones who make you laugh and who talk video games, who have a costume like you or like the same movie/TV show/anime? That's what you want to talk about. Everyone understands ``a group of friends who like the same thing I like.'' What you all like is also interesting, but secondary.)

That's who my original post was aimed at, people who cited the primary reason for hiding their furriness as ``I don't want to be associated with those people in the news.'' If you're a casual furry, or if you're distant from your family and non-furry friends, then sure, they don't have to know. But if one of your family, your co-workers, or your friends is trying to get to know you better, and they ask ``why'd you go to Pittsburgh?''… well, before you automatically say, ``just to see friends'' and change the subject, pause for a second and think. Maybe that's a good time to ``come out.'' Maybe that'll help you get closer to the other people in your life. You might have to take a little teasing, but take it with good humor, and it'll be fine. As I said before, as K.M. and I have said on the podcast and many people have said in many venues over and over: if you act like it's something to be ashamed of, people will pick up on that. If you act like it's a cool thing, fun, and a positive part of your life, which I think for most of us it is, then that's how your friends and family will view it. And isn't that what we all want?
