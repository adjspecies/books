\articlehead{Fursuiting Magic}{Rabbit}{2012}

A few years back , I had the opportunity during a large convention to converse with a fursuiter whose talents and antics I admired greatly. Long ago, you see, before furry was much of an organized fandom I was a suiter myself. As we spoke, just a few feet away two dogs and some sort of feline were playing gently with a little boy and girl, whose eyes were as big as saucers. All was going well until another suiter -- who I'll very carefully fail to describe -- walked up, took his head off, and asked one of the dogs when the F he was going to come back to the room. And that, needless, to say, ended that. In a flash the moment was gone and the offended parents were dragging their screaming kids away.

``There ought to be a law against that,'' I muttered.

``Yeah,'' my friend replied. ``Abuse of magic in the first degree.''

For fursuits are magical, you know. Anyone who's ever worn one in public understands this, all the way to the innermost depths of their soul. The human mind is built to function via the complex interaction of symbols, and few objects carry more symbolic weight than a ‘suit. Putting one on messes with your head -- and those of others -- by eliminating or masking all the cultural symbol-sets by which others understand at a glance who you are and how you fit in. Your hairstyle, your clothing, your posture and gait\ldots even your face, the most important identifier of all, is whisked away. No one can even tell for sure if you're male or female. All the social markers so vital to everyday hominid life are tossed to the winds, leaving all who see you adrift and in a state of vague unease.

The effect is double upon the suiter himself, however. When he looks in the mirror, he greets a total stranger. A blank slate, in other words, whose duty it is for him to fill out and bring to life. The ``mask effect'' is well known in psychological circles, so I won't belabor it here. Suffice it to say instead that a good costume, particularly a whole-body one like most fursuits, is both liberating and exhilarating. For my own part, wearing one had the effect of sort of turning up the ``contrast'' knob on life. Everything became sharper edged, I grew more aware of my surroundings\ldots Every second of every minute burned itself into my memory. I was alive, in short, fully and completely in a manner that I've never achieved in any other way. I wasn't a very good fursuiter -- in fact, I rather stank at it and that's why I gave it up. Yet\ldots those few hours I spent here and there in suit remain among the most intense memories of my life.

These mental/social identity-softening effects lie at the root of a fursuit's magic, and also explain why it's so powerful. Nothing affects how we see the universe more profoundly than our viewpoint, and nothing I can think of short of hallucinogens in massive doses does more to alter said viewpoint (on the part of the suiter and bystanders alike) than a fursuit.

This is particularly true of children, whose grasp on reality is still not yet all that firm to begin with. After all, are their TV screens not alive with capering animals who smile and laugh and play? The jump from reality to fantasy is much smaller for them, so that the mere sight of a decently-made fursuit can transport them into a sort of delightful alternate world.

``Yeah, yeah, yeah,'' I can hear my patient readers thinking just about now. ``We all know that clowns and such have a special role in society, and that it's wrong to break the illusion and ruin the magic.'' But\ldots As furries, I suspect we've grown a bit jaded to the wonder of it all. We tend to forget how powerful our magic is, like the unnamed fursuiter in my opening example, and abuse it with more terrible effects than we know.

Back when I fursuited, the only venues available were kid-related activities. The first major one I ``worked'' was playing the Easter Bunny at an orphanage for abused little girls. It was a pretty tough gig, because most of the residents were in their teens and thought a guy in a bunnysuit was about the uncoolest thing possible. But one little girl about six or seven years old -- I'll call her ``Alice'' -- met me with wide eyes and asked ``Are you a real rabbit?''

``Of course!'' I reassured her. Then for the next hour she prattled on and on with me about what it was like to be a bunny while I improvised. ``Of course I eat grass! But only cooked grass -- I'm a civilized rabbit!'' Then the Easter Egg hunt was finally held, and while Alice was off hunting eggs (she won!) the orphanage's counselors closed in on me en-masse and demanded to know what I'd been talking to her so long and intently about. At first I was angry -- it sounded like they were accusing me of something pretty terrible.

``No!'' they explained. ``It's not that at all. You see, she was terribly abused about six months ago, and hasn't said a single word to anyone since.''

When Alice came back with first prize she was still full of chatter, and when I left she was energetically telling her counselors all about me. Half of them were crying.

And so was I.

So\ldots Fursuits are magic of a very special and sacred kind. No one will ever convince me otherwise. Not only that, but they're potent. Those wearing them carry deep obligations to use their powers responsibly. Far too few seem to understand this, in my book.

And the fursuiter who used the F-word in front of children with his head off in public? He told me to F myself a few minutes later, when I tried to explain my concerns. It's too bad that every story can't have a happy ending.
