\articlehead{Talkin' About Our Generation}{Rabbit}{2012}

\textit{This is an edited reprint of an article that first appeared in Anthro \#11.}

It sometimes feels like I've tried to spend most of the last ten years of my life trying to explain the fast-growing Anthropomorphic Animal, or `Furry', phenomenon to outsiders. Yet the trend absolutely begs explanation. Attendance at furry-themed events is doubling roughly every three years, fur-fans (or, simply, `furs') are becoming a highly-visible presence in many online communities, and more and more anthro-themed marketing campaigns appear every day.

So what's behind the sudden explosion? There have always been anthro-themed ad campaigns, as any consumer of breakfast cereals can testify. Practically all of us grew up with Tony the Tiger, Toucan Sam, and Sugar Bear. Nor are anthro characters anything new in entertainment, as attested to by Tom Cat, Jerry Mouse, Pepe le Pew, and Speedy Gonzales. Children have been sleeping with stuffed animals at least since the time of Teddy Roosevelt, and as early as 1922 they were common enough to serve as a powerful literary symbol in the classic story The Velveteen Rabbit. Even long before modern times, humanized animal characters occupied an honored place in the human heart; where would Aesop have been without them?

Yet there's clearly a new dynamic at work today. Artists have drawn anthropomorphic creatures before, but never in such mind-boggling variety, or to such an appreciative audience. And, more pertinent to the blog you're currently reading, writers have written tales in which half-animals appear since the very beginning of things. But never before has the usage of such characters been so widespread or executed so skillfully. Never before, in other words, have authors so openly and unashamedly incorporated anthropomorphic characters into works intended for adults, written with an adult level of depth and sophistication. Or, at least, it's never happened frequently enough to be noticeable as an artform in and of itself.

Which brings us right back to our original question. Why now? Why is furry fiction taking off and growing legs today, after lying near-dormant for so long? Why are the adventures of cartoon-like bunnies suddenly acceptable as the stuff of serious novels, instead of for Saturday-morning-only consumption? There are two important and largely unrelated reasons for this, I think.

Everyone knows that children are very open to the power of suggestion. During childhood the human mind develops like nothing else in nature, desperately attempting to gather and incorporate everything it needs in order to master the environment around it. This adaptive process runs far deeper than merely mastering the art of counting to ten and learning that cows go moo-moo. In one key phase of development, for example, infants become obsessed with the human face and will spend hours either staring at the faces of others or else scrawling increasingly human-like visages on whatever surfaces happen to be handy. During this period the child is among other things learning what is human and what isn't, not just how to read faces but what a face is and what it represents. The child is, in short, defining itself as a member of a group of others like it. Yet it is during this same key developmental period that children are perhaps most exposed to anthropomorphic animal images in the forms of stuffed animals, picture-books, and animated films. Furthermore, the level of exposure has increased dramatically both in volume and `quality' (via television, DVD player and VCR) over time. Would it be any wonder if, surrounded by more and more anthro images during a critical developmental stage, kids began to blur the lines a little in learning what is human and what is not? Would it take a miracle for a substantial and growing percentage of kids raised in this way to grow up feeling most at home interpreting and understanding the universe through the eyes of half-animal characters? Might children raised in such an environment develop an otherwise inexplicable attraction to anthropomorphic art and fiction as adults? Indeed, wouldn't it be even more surprising if, exposed to such a saturation of anthropomorphic characters at such an impressionable age, said characters didn't come to play an important role in their inner lives?

The second key factor behind the new explosion of interest is, I believe, the Internet. For the first time, people who admire serious anthropomorphic art and literature have been able to find one another and share their creations. The pent-up potential is finally being released, and the result is the veritable explosion you see today in the anthropomorphic arts.

Furry art is still not for everyone. However, it does seem to be for more and more of us every year. Given the stratospheric average IQ among the furs I know and their tendency towards careers in professions such as IT and the sciences, it's fair to say that the cultural impact of anthropomorphic art is not only well out of proportion to the numbers involved but continually rising. Today, for example, ethicists blanch at the idea of merging human and animal characteristics via gengineering. But tomorrow, who knows?

The future may be closer than you think. And it just may be brought to you by a guy who likes to look at pictures of horses walking around on two legs\ldots
