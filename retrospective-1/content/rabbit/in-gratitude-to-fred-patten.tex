\articlehead{In Gratitude to Fred Patten}{Rabbit}{2012}

Hello!

First of all, I suppose I ought to introduce myself since I'm new on this block. My name is Phil Geusz, and I've been around the fandom more or less since about 1997. I wrote my first novel that year, and haven't spent much time not-writing since. I'm one of those people you hear about for whom the discovery of the furry fandom was a life-changing event, and in my case the change was all for the better. Fifteen years later, I've either published or am in the process of having published twenty-one mostly furry novels and novellas. The fandom has brought me happiness beyond measure and sparked a creativity inside myself that I'd never have unlocked on my own. I'm grateful to you all, and these columns, like the ones I've written for other furry publications, are meant to at least partially serve as a form of repayment. It's wrong to take, take, take and never give.

That taken care of\ldots

As an author, I'm far more aware than most that we live in rapidly changing times. Even a mere decade ago, when I first began attempting to sell my fiction in a serious way, the publishing world (or at least the significant money-making part of it) was ruled by a handful of editors and agents. These individuals served as ``gatekeepers'' or ``herd thinners''; in choosing who and what was published, printed, and then shipped out by the railcar load to the nation's bookstores, they effectively controlled the nation's literary tastes and (much like the record labels) which artists grew rich and famous and which didn't.

Then, however, came the internet. Anyone could put anything on a web page. And nothing was ever the same again.

For the most part, the internet has been a good thing for society as a whole -- certainly its been good for our fandom, which in my opinion would at best be a tiny ghost of what it is today without the interconnections the web has given us. So, please don't take what follows as an anti-technological rant. I'm no luddite by any measure! And yet\ldots

\ldots sometimes when I surf the unwashed, unedited, ungrammatical, unspellchecked, unstructured and uncensored recesses of furrydom's fiction websites, I have to admit that I find myself wishing for an old-school gatekeeper or two. I mean, face it. Surgeon's Law informs us that 99\% of anything is trash, and this has certainly always held true for me. Yet\ldots  How long and how hard must the average fur search in order to find stories that suit them?

The problem, you see, is that the internet is ``flat''. Everyone has equal access, in the absence of gatekeepers and the like, and therefore everyone's work has equal prominence. Yet\ldots  The fact of the matter is that everyone's work doesn't \textit{deserve} equal prominence. It's commonly understood by pretty near everyone that in order to become a top-flight athlete, for example, endless hours of hard work and coaching are required. When it comes to writing (and in my experience visual art as well, though I claim no expertise there), however\ldots  Somehow would-be authors don't want to accept that they need as much coaching and dedication as an athlete in order to perfect their skills. Instead they internet-post what in all honesty is often grossly substandard work, where it ends up just as prominent as the very best of the genre. And so, Joe Furstoryreader must wade through what feels like an endless swamp in search of quality stories.

Please, don't get me wrong here. I'm all in favor of encouraging new writers, and have done more than my share of exactly that over the years. Nor do I have a problem with people being proud of what they've created and wanting to share it -- I've not forgotten where I began myself. But back then I desperately needed a writing coach too -- I still do, in many ways! -- and my work certainly wasn't ready for prime time. All I'm trying to establish here is that there's a crying need in our fandom for individuals willing to read boatloads of furry fiction and then tell the rest of us that work ``A'' was excellent, ``B'' sucked, and ``C'' was somewhere in between. Such individuals are far too rare, especially in the flat, gatekeeperless world of the internet. The ones we \textit{do} have, however, are worth their weight in gold to our fandom, and in my opinion we've done far, far too little to thank them for their often painful efforts.

Fred Patten is such an individual, and my purpose in writing this article is to thank him for the many services he's performed along these lines for our fandom. Not only did he edit the furry fandom's first significant collection of short stories (plus two other more recent ones), he can truly be said to be one of our founding fathers. (He also helped found the anime fandom as well. How many other people can claim such a distinction in two disparate fields?) Even today, though suffering from the effects of a severely debilitating stroke a few years back, Fred continues to pound out well-considered and thoughtful story and book reviews like a machine. Thus, Mr. Patten serves as the modern-day internet equivalent of the old-time magazine editor -- a reader looking for good stories need search no further than Fred's bottomless stack of reviews in order to locate the sort of material they seek. In the old world of physical books and paper-publishing, valuable people like Fred were well-known to authors and readers alike and honored and respected accordingly. The ``flat'' world of the internet, however, isn't so kind. Only in Australia has Mr. Patten and his multiple contributions to furrydom been recognized; so far as I know in the United States and Europe he's practically a nonentity.

And that's a damned shame, or maybe something even worse. It reflects very badly on us indeed.

Every single fur contributes to the furry fandom in one way or another. But some contribute far more than others. Fred Patten has given more to this fandom than any other single individual I can name, yet because his efforts have been focused on criticism in a time and place where the role of the critic often goes unappreciated, well\ldots  He's been under-appreciated too. So much so that I'm dedicating my first column here to pointing the fact out, in the hope that someone somewhere will do right by the man while he's still with us to appreciate it.

We furs have a reputation for taking care of each other, and seeing that justice is done. I can only hope that someone who is in a position to offer more than mere words will read this and live up to the best traditions of our fandom.

(Author's note -- For the sake of full disclosure\ldots  I've never met Fred in person, but have corresponded with him regarding various literary endeavors for roughly a decade or so. It should be noted that he's reviewed many of my works, mostly favorably.)
