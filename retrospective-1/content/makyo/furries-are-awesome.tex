\articlehead{Furries Are Awesome}{Makyo}{2012}

First of all, I'd like to apologize for the dearth of articles, recently. It really weighs on me, and I feel that I've been neglecting one of my favorite things ever: writing too-long articles about animal people.  Not all of my time was just sitting, twiddling my thumbs, though.  I did wind up with a cool new job, and that panel for RMFC took up quite a bit of my time, actually.  Most of what has been going on, though, at least in my spare time over the last few weeks, has been dealing with a few health problems that had me a little down.

My general solution to the anxiety and emotional weirdness involved with those sorts of things, when they get bad, is to seek out as many positive (pawsitive, if you will) things.  The usual method is to ask on Twitter ``what's awesome right now?''  I love getting the responses, hearing what people think is neat and cool, hearing all the wonderful things that are happening to people.  ``Exciting new development at work for me!''  ``Free bagels.''  ``My coffee.''  As I poke my way through the replies, though, favoriting most of them, I notice that just about every icon has a muzzle and ears.  So you know what's totally awesome to me? Furries.

A lot of what this site focuses on is not really all that negative.  Zik is exploring the world of furry, JM is peeking into specific aspects of our subculture, and Klisoura is being wonderful by pulling specific data for us. (Kyell is automatically awesome, because fox; and guest authors get a free in, here.)  However, the topics tend to be obviously interesting, and I've noticed that we do tend to approach issues from both sides, even if we wind up more firmly on one side than the other.  JM's articles on Zoophilia and cub porn both take this tack: they start with an  exposition of both sides, even though they tend to come down on the positive side.  It's definitely a successful method, and it seems that a lot of our readership does appreciate the more exploratory style articles.

I'm going to take a step back, though, and just spend a few words on some blatant positivity.  I really like furries.  I really like  being a furry.  I think we are, all around, a great group of people focused on a few great core ideas, but with plenty of diversity thrown in to make sure that we lead interesting lives.  We are awesome.

There's a rhetorical technique known as hendiatris, which is one of those things which you will spot everywhere once you know about it.  It means making one point through three statements.  I know that it figures prominently in my own writing, but I see it everywhere.  Especially in this most political of (US) seasons, the hendiatris makes a comeback.  I'm going to use that here, and the reason I'm even bothering to preface that is that I want to note that I try to fit all of my articles into three categories: participation mystique (how we base a portion of our identity off our membership with the fandom), character versus self (the concept of creating and interacting via an avatar), and interaction (what we gain by being a subculture, rather than being solipsistic).

\subsection*{Participation Mystique}

Participation mystique is basing a portion of your identity off of membership to a group or participation in some sort of idea.  I've written about it before, but it's worth bringing up again specifically for the benefits that it offers within the furry community.  The idea that we can structure a portion of what we consider ourselves around our membership to this sometimes quite odd subculture is quite impressive.  I know that, for myself. I feel that I would be a less complete individual without the fandom.

That's part of the issue with anything that uses the words identity, though.  By their very nature, they are things that, without which, we would find it nearly impossible to picture ourselves.  If I try to picture myself without furry, for instance, I come up with a blank for several parts of my day -- checking Twitter, relaxing online with friends during some downtime, planning for a convention panel, or even right now, sitting and writing a meta-furry article for a blog with a giant wolf on the banner.  Without furry, would I substitute that portion of my identity with something else?  Would I have taken part in some other participation mystique that would have filled out the same spaces in the topology of my soul?  I'm sure it's possible.  There are a lot of things that I'm interested in besides furry, to be truthful.  Would I be the same person, though?  Of course not.

There are, as someone mentioned to me on Twitter, inherent ties between the fandom and identity.  It's not just that I am experiencing this sort of participation mystique, many of us are.  There is a certain sort of subconscious, unvoiced togetherness that we gain from sharing this mystical participation, this joining of ourselves with a group.  It sounds a little cultish, when I write it out like that, but I do think it's true.  I've noticed that, if you run into a furry that you have never met before, there's always at least one thing you can talk about: the ways in which you base a portion of your Self on your being a part of this larger group of animal people.

\subsection*{Character Versus Self}

Character versus self is another theme that I've written on before.  There are several ways in which we interact with the world around us, and one of the most important for us within the fandom is through our own characters, those avatars which stand for the core of our being tied with our interest in anthropomorphics, as well as our identity in the fandom.  It came up during the RMFC panel that many furries can even have several different characters, as opposed to just one avatar that they keep.  That we can hold that in our minds, that we can wear a mask to fit our moods and our desires, to be the type of individual we want to be, that is quite amazing, I think.

To paraphrase a friend, we put so much work and creativity into creating something that represents our most intimate of aspects, and then we wear it openly, making that the type of person with whom others should interact.  You all know that I write and care about gender and all of the complexities involved with it (I can think of at least two articles that have surrounded it that I've written, after all), and I think that this idea of taking a personal aspect, much more personal than might be normally shared outside of the fandom, and making it a core part of the character that we create is definitely useful.  Gender can often be one of those things, where one can play a character of whatever biological sex, or even gender identity, that they want here in the fandom, and have it be just fine.

Another example, and a good way to tie into the next section, is the ways in which we benefit from having an avatar through which we interact.  There are, of course, varying degrees of introvert and extrovert, and beyond that, varying degrees of social anxiety.  These are things that just about everyone experiences, even if it's on the extreme far end. I can say for myself that, although I like to think of myself as reasonably extroverted, I have quite a bit of social anxiety, and it takes a lot of effort for me to have successful interactions in the world.  If I'm pretending to be a fox or whatever, though, I can hide behind the fact that I'm doing just that, and the interactions go a lot smoother.  Perhaps it's just the fact that I'm interacting with other furries, but I do feel that having that layer of Who I Really Feel I Am between me and my interlocutor does provide an additional level of comfort.

\subsection*{Interaction}

The idea of a chosen family is not a new one.  I know that, at the very least, it ties into the idea of being kicked out of one's home, and adopting a chosen family of sorts to help be the surrogates for those whom are no longer in ones lives.  Even beyond that, however, I think that the idea holds true within furry.  There is no one in my family with whom I am closer than some of my friends in the fandom.  The fact that my chosen family here, outside of my normal family whom I still love, can continue to grow and change just tickles me pink, too.  I can honestly say that, within the last two weeks, at least one additional member has been added to this family, someone with whom I am more comfortable talking to than most members of my blood-related family.  This always amazes me: the mutability of who we consider family is odd enough, but within the fandom, just how quickly those relationships can grow.

I'm not alone in this at all, either.  I asked on Twitter, before I started this article, what the most positive thing was that my followers could think of the fandom, and the majority of the answers revolved around the interconnectedness and relationships that spring from it.  ``Made so many good friends,'' ``Given me [\ldots] a husband,'' ``that I am not being judged or ridiculed for who I am.''  These are all, to me, true signs of affection for the other members in our subculture.  That we have not found, but created an area where all of these things can be the case is quite singular, to me.  Of all the other subcultures to which I'd consider myself a member -- programmers, musicians, awkward people -- I don't think that it's likely that I would be able to build a friendship quite as quickly.  Sure, in programming, we can debate the (de)merits of PHP, or in music we can talk about preferences for music to perform versus music to listen to.  Neither of those things (thankfully) take up much of my identity, however.

See, here in our subculture, we combine all three of these levels of participation.  There's the utmost personal level of creating a part of our identity around it, there's the level wherein we create a front-stage mask that may, in some cases, more closely relate our back-stage personas, and there's the level where we actively participate in the little micro-world around us.  So many of us have bought into the fandom (many in more ways than one) that it's become something greater than the sum of its parts.  I challenge you all to do the same and imagine where you'd be without the fandom, try and figure out what theme, idea, culture, or group, or combination thereof, could take its place, and define the borders of furry in your own lives.  We really are pretty awesome.
