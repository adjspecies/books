\articlehead{Art and Money}{Makyo}{2012}

The relationship between art and money is always tense. In fact, one of my favorite books that I read during my time in the music composition department at school was Art and Fear by David Bayles and Ted Orland (which I very highly recommend to any artist readers out there).  They describe the relationship, in part, as ``There's one hell of a lot more to art than just making it.''  The tension shifts in the world of `crafts`, functional art, and the like. The website What The Craft dissects the problem of working with money in craft in two excellent posts, one about why handmade is ``so expensive'' and another about how to price hand-made goods.  In both cases, the author explains that ``[h]andmade goods mean attention to detail, quality craftsmanship, and a significant amount of TIME and SKILL'', which can in turn lead to the higher price.

Furry art, then, fits in a strange place in the middle, what with the ``traditional art'' aspect of a commissioned artist creating a work, as well as the custom, attention-to-detail oriented aspect of handmade crafts providing a visual representation of our characters. I've written before about the how the connection between a visual representation of one's character can affect the way one interacts with an artist, but I spent little time on how the financial aspect of the transaction plays in the scenario.

In order to gain some insight on the matter, I conducted interviews with various artists, asking questions suggested to me by a few others.  The truth is, I simply have very little basis on this to work from in my own past.  I have had exactly one piece of music commissioned of me, pro bono, and it went terribly.  The work I do on my own in web design is a little more expansive, but still hardly worth much in the way of experience points.  Having collected the answers into one place and read over them a few times, I started to notice a few points of tension that stick out beyond simply ``drawing one's character''. I asked questions about how the artists had come up with their pricing schemes and how they interacted with customers, and each showed that a good amount of thought went into their role as furry artists.

When it comes to pricing the work of a furry arist, there seem to be two main ways of going about it. The artist will either come up with a rough guideline as to how best to price their work on an hourly scale – for example, given that a certain type of drawing takes x number of hours, they'll come up with an estimated range for pieces of that type. The other way in which a commission price is determined is by checking prices against their peers and estimating from there. An artist of a certain style and perceived skill level can get a pretty good idea of how much they might charge for work by looking at their friends` work and how much that goes for.

That said, the overwhelming response from those that I interviewed was that furry artists most definitely undercharge for their labor. One artist, Ten, mentions, ``I've been to far too many artists pages', even talked to friends of mine who do outstanding work, and they're all `is fifty bucks too much? That sounds like too much', and it turns out they think fifty bucks is too much for a fully colored custom work.'' Another artist, who wished to remain anonymous, echoed the point clearly: ``I have seen some very talented people charge very little for their work, and I try to point that out when I can. `You could charge twice as much, you're so talented!'\' is what I usually say.''

The question of why many artists charge as little as they do and why they don`t often raise their prices is a fairly interesting one. Certainly one of the reasons that many do not charge more is that it isn`t their primary source of income, but out of the five artists that I interviewed, only two of them had additional sources beyond their own art. So, if many artists are making art in order to support themselves, why is it that there is a general impression of undercharging art?

Part of it, I believe, is tied to the expected consumers of the art, the patrons who pay for the commissioned works. There is an expectation that furries simply will not have the money at hand in order to afford what would be full-price for a similar commission outside the fandom. Rhazafax mentions, ``if it were possible to raise [prices] without losing a chunk of clients, I won't lie, my pocket book sure could use it,'' somewhat supporting that idea, while the anonymous artist mentioned that they ``certainly charge furries less than what [they] would charge at a professional artist level.''

There seems to be quite a bit of mental strife involved in valuing one`s work in terms of dollars, pounds, or yen.  In order to come up with a price point, not only does hourly wage need to be taken into account (the ``am I making enough'' aspect), but also how that relates to one's peers in style and skill level (the ``am I asking the right amount'' aspect).  For those who do it for a living, the point is quite fine, there.  The artist needs to pay for their rent and food, as does the client, and so their output needs to be high enough or of high enough quality; as Sigil puts it, ``you can sell one picture for $100 or ten pictures for $10…which would be more rewarding?''.

But what about the client?

I should be honest that the impetus from this post came from seeing a rash of ``wish I could afford it'' or ``those are cool but too expensive for me'' comments on FurAffinity when an artist opened up for commissions.  I understand the difficulty of finances first hand, having paid my way through three years of college, then going on to buy a house.  Even many of the artists I asked sympathized on some level with these comments.  However, many of those comments seemed to be implying that the artist should lower their prices, even if only for the one who posted the comment.  Ten addresses this directly: ``[I] wish I could cater to their price level, but then everyone would expect alterations for them, and it'd through off my whole point of having specific price points.''

This leads to another mechanism of catering to many when it comes to commissions: target audiences.  Sigil mentions that everyone can save up for a \$20 piece of art, though the sentiment is echoed by many that I interviewed, leading to varied price points for different levels of work for the artist.  These are often exemplified by the ubiquitous pricing sheet (Floe, Ten, Rhazafax, and Sigil – the four named interviewees – all have their own in their galleries).  Another example of a targeted client base was provided by Floe: ``My target audience is repeat customers.  I tend to get better every time I draw them.''  She mentions that her prices are structured around this idea.

All these financial reasons surround this tension, and yet one main economic factor is very much subdued in this market: competition.  Most of the artists that I asked mentioned that competition plays a relatively small role in their interactions with others, often due to style.  ``Furries are going to commission the artists they like and the artists they can afford,'' Ten explains, and Sigil echoes this: ``if someone wants a Sigil picture, they will come to me.''  Even though there may be competition within price range, Floe explains that this is why she strives to build a relationship with her customers.  As a concrete example of this, Floe created our delightful RandomWolf banner for us at the top of the page, and I commissioned that from her last year after meeting her…gosh…five years or so ago, and having received several pictures of my own characters from her.

In the end, some of the tension surrounding money and art may indeed be due to the ``yes, but this is ME!'' aspect of having one's character drawn by another, but there are often simple and mundane reasons at work, as well.  The artists need to make their money for their own reasons, whether to support themselves completely or simply to supplement their income, and the clients need (or want; I say need because I'm so terrible at drawing) art of their characters created by others if they want some sort of visual representation of the avatar into which they've poured so much of themselves.  It's economics at its (complicated, puzzling, sometimes hurtful) finest.  And in the long run, well, we seem to do pretty well by ourselves.

I'd like to thank the artists who provided me with their input, and one of the best ways I can think to do so is to encourage you all to go check out their galleries, they're really awesome!  Their input was invaluable not only in constructing this post, but also increasing my own understanding of the other side of the trade.  If youd like to check out their responses in full, I`ve posted four of the interviews here.  I asked seven base questions, but, of two of the artists, I asked an eighth question that was put to me by a few friends.  Sigil broke this down into two delightful sub-questions that anyone can answer in their own way; feel free to let us know what you think in the comments!  Sigil's in-depth response is available on the interview page mentioned above.
