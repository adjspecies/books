\articlehead{Just Like the Rest}{Makyo}{2011}

I can almost pinpoint the time I realized that furry was just a slice of humanity as a whole, and not some special fandom elevated above the dregs of the world.  I think it came sometime in around 2007, and it probably happened in a text-only, electronic gay bar on the Internet (and I'm pretty sure it was while pretending to man-sized fox wearing a nice suit on the internet, but that's a given).


The subject was girls.  In the Purple Nurple (t tpn on FurryMUCK), this comes up occasionally.  Being a gay bar of sorts, the e-bar tends to attract some very gay people.  Which is to say, it attracts everyone, but since it's a gay bar, most people tend to gay it up pretty hard while there, and so when girls come up, reactions are pretty much as you'd exepct:

\begin{itemize}
  \item The nice folk -- a few who are probably a Kinsey 6, but most who are somewhere less than that -- tend to just ignore the topic.
  \item A few who are feeling pretty snarky or eager to fit into the very-gay scene will pull the “ew girls!” card out and wave it around.
  \item The token straight guy will start “throwing people out the window”.
  \item Any girls present seem to fall into two categories:
    \begin{itemize}
      \item Those with female players will likely roll their eyes. Whether they act that out on the MUCK or not is up in the air.
      \item Those with male players will pout, get defensive, or say nothing, depending on why they're pretending to be a female animal-person on the internet.
    \end{itemize}
\end{itemize}

This sort of scenario seems to come up every once in a while in the Nurple, where females are mentioned in a sexual context among a group made up of primarily homosexual males; and that's not a grammar gaffe: several homosexual males I've met online seem to base a large portion of their personality and social interaction on the fact that they're homosexual.

While I don't remember for sure, what I think happened is that I was dwelling on this as it was happening some time around early 2007.  It was a pretty introspective time of my life, with bits of college working out very well while others collapsed around me in ruins.  I was spending a lot of time reminiscing about high school and the way I had changed as I grew up.  When I was depressed, it would border on “where did I go wrong?”, and when I wasn't, it tended towards “how did I get here and how can I get where I want?”. It was the romantic, introspective springtime of youth that all young foxes must go through at some point or another.

During high school, I had been part of a support group of sorts, OASOS: Open and Affirming Sexual Orientation (and gender identity) Support.  It was a group organized by the Boulder County Health Department, and was made up almost entirely of young men and women trying to find the easiest way to fit into their imagined roles of gay and lesbian, or, more accurately, \texttt{GAY} and \texttt{LESBIAN}.  One of the defining moments of my life came from this group when I met a female-to-male transgender guy by the name of Michael.  The reason this was a defining moment in my life (and part of the reason Michael and I started dating) was because it helped me to understand the difference between sex and gender, and more importantly, how that changed my outlook on how these young \texttt{GAY}s and \texttt{LESBIAN}s were acting within their stereotyped roles.

Something clicked inside, that day in 2007 as I was sitting in a fake gay bar on the internet populated with fake animal people. Being somewhere less than a Kinsey 6 myself, I was one of the ones who kept quiet, and as I watched, I realized that this was OASOS all over again.  These were almost all \texttt{GAY} young adults saying ``ew, girls'' while the \texttt{STRAIGHT} young adult e-threw them out the i-windows.  Those in the Nurple who I had perceived as basing a large portion of their personality on the fact that they were homosexual were really no different than those at OASOS struggling to do exactly the same thing (though, being older, those in the Nurple were probably a little less fraught with hormones and acne -- but maybe not, who knows).

I feel it's important that I say that I love all the wonderful people I've met online and in the Nurple especially, and I really don't mean to cast aspersions on those who hold true to the Kinsey 6s and 0s out there.  My point here is that society contains several sets of roles that, in the western world, tend toward heteronormative. My discovery those years ago was that these roles existed through all of western society and permeated even into my messy little fandom -- furries really were just a slice of society as a whole, trying to carve themselves a new, more exclusive role.  Perhaps this change in my perception began even sooner, though, and the shift in thought was more the final step after a long build-up.

I had been to a few conventions by this point -- I believe AnthroCon '06 and FurtherConfusion '07 -- as well as a few considerably large parties down in Denver and the normal weekly furmeets.  When I had stopped hanging out with furries solely online and moved my interaction to real life as well, perhaps that's when my slow realization began.  It was undeniably fun to head out with a group of people who wore tails and ears, who made their stupid noises and were overly affectionate in public (if not to me, than certainly to the non-furs around us).  It felt good to belong to this exclusive group with shared interests and ready conversations.

After I'd suffered my sea change, however, the boundaries between our little (or big) groups and the world around us started to blur, for me.  I saw the same societal currents moving within the fandom that were moving in the world around me, and I began to see furries more as a group of mostly middle class, mostly western, mostly young adults.

The changes in perspective were subtle at first.  ``Perhaps furry is just more welcoming of the misfits and the minorities than other groups,'' I thought.  ``Maybe the preponderance of homosexuality within the fandom is due to the more liberal attitudes therein.''  Over time, however, these views have changed, though only slightly.  I feel it would be more accurate in both cases to put the sentiments in the subjunctive mood: ``Furry wants to be seen as more welcoming of the misfits and minorities than other groups''; ``The preponderance of homosexuality in the fandom is due to the liberal attitudes the fandom wants to be perceived by the outside world.''

This, of course, makes it all seem a little sinister, though it's nothing of the sort.  This is just the politicking that happens with any subset of humanity in order to increase its chances of survival.  If the western world as a whole is shifting towards more liberal attitudes towards homosexuality and minority groups, then a group can ``get ahead'' by being perceived as having liberal attitudes those things.  The fandom is really just like the rest.

I see this same thing played out time and again within subsets of the community around different issues.  Recently, our local furs went through something of a upheaval due to the very same gender issue as above.  There have been  issues surrounding the use of one site over another, issues over those who like fursuits and those who don't, and even within that, issues between those who like fursuits with certain holes and those who don't.  It's even been claimed that the fandom is more drama-filled than any other group or the society as a whole; a claim that's easily debunked by listening to an episode of This American Life (really, just pick any one, it doesn't matter!) or by watching any news around election season.

Our only real claim to uniqueness is that we do tend to be more interconnected than most other groups of people.  Currently, I would hazard a guess that furry is much more interconnected than most other social groups, thanks to the internet.  However, if you had asked me that five years ago, I would've suggested that it be twice as interconnected.  This is an arms race we're going to lose, and that's okay.  We really don't need to be different or better or more distinct than other social groups; we've cemented our place in western society already and our little supposed enclave is secure for the foreseeable future.  Just that we're all just like the rest, is all.
