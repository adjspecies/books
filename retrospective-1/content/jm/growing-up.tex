\articlehead{Growing Up}{JM}{2012}

It is generally considered that you become an adult at 21 years old. Anyone who is 21 years old or more knows that this is completely false.

We might be physically mature, but there is a big difference between physical and emotional maturity. An emotionally mature person likes and accepts themselves. This takes a lot more than 21 years.

Personal hygiene is hard. Getting up in the morning is hard. Feeling lonely is hard. Managing suicidal thoughts is hard. Holding down a job is hard. Looking in the mirror, and liking what you see, is hard. And pretending that you don't, deep down, consider yourself to be a failure is really hard.

Everyone feels like a failure at least some of the time. Society dictates that we must pretend that we're doing fine, so we hide how we really feel. Everyone else wears the same facade of competence, which means that it's easy to look around and see people apparently doing well. This reinforces our own feeling of isolation, unworthiness, and failure.

As you gain emotional maturity, you learn to take pride in your own self-improvement. You learn that everyone else is struggling too. You learn your real value and you gain empathy for all those other losers out there.

Ironically, for a group of people who identify as pretend animals, the furry community is a great vehicle for self-improvement. This is obvious through observation of our furry friends: seeing those pursuing education, weight loss, jobs, relationships, and other avenues to happiness. The science explains the value of furry as well, through clinical psychology and therapeutic experience.

Furries, unwittingly, act in ways that reflect psychological techniques for self-improvement and the pursuit of happiness. We have a healthy and effective method of managing our internal world, and of improving our relationship with the external world. The first helps us feel better about ourselves; the second helps us grow relationships.

The cornerstone of the furry world is roleplaying. Each of us creates an anthropomorphic animal alter-ego and acts as if this fiction were real. We routinely do this online but we also take this roleplaying into the real world. At conventions, or furmeets, or just among furry friends, we tend to act as our avatar. When you meet me in person, it's entirely clear that I am not a horse (and my passport says Matt rather than JM) however furries will treat me as if I were. I like to hang around with other horses and complain about My Little Pony; I get why-the-long-face jokes; carnivores eye me up in a slightly disturbing fashion. It's as if JM Horse were really there.

I, of course, return the favour and treat my fellow furries as if their avatar were real. (Hungry carnivores can be satisfied, I've learned, with candy. Or beer.) There is a mutual contract to reinforce the fantasy of our furry world.

Through this roleplaying, we are following a common Cognitive Behavioural Therapy technique known as `modelling'. Modelling is a self-improvement technique, usually applied to manage the internal criticism (from an inner voice) that we all struggle with when we are anxious.

A cognitive behavioural therapist will ask a client to identify a rolemodel. The rolemodel will be someone who is skilled at the behaviour provoking the client's anxiety. So if the client is nervous about public speaking, he will be asked to think of an excellent public speaker. The client will then be asked to imagine what it would be like to `be' that person. They will be asked to imagine how that person's clothes might feel, see through that person's eyes, hear that person's internal dialogue.

Several things happen during this process. The first is that people become less self-focussed when they are roleplaying. When people are anxious, they tend to become more self-aware, which is usually counter-productive. The roleplaying exercise makes a connection between the new skill (public speaking) and a state of low self-awareness.

Pretending to be the rolemodel acts as mental mask, which has very similar effects to wearing a physical mask. A 2003 study from the Personality \& Social Psychology Bulletin showed that people are significantly less self-aware when roleplaying (or wearing a physical mask). The effect is more than doubled when the subject is performing a task that makes them anxious.

Low self-awareness means that physical sensations – pain, nausea, heat – are felt less strongly. Awareness of the outer world is heightened. These effects from the mental mask of roleplaying make you feel more competent and confident.

The technique can also be used to try out new personality traits. Therapuetically, this is often used to treat depression or low confidence. The client will be asked to roleplay a happy person, which gives them experience as to how that might feel. Ideally, the client will draw some of the roleplayed traits into their normal lives.

Dr Robin Rosenberg is a clinical psychologist and editor of The Psychology of Superheroes. She believes that cosplaying is a form of self-administered mental health treatment. Dressing up as Batman, say, requires a client to practice acting like their rolemodel. At a sci-fi convention the stakes are low, so an insecure client feels safe to act brash and outgoing. This works as a practice run, giving the client some resources that may be summoned in times of stress.

There is more background on Dr Rosenberg's blog in Psychology Today – http://www.psychologytoday.com/blog/the-superheroes

There are obvious parallels to fursuiting, where an otherwise shy suiter becomes confident and outgoing. The experience of being in suit is liberating, in that you lose much self-awareness and become more engaged with the outside world. The effect is also comparable to furry interaction online.

Online, we can test personality traits in a friendly and low-risk environment. For some furries, online interaction is a stepping-stone to becoming a more confident social animal in real life. Some will roleplay as a dominant individual, which will help them to be more assertive. Some will roleplay as a different gender, which helps gain self-acceptance of an unusual gender identity. And many furries will experiment and discard traits that don't fit.

The best example of the value of furry roleplaying is sexual preference. Many furries experiment with a gay or bisexual persona as a first step towards self-acceptance of their true sexual preference. Data from the furry survey, which forms the basis for an analysis here on [adjective][species] earlier this year, showed that around 50\% of furries re-evaluate their sexual preference within their first five years in the community, typically from straight to gay.

The positive influence of furry extends beyond modelling. The value of `play' towards self-improvement and maturation has been the subject of increasing research in recent times. It is understood that children develop social and cognitive skills through play, improving self-confidence and maturity. The benefits of play also continue into adulthood, in any areas where the subject is growing and learning new skills.

Social play, loosely defined as unstructured recreational time, helps improve social coordination and development of `social scripts'. Social play includes anything where there are no formal rules and where the social experience is unconsciously negotiated by the parties involved. Examples include a conversation or a drawing circle (but not TV or most gaming sessions). This includes so-called `parallel play', where two or more people will engage in separate activities without much formal interaction.

The most relevant form of social play amongst furries is probably `pretend play', which covers the sort of roleplaying that forms the foundation of much furry interaction. Online or in person, playing the role of furry characters and exploring different ideas helps develop self-identity and empathy. This translates into self-confidence.

There are a few examples where furry `play' may have a direct positive effect on self-confidence:

\begin{itemize}
  \item A furry meeting others for the first time may find themselves less anxious if this takes place in a safe, furry-only environment. Success in such an environment may grow the confidence of our shy furry, and embolden him to socialize elsewhere.
  \item A furry who feels outcast may socialize online, where she can find kindred spirits. Being accepted will help her understand others, improving her empathetic skills.
  \item A furry lacking sexual confidence might roleplay a sexual situation over text. This might lead to a low-stress real-life meeting, where intimacy already exists and the sexual mechanics are pre-negotiated.
\end{itemize}

The standards set by the furry community are important to help people discover a realistic target for themselves. Be it professionally, personally, or sexually, a person exposed to a healthy and happy community will tend to be drawn along a positive path. (Conversely, someone exposed to a negative environment will find it very difficult to rise above the norms of their peers.)

Happily, the furry community is a broad school. Someone new to the community is likely to find a positive and realistic rolemodel. We have many ad hoc variations of the mentor/mentee, or big-brother/little-brother relationship. Furries who are struggling with self-acceptance can expect to treated with care and respect.

Furries are unconsciously appropriating fun and using it for personal growth. The furry experience is externally enjoyable and internally rewarding. It's making us happier critters.
