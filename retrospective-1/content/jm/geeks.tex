\articlehead{Geeks}{JM}{2012}

The regular Londonfurs (londonfurs.org.uk) meets are a great environment for getting to know furry friends, old and new. The meets are held in a City bar on Saturday afternoons and are attended by upwards of 100 furries. They have an easygoing vibe, fuelled by the sort of bonhomie that's engendered by drinking with friends in the afternoon.

I was chatting at the bar with a couple of furry friends at a recent meet when we were approached by a geeky furry acquaintance of mine. Most furries will be able to guess what happened next. Our geek delivered a deadpan anecdote, describing a workmate who had become confused about two different types of barcode. His story -- which was incomprehensible to anyone not intimately familiar with the ins and outs of barcodes -- had nothing to do with the topic at hand.

Everyone in the conversation immediately understood that we had been ‘geeked'. We tried to steer our geek away from his topic and predictably failed: our geek paid no heed to the usual social cues of conversation. Everyone else managed to escape and I was left with my geek, doomed to listen on in feigned interest and rising annoyance.

Every socially active furry will be able to identify with my experience. Why, I asked another furry following my eventual escape, am I socializing with these infuriating geeks?

Another question struck me later in the day: why are there so many geeks in the furry community?

There are definitely a lot of geeks. I asked Klisoura, [adjective][species] contributor and curator of the Furry Survey. We found three questions that, if hardly authoritative, help us guess at the proportion of geeks in the furry community:

\begin{quotation}
  Would you describe yourself as:
  \begin{itemize}
    \item a fan of RPGs? (Yes 55\%)
    \item a fan of science fiction (Yes 61\%)
    \item a fan of anime (Yes 49\%)
  \end{itemize}
\end{quotation}

This data suggests that around half of the furry community might be considered geeky, although that's really just informed speculation. Suffice to say that there are plenty of geeks out there. (Further data mining suggests that geeks and non-geeks have different experiences within the furry community, but that's fodder for a future post. It's interesting stuff though.)

The furry community has always had strong connections with geek culture. Geek culture informs a large number of furries from a political, social and personal point of view. And, most pertinently, the fledgling furry community of the 1980s and 1990s was essentially a geek phenomenon. Most furries, especially pre-internet, discovered furry through a variety of geek fandoms.

The furry community of 2012 is not an exclusively geek phenomenon. Conversations about (say) programming languages may be common amongst furries, but these are not furry conversations per se. Such conversations occur because a lot of furries care about programming languages. I might chat about music with furry friends, but that -- like that lively exchange of ideas on programming languages -- is just two people discussing a common interest.

A full disclosure: I am not a geek, at least by furry standards. I have a science-based qualification but I don't work in IT. I don't frequent geek culture websites (like xkcd). I don't read speculative fiction and I don't watch animated TV programmes that are designed for children. I do, on the other hand, enjoy many non-geeky activities such as playing and watching sport.

I'm not anti-geek. I think geeks are great. My partner of some six years is a geek. Geeks can be frustrating, but they are also rather amazing.

My experience with the barcode geek is a common one, and a hazard for anyone socializing with furries. But geeks aren't all about derailed low-empathy conversation topics, there are big upsides.

I think the thing that most amazes me is the ability of geeks to intensely focus on some logical or mechanical problem. That single-minded intensity, which geeks often glibly refer to as ``the zone'' (without realizing the rare genius that seems to result from it), astounds those of us who don't work that way. Geeks are also direct and honest, have a knack for seeing unexpected solutions to complicated problems, and are rather charming to boot.

As it turns out, a combination of personal introspection, fierce intellectual pride, and charm is downright sexy.

Geeks are not always the most self-aware people. The unfortunate downside is, like my barcode-loving friend, they don't always meet the nebulous and ever-shifting rules and expectations of society. Examples, all of which will be in bold display at every furry gathering all over the world:

\begin{itemize}
  \item Geeks often have poor personal hygiene.
  \item Geeks often fail to read social cues that suggest they're acting inappropriately.
  \item Geeks often have poor interpersonal skills.
  \item Geeks often dress very poorly.
\end{itemize}

The good news is that geeks are open-minded. Geeks can, and do, get better at their shortcomings because they are open to change. If the stereotypical teenage geek is a smelly escapee from their parents' basement, then the stereotypical middle-aged geek is beardy, wise and all smiles.

The furry community is good for geeks who don't feel comfortable in a social environment, because it is welcoming and tolerant. Furry accepts all comers, regardless of social skill, but also provides a template for improved behaviour. Any new furry will meet a wide range of people and learn more about society's confounding rules through observation and experience. And everyone is improved by chatting with the wise beardy man.

The Londonfurs meets are good example of this. The geeks are welcomed and able to socialize amongst their peers, but they are also required to abide by society's rules in a large public space.

Regardless of furry's geeky genesis, it's logical that the community would attract a preponderance of geeks. Geeks are less likely to find an appealing mainstream societal norm, so they are more likely to be looking for somewhere to belong. Geeks also tend to be introspective and imaginative, ideally suited to our ultra-personal version of unreality.

It's difficult to be respected and considered sexy if you're a geek. Furry is a community that accepts geekiness, as well as providing a framework for geeks to have a positive self-image. Furry's acceptance of alternate anthropomorphic identities allows geeks to be accepted in that wonderfully counterintuitive furry way, where the most real version of someone is their imaginary avatar. Geeks can be sexy, confident, respected, and human.
