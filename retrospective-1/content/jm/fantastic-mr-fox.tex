\articlehead{Fantastic Mr Fox}{JM}{2012}

\textit{Fantastic Mr Fox}, the brilliant 2009 Wes Anderson film based on Roald Dahl's children's novel, is driven by two coming-of-age stories.

The first, and more traditional, follows Mr Fox's 12 year old son Ash. Ash is short, awkward, and prone to theatrical sartorial choices that reinforce his status as an outsider. He is forever comparing himself to his fantastic father and implausibly gifted cousin, Kristofferson. Over the course of the film, Ash learns to make the most of his strengths.

The second coming-of-age story is that of Mr Fox himself. Despite being a husband, father, home owner, and provider, Mr Fox sees himself as a ‘wild animal', a kind of perpetual teenager who continually needs to prove himself to the world.

In the opening scene, Mr Fox and his wife are caught in a fox trap while raiding a squab farm. Mrs Fox reveals that she is pregnant, and Mr Fox agrees to settle into a safer life for the sake of his family. This brings about an internal conflict in Mr Fox. He retains his self-image (a wild animal), which is at odds with the safe domestic life he makes as a father and newspaper columnist.

\textit{Fantastic Mr Fox} is a furry movie in that it features anthropomorphic characters. I'd also argue that Mr Fox's internal conflict has parallels with the furry experience. His internal conflict is similar to the disconnect of identity experienced by many furries: we present one version of ourselves to the real world but have an internal life where our furry identity looms large.

I don't want to overextend my linguistic gymnastics by stretching for too many parallels between the identity crises in Mr Fox and in furries. However, both we and Mr Fox must find some way to manage our split personalities.

Mr Fox does a poor job of this in the start of the film. He is prone to self-aggrandizement and risky behaviour, as if he is trying to prove his wildness despite his domesticity. He treats his friend, Badger, poorly – physically threatening him after being advised against a risky purchase, and cutting him off mid-speech. Mr Fox does so because he feels he must prove himself as the wild, fantastic animal he imagines himself to be.

Mr Fox's crisis is resolved in the best – and most flawed – scene in the film: the wolf scene.

Just after the climactic action sequence, Mr Fox spies a wolf in the distance. The wolf is a wild animal: quadruped, mute, strong. Mr Fox, despite his self-professed lupophobia, tries to engage the wolf in conversation. The wolf remains silent. The scene ends with the two making a non-verbal connection, acknowledging each other with a raised paw. As the wolf leaves, Mr Fox says to his son and nephew ``What a beautiful creature. Wish him luck, boys.''

It's a powerful and understated scene. The connection between Mr Fox and the wolf indicates the reconciliation of Fox's splintered identities. The gesture of acceptance shows the domesticated Mr Fox making peace with his atavistic self. With this acceptance, Mr Fox can find balance between his wild, internal world and domestic, external world. His newspaper column becomes edgier (``Fox on the Prowl'') and his next raid is on a safer target – a supermarket.

The gesture between Mr Fox and the wolf is a moment of personal triumph. It's something we can all strive for.

Unfortunately, the wolf scene is arguably a racist one. The black wolf stands in counterpoint to the civil world of Mr Fox et. al., and is a representation of the wild.

The black wolf is intended to be a metaphor for Mr Fox's internal atavistic shadow. However there is a history of blackness in cinema, where it is shorthand for mysteriousness and untamed animalism. This is a fundamentally racist association as it degrades blacks as being more like animals (and so less human). The black wolf is pure animal.

And, unfortunately, the key gesture between Mr Fox and the wolf looks a lot like a black power salute.

There is a long history of film using black characters in a racist fashion, even in otherwise excellent films. Consider Morgan Freeman's benevolent servant in \textit{Driving Miss Daisy}, or Michael Clarke Duncan's ``magical Negro'' in \textit{The Green Mile}. Such black characters only exist to act benevolently towards the white main characters, and have little other apparent motivation. Freeman and Duncan, in these films, are playing the stock character of the noble savage. Neither film is intended to be racist, however the characterization of the black characters in anachronistic.

I don't think that Wes Anderson intended the wolf scene to have any racial connotations. Anderson has form: 2007's \textit{The Darjeeling Limited} is about the three Whitman brothers (literally, the White Men) who get lost in the Rajasthani desert. The Indians in the film are broadly characterized, but this is a deliberate device to reflect the privilege of the Whitmans and their unfamiliarity with the world outside their bubble. \textit{The Darjeeling Limited} is a direct exploration of ‘whiteness', arguably a theme carried throughout many of Anderson's films.

Mr Fox is equally privileged and suffers the condition of being white. He is nattily dressed, speaks in a quasi-formal manner that suggests a traditional British-style education, is fluent in French, and is comfortable with Latin. While none of these things necessarily qualify him as white (he is, after all, a delightful shade of orange), it's a reasonable assumption to make in the context of Anderson's other work.

Like the black characters in \textit{The Green Mile} and \textit{Driving Miss Daisy}, I suspect that the black wolf's cameo will become anachronistic over time. It'll remain a small criticism of an otherwise excellent film, at least until the world improves to a point where skin colour doesn't have associational baggage.

I, for one, would be happier if Anderson had taken a page out of the furry book and made his wolf blue. (Neon green bangs optional.)
