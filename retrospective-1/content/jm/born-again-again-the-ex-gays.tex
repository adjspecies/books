\articlehead{Born Again, Again: The Ex-Gays}{JM}{2012}

Last Monday I posted an article comparing born-again Christians to born-again furries, those of us who found their life's ear-and-tail-filled path at a more mature age. The discovery of something so important and personal often leads to born-again furries (and Christians) to be evangelical about their revelatory experience.

I used this to introduce a rough truism -- \textit{a closely-held extreme belief often belies a transformative personal experience.}

We have all met furries who hold extreme opinions on various topics. These opinions are regularly infuriating -- however there is often a personal story behind the opinion. And that story will help cast the extreme opinion in a new, more understanding, perspective.

This is a theme you've heard from me before and will likely hear in the future: that it's important to be tolerant.

I want to continue the theme by talking about the so-called ex-gay movement. It is a great example that has an equivalent resonance within the furry community.

The ex-gay movement is largely Christian-driven, based on the idea that homosexuality is a learned behaviour that can be treated. There are a couple of big problems here:

\begin{itemize}
  \item Modern psychology has proven beyond any doubt that sexual preference is innate and essentially immutable.
  \item The movement reinforces the destructive idea that there is something wrong with homosexuality. Worse, many teenagers are cajoled or forced into a programme. Predictably, this can reinforce internalized homophobia and lead to mental health problems. The ex-gay movement is very probably indirectly responsible for many suicides.
\end{itemize}

Moving on from the moral failings of the ex-gay movement, the experience of people who attend and claim conversion is worth examining. Just like born-again Christians and born-again furries, ex-gays can be evangelical: they believe that they've had a life-changing experience.

For some of the ex-gays, this will be true. These are the ones who were never fundamentally homosexual: they were heterosexuals who were experimenting, or possibly hetero-leaning bisexuals.

Many furries have had a similar experience. There is no stigma on gay sex in furry, and most furries are not exclusively heterosexual. In this environment, many fundamentally heterosexual furries will have had gay sex. (The opposite of the non-furry world, where many fundamentally homosexual people will have had straight sex.)

Some straight furries will realise their sexuality later in life, perhaps after falling in love with a member of the opposite sex. Like the ex-gays, this can be a revelatory experience. And some of those furries will become evangelical about their experience, resenting the furry community's complicity in stymieing this revelation.

Consider an ex-gay-furry -- let's call him StraightFox -- who decries the community for pushing homosexuality on new members. StraightFox is convinced that vulnerable young people in the furry community are being placed at risk by established gay members. (I expect that most people reading this post have been exposed to someone like StraightFox at some stage.) StraightFox is probably going to get shouted down for trolling. This is wrong: StraightFox is misguided, but so is the furry -- let's call him GayWolf -- calling for his head.

Here's why StraightFox is wrong:

\begin{itemize}
  \item Nobody in the furry community turned him gay. He was straight at the beginning; he is still straight now.
  \item He is assuming that everyone else has the same experience. StraightFox may have had a negative experience but he is neglecting those that have had positive experiences. Furry's enthusiastic acceptance of sex is manna to the repressed, the closeted, and the shy.
\end{itemize}

Here's why GayWolf is wrong:

\begin{itemize}
  \item StraightFox has been damaged by the furry community. He deserves respect (for finding his true self and for being brave enough to voice a contrarian opinion) and pity (for his difficult experience).
  \item GayWolf, being gay (and a wolf), should think of his life in non-furry world, where he is a member of a minority. GayWolf should know that it hurts to have a personal and important belief shouted down.
\end{itemize}

Like StraightFox, those people who have ``successfully'' been treated through the ex-gay movement are worth talking to. They will have fascinating stories to tell. They might be in complete denial; they might be struggling to fit into a world that doesn't accept their sexuality; they might have an interesting perspective on sexual experimentation. And just like everyone else, they are probably a good person struggling to manage their human failings.

Ted Haggard is a high profile, oft ridiculed ex-gay. Haggard was a high-profile evangelical pastor who was caught buying methamphetamines and sexual services from a male escort. In the ensuing scandal, Haggard resigned or was fired from his various religious posts.

And we all enjoyed the sweet, sweet taste of schadenfreude: someone who supported anti-gay legislation was publicly shown to be a hypocrite.

Following the scandal, Haggard underwent ex-gay counselling and was declared ``completely heterosexual'' just three weeks later. This claim was met by general disbelief, derision, and laughter.

It is easy to conclude that Haggard is a deluded and/or calculating individual: that he is trying to fool either himself, his family or the general public into believing that he is straight. It's easy to disregard him as a caricature: a fake, greedy, self-promoting hypocrite masquerading as a community leader.

Yet it is wrong to cast Ted Haggard in such a simplistic way. In 2011, a journalist, Kevin Roose, went on a camping trip with Haggard and his two sons. Late night, over the campfire, Haggard put a different spin on his situation:

\begin{quotation}
  ``I think that probably, if I were 21 in this society, I would identify myself as a bisexual.'' After a weekend of Ted trying to convince me of his unambiguous devotion to his wife and kids, I'm at first too surprised to say anything.

  ``So why not now?'' I ask finally.

  ``Because, Kevin, I'm 54, with children, with a belief system, and I can have enforced boundaries in my life. Just like you're a heterosexual but you don't have sex with every woman that you're attracted to, so I can be who I am and exclusively have sex with my wife and be perfectly satisfied.''
\end{quotation}

This is not to say that Haggard should not be criticised. But it's easier to see him as a flawed human being who deserves pity for his situation. Like the ex-gay furries, he's in a situation he didn't choose and he didn't foresee.

The story of the ex-gays is different from last week's born-again furries but the general conclusion is the same: people are interesting and vulnerable, but this is rarely evident on first impressions. This is why tolerance is a great virtue: it gives people a chance to show themselves in all their complexity.

So be friendly and respectful towards the people you disagree with. You might be surprised.
