\articlehead{Horses and Houyhnhnms}{JM}{2012}

Many of you will be vaguely familiar with Gulliver's Travels, the satirical novel written by Jonathan Swift and published in 1725. However you may not know that the book is overtly furry.

Gulliver is a traveller who, through misadventure, voyages to four unknown lands: Lilliput (a land of little people); Brobdingnag (big people); Laputa (a scientific ruling class repressing an uneducated populous); and finally Houyhnhnmland -- land of the rational horses.

Pronunciation note: `houyhnhnm' is the name the horses have given themselves and so should sound much like a horse's whinny -- `hwinnum'.

I won't go into the plot in detail (although I will discuss Houyhnhnmland a little later on) but suffice to say that it's a very easy and entertaining read. The language isn't as antiquated as you might think; no more so than the contrivances used by some fantasy authors.

And then there's the furry content.

For starters there are the rational horses, the houyhnhnms themselves. They talk, they use their forelegs to handle tools and to eat, and they live in a society. In short, they are anthropomorphs.

Swift uses his anthropomorphic horses much in the same way many furry books, and sometimes furries themselves, use anthropomorphs: to reflect on human society. The houyhnhnms are entirely rational and live in a peaceful collective where the concept of lying is unknown. To draw a parallel with a more modern invention, they share similarities with Star Trek's Vulcans and in many ways, Borg.

Swift goes a step further by including zoomorphic humans: the yahoos. The yahoos are humans stripped of their rational nature. The resulting animal is reduced to a violent, selfish, scatological and sex-driven being. Gulliver is so disgusted by the yahoos that he begins to hate himself as he sees his instincts reflected in their behaviour. He yearns to be less human; more horse.

The entire book, not just the Houyhnhnmland voyage, looks at human society from an outsider's point of view. This, in my opinion, is how many furries see the `human' world: as a collection of laws and unsaid rules that are often illogical and arbitrary. In each of the four islands visited by Gulliver, he experiences an askew version of England and English society.

Most famously, on the first island, Gulliver is a giant amongst two nations of tiny people who are at war. They are, literally, at war because they disagree over which end of a boiled egg should be sliced off before eating: the big-endians and the little-endians. On his second voyage, Gulliver finds himself the diminutive amongst giants. He attempts to justify the slaughter of his fellow tiny men in the war between England and France by the insignificant perceived differences between the two nations. His explanation is met with the same disbelief and horror that Gulliver expressed over the endian war.

England is no longer at war with France however the metaphor is just as strong today. I think many furries consider themselves to be outsiders from human society, and see many of society's actions as equally illogical and harmful as the endian war. I don't think you, reader, will struggle to find a relevant modern-day analogue.

Back in Houyhnhnmland, Gulliver's Travels explores the conflict between our instinctual, atavistic side and our rationality. By creating beings that are purely rational (the houyhnhnms) and purely animalistic (the yahoos), Swift asks the reader to consider himself. We like to think that we're rational beings, but how true is that? Surely most of our decision-making is driven by instincts like fear, or sexual desire, or love?

Furries explore the same questions pretty directly. By presenting as non-human (or part-human) animals, we're disassociating ourselves from the rules of `normal' human behaviour. Starting from a position a half-step away from humanity, and a half-step towards our furry avatar of choice, we think about our animal instincts and consider that perhaps some of the artifices of human behaviour are untenable. The traits that we've appropriated from our avatars are usually instinctual ones; instincts that bring us closer to the animal world, and closer to one other. We've learned that a hug is often preferable to a handshake.

Through this lens, furries, like Gulliver, can see how humans everywhere are guided by instinct. (Many, if not most, people would deny this.) Once you think of everyone as an animal, it's easy to see selfish or territorial or lustful behaviour. And it's easy to see that denying that these behaviours are instinctual, and so applying a sheen of redemptive `reasoning', often leads to harmful outcomes.

The houyhnhnms have no such instincts and accordingly their lives are guided by purely rational principles. They know neither love nor empathy. Decisions are made collectively and never second-guessed. Mating pairs are selected based on genetic synergies. They enslave and freely execute yahoos, rationalising that such wretched creatures cannot have worthwhile quality of life. They eventually exile Gulliver after observing his human flaws.

The furries might say that embracing instincts for what they are -- natural -- leads to a new understanding of ones self, and leads to the possibility of a richer life. If you are naturally flawed, it's easier to accept that everyone else is too. The furry community, for all its problems and drama (brilliantly encapsulated in these virtual pages by Makyo), is a welcoming and tolerant one. Swift's houyhnhnms and yahoos, representing the two extremes of our human animal nature, live in two very different but equal hells.

Gulliver's Travels is out of copyright, and so is available to download for free from Project Gutenberg (www.gutenberg.org). You can get it in any format imaginable: plain text, Kindle, HTML, even as an audiobook.

No copyright means that there are no royalties payable, and so the story of Gulliver's Travels has been adapted countless times. The `fantastical voyage' aspect of the story makes it ripe for adaptation into children's stories, much in the manner of One Thousand and One Nights. The book has also been exploited with varying degrees of adherence to the source material in middlebrow cinema, notably the 1990's TV miniseries featuring Ted Danson and the recent film starring Jack Black.

Adaptations of Gulliver's Travels usually focus on the first two of Gulliver's voyages, where he is respectively huge and tiny amongst the native population. The adaptations usually water down the occasionally explicit sexual content of the novel, which is a key theme. Human sexuality is a major societal motivation and Swift does not withdraw from the topic: Gulliver's comparatively massive genitals are key to his activities in Lilliput; he becomes a sexual plaything for teenage girls in Brobdingnag; crude sexual advances from pubescent yahoo girls lead to his eventual abandonment by the houyhnhnms.

Gulliver's Travels and Swift's houyhnhnms helped me understand my own identity as a furry. It's given me an insight into myself and also provided the language and framework to allow these ideas to become fully formed. Swift's focus on the true motivations of the human animal -- instincts such as sex and fear -- helped me understand that my own motivations are just the same as everyone else.

I sign off my emails as `your friendly local houyhnhnm', but this is not to say that I see myself as a rational being. It's quite the opposite. The houyhnhnms, for me, are a reminder that I am just an animal. I have instincts that I can't deny or rationalize away.

Gulliver, on the other hand, is seduced by the logic and the reason of the houyhnhnms. Despite being cast out by the houyhnhnms and being returned to England by sympathetic humans, Gulliver rejects human society outright, seeing only a group of yahoos. He ends the novel as an embittered misanthrope. Sadly this is also the fate of many furries. Like Gulliver, they are blinded to the greatness of society by contemptible human actions, false rationalisations and the other ways in which we humans fail.

Swift encourages the reader to empathize with Gulliver and this is part of the book's genius and power. We watch and understand his downfall but we ultimately reject Gulliver and his beloved houyhnhnms. We can choose a happier path.

My furry self, the horse, stands for two things: my animal nature that I need to learn to embrace and accept, and the fact that I can use reason and rationality to improve myself and my life. The houyhnhnms and the yahoos are metaphors for our dual nature as human animals. They're something I'm running towards and away from at the same time.

There is irony here. By stepping away from my human form into my furry one, I've learned how to be human. The most human version of myself is the horse.
