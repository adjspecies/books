\articlehead{Furries and HIV}{JM}{2012}

HIV is scary.

This article is going to scare some people. Some will find it difficult to read because it makes them feel queasy. Some people won't click the ``Continue reading'' button due to fear.

Fear is a natural response. All people are naturally risk averse. We prefer to pretend that scary things like HIV don't exist. We prefer to be ignorant. We find the following topics uncomfortable:

\begin{itemize}
  \item Furries who are HIV positive.
  \item The incidence of HIV infection.
  \item The symptoms you'll experience if you contract HIV.
\end{itemize}

As Michael Harris, a gay Canadian journalist and author, wrote in 2011:

\begin{quote}
  I live in fea r — because HIV is a cipher for everything that constrains my sexuality and my potential for happiness.
\end{quote}

The furry community is often perceived to be at risk of a HIV outbreak because we are closely sexually interconnected, far more so than a normal social group. There are three main reasons for our close interconnectedness:

\begin{itemize}
  \item Furries don't strongly separate along gender or sexual preference lines. So a completely heterosexual furry male (or homosexual female) is likely to have a ‘short route' for infection via a homosexual male.
  \item The furry community is relatively sexually active. Consider that the average American, aged 15 to 44, has had just five sexual partners (Ref).
  \item The furry community is fairly incestuous. This is because most furries socialize within the community where sexual availability is high, and furries often prefer to have sex (and relationships) with one another.
\end{itemize}

The best test of furry sexual interconnectedness that I am aware of is the Yiffchart, a social experiment conducted by Tursiae, an Australian furry. In 2007, Tursiae invited Australian furries to list their furry sexual partners, with collated results presented anonymously in a chart. The charts shows furries closely clustered together, with few ‘degrees of separation' between any two datapoints. The results are particularly striking given the large distances between cities in Australia and the voluntary (and therefore incomplete) data collection method.

You can see the Yiffchart here (http://ofyc.thelair.org/).

There are some HIV positive furries, too. I had a chat with Yama Roo, who is HIV positive and was happy to be quoted publicly here on [a][s] (thanks Yama). Yama contracted HIV through unprotected sex with a stranger outside of the fandom. As he puts it:

\begin{quote}
  It was after a bad breakup, so I just went out looking for fun.
\end{quote}

Yama has found that his HIV-positive status changed his relationships with other furries:

\begin{quote}
  They usually run away when they hear the letters.
\end{quote}

Yama's experience probably explains why the furry world has never experienced a significant HIV outbreak: our interconnectedness means that at-risk furries are likely to be identified by the community. Yama's responsible attitude towards disclosure (\textit{``Anyone who knows me knows I'm HIV positive.'' ``I don't even let [furries] flirt with me online without knowing about it.''}) is probably matched by the capacity of furries to gossip.

While furry's tight-knit community has provided some protection against HIV transmission to date, furries are placing themselves at risk by failing to practise safe sex.

I talked with Biramaye (@biramaye), an Australian furry, equality campaigner, and paid porn actor. He has noted the reluctance of many furries to use condoms, and thinks that furries should follow the gay community's example and embrace condom usage. As he puts it, there should be ``recognition of social responsibility amongst the more promiscuous''.

Biramaye's point, that sexually active furries have a responsibility to use condoms, is simple and compelling. The need for change is illustrated by this story from Yama (before he was infected), an example of today's furry world:

\begin{quote}
  \ldots we had a group of [furry] friends we played with. It was only those guys, and we all played bare because of it. One of them had a scare, and we all stopped playing for a while. He went outside of the group and it angered all of us. Still, instead of learning our lesson, we just stopped playing with that person (once we all tested negative for things) and started playing bare with each other again. It was pretty stupid.
\end{quote}

Such attitudes are common within furry. Yama's candour is rare (and greatly appreciated), but his sexual experiences are not. As he says:

\begin{quote}
  [Furries] think that as long as they play inside the fandom, they're safe. Hearing HIV makes them realise the reality and is a buzz kill, so they ignore it.
\end{quote}

A large proportion of gay men are HIV positive. In major cities in North America (Ref), Western Europe, Australia \& New Zealand (Ref), one in five men who have ever had gay sex are HIV positive. Worse, over 40\% of these men do not know they are HIV positive (Ref).

This means that, assuming you live in a large western city:

\begin{itemize}
  \item If you have had gay sex with four men, the chance you have been exposed to HIV is greater than 50\%.
  \item It may statistically be safer to have sex with an HIV-positive man (with a condom) compared with a man who thinks he is negative (also with a condom). Men who are HIV-positive are likely to have sought medical treatment, and therefore have a much lower viral load than those who falsely believe they are negative.
\end{itemize}

Of course, exposure to HIV – which I define as sexual contact with an HIV-positive person – does not guarantee transmission. The transmission rate from unprotected anal sex is around 1\% to 10\% depending on viral load, dropping to close to zero if you use a condom.

If you have contracted HIV there are common physical symptoms which will usually occur together, two to four weeks later:

\begin{itemize}
  \item A very high fever.
  \item A very sore throat.
  \item A whole body maculopapular rash, which is like heat rash or measles.
\end{itemize}

There are a few other symptoms which are common but not universal as the first three:

\begin{itemize}
  \item aches and pains
  \item headache
  \item mouth ulcers and sores
  \item abdominal pain
  \item vomiting and diarrhoea
\end{itemize}

HIV is, of course, very easy to avoid: always use a condom.

Condom usage among gay men has become commonplace in the wake of the 1980s HIV/AIDS epidemic. Today, condoms are much more likely to be used in homosexual intercourse than heterosexual intercourse, to the point that only around half of new HIV infections result from male-to-male sexual contact (Ref). This is despite the much greater risk of contracting HIV through gay sex, and that's not just because a homosexual sex partner is (much) more likely to be HIV-positive:

\begin{itemize}
  \item HIV transmission rates are higher for anal sex than for vaginal sex.
  \item Transmission rates are much higher when the penetrative partner is HIV-positive. This means that HIV transmission from women to men is rare.
\end{itemize}

Condoms as a normal part of gay sexual contact is reinforced by many facets of gay culture. This includes fantasy representations of gay sex – the porn.

Condom usage is depicted in 80\% of gay porn scenes, compared with 3\% of straight scenes. Gay porn scenes without condom usage are typically marked as being unusual by the term ‘bareback', and some gay male audiences regard such scenes as ‘viewing death on the screen' (Ref).

Such depictions help make condoms an expected element of gay sexual contact. There is an argument that increased visibility of condoms in furry art would similarly normalize their usage among furries. As Biramaye puts it: ``most [furry] art is bareback fantasy''.

Yama's experiences also suggest that depictions of furry sexual contact should include condoms more often:

\begin{quotation}
  As for the furry fandom, it seems for us majority of the fandom works off of innocence. We came into the fandom young, we have cute characters, and cute words for things. We don't fuck, we yiff. We snuggle, we cuddle, we're very open about the innocence and the cuteness of it. I think a lot of the fandom chooses not to believe in HIV because it destroys that feeling. We all go off to a fantasy land. Furries (in art and stories mostly) don't get HIV. They fuck who they want, and they have fun, and they don't have to worry about consequences.

  Most of the fandom is done on the internet, and in the mind. It's an escape, and it has a world of its own. The unfortunate side effect of that is thinking that a mental escape also equals a physical one. We can choose to ignore the social rules of the regular world all we want, that's the mental aspect, but you can't physically remove your body from the rules that govern it. I think that's where things go wrong, and I think that's why most furs ``choose'' to ignore that STDs even exist.
\end{quotation}

Furry art, of course, is often set in a world where condom usage would be anachronistic (such as in a medieval setting) or otherwise out of place. However much furry art is ‘real world' enough to include condoms. Based on a keyword search of e621.net, I estimate that condoms are depicted in less than 1\% of penetrative furry porn although, happily, such depictions show condoms in a positive context: either as a fetish item or as a positive part of the seduction process.

Condoms sometimes appear in the written furry world too: [adjective][species]'s own Kyell Gold uses them at times in his novels, including in the upcoming Out Of Position 3.

Yama goes further:

\begin{quote}
  I think [increased depiction of condoms in furry art] would only be a small start. One of the main ways furries interact and see each other is at conventions. I think cons should, for one, hand out condoms. The second thing they should do at conventions is have a safe sex panel. Possibly even have someone who is dealing with HIV who is willing to talk at the panel. I think in person, real life awareness, is the first step to getting the word out. The Internet can be ignored, but if someone who you're talking to has it, and is telling you their stories, it's harder to ignore. You can't just turn them off.
\end{quote}

I agree. I started researching this article sceptical of the value of condom depiction in furry art, but become otherwise convinced as I learned more and chatted with other furries – Yama and Biramaye in particular. Free condoms are handed out at many sci-fi conventions, a practice that should be emulated at furry conventions. The logic is compelling: normalization of condom usage will reduce HIV transmission rates within the furry community.

Hopefully this article is a small step in the right direction. I would like to encourage you to share this article among your furry friends and social groups: forums, Twitter, FA, whatever. I'm also curious to hear your thoughts and reactions – you can comment below or contact me directly at jm@furrynet.com (email/MSN).
