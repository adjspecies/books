\articlehead{It's Raining Men}{JM}{2012}

It’s common for furries to look within the community for potential long-term partners. For many people who are serious about furry, like me and presumably many of the readers of this article, a potential partner is required to be furry.

It’s logical that furries will form relationships together, because furry is about identity. If your identity as a virtual animal-person is internally important, you’re going to want to share that with your partner and express that within your relationship. I can’t think of a better example than [adjective][species]’s own Makyo, who was married last week and posted a thoroughly charming picture of him and his partner in suit.

Furry is a very social group and it’s easy to meet new people, so there are a lot of opportunities for relationships. That is, unless you are heterosexual and male.

If you’re a heterosexual male and you want to find a partner within the furry community, your odds aren’t good. The ratio of eligible men to eligible women is about 3:1. And that’s being optimistic.

Here’s how I arrived at this ratio. Anyone not interested in the maths may wish to look away now.

Firstly, some assumptions and simplifications:
\begin{itemize}
  \item All my data comes from the 2011 furry survey.
  \item I’ve lumped the pansexuals in with the bisexuals for convenience.
  \item I’ve excluded asexuals.
\end{itemize}
And the base statistics:
\begin{itemize}
  \item Men make up about 80\% of the furry population; women 20\% -- this split is pretty consistent if you look at biological sex or self-reported gender.
  \item The proportion of straight:bi:gay men in the furry community is 37:34:25\%.
  \item For women, it’s 46:41:8\%
\end{itemize}
And the maths:
\begin{itemize}
  \item I exclude gay men and gay women.
  \item I assume that bisexual men and women can end up with opposite-sex or same-sex partners based on availability. (So bisexual men end up with mostly other men, simply because there are more gay/bi men available than straight/bi women.)
  \item Based on this calculation, I exclude the proportion of bisexuals who end up with same-sex partners.
  \item This leaves us with just those men (straight or bisexual) who are competing for available women (straight or bisexual).
\end{itemize}

The results:

\begin{itemize}
  \item 46\% of furries are men available for a female partner.
  \item 16\% of furries are women available for a male partner.
\end{itemize}

In all likelihood, this is optimistic for men seeking heterosexual relationships within furry. Women are a small minority, but they also tend to identify less strongly as furries (according to the furry survey, although this isn’t reported anywhere public). So I’m guessing that this means that furry women are less likely to look inside the community for a partner, which will further deplete the available women.

A further problem is that the furry community is not very welcoming to the small number of women that do socialize. I am aware of several occasions where women have had trouble with unwelcome attention from guys within furry. This annoyance has crossed the line into sexual harassment and sexual abuse all too regularly. Of the furry women I know, a very high proportion have suffered. I have no doubt that this is a contributing factor to the small number of furry women, and their lack of engagement with the community.

There is also a sizeable minority of gay male furries who exhibit a less aggressive antipathy towards women. Their attitude might be described as Calvinesque, as in Hobbes. While there is often a lighthearted element to an ``ew girls gross'' attitude, it is still unwelcoming.

This problem is not restricted to the furry community. Inherent sexism is a problem in many male-dominated geek fandoms, an issue that is starting to be addressed in some circles.

Last year, a Texan gamer group decided to ban women from attending a LAN event. The bigotry was punctuated by irony: organizers decided the event should be male-only because they were worried that women attending the event would be subject to sexist comments.

Happily, the group was widely attacked for their sexism. It’s about time that sexism within furry was addressed as well.

A quick caveat: acceptance that the furry community is inherently sexist does necessarily not imply that furries are sexist. It is a norm -- there are patterns of behaviour within the group that make it unwelcoming for many furry women. As standards with the community change, furries will adjust and act accordingly.

The best, and easiest, step towards change is to start talking about how women are treated within the furry community. It’s important. More happy furry women will make the group better for everyone.
