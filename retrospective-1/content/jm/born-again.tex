\articlehead{Born Again}{JM}{2012}

George W. Bush is probably the world's most famous born again Christian. At age 40, he was a borderline-alcoholic, a failed businessman, and the son of a successful politician. He credits a conversation with the Reverend Billy Graham in the mid-1980s, a high-profile preacher and Bush family friend, with turning him around.

Whatever you think of Bush as a politician, and whether you believe his story about rediscovering religion (plenty of people feel it's a convenient fiction), it's a compelling narrative: ``ne'er-do-well boozehound finds God; becomes president''.

Bush's story is unique but the sentiment is common amongst born-again Christians. Born-again Christians like Bush credit their faith for showing them the path to becoming a fully realized person. It is a revelatory experience to discover, or rediscover, your direction in life. The strength of that experience is such that born-again Christians are notoriously evangelical about their faith.

Which brings me to the furry community.

Many furries don't discover the furry community until they are already an adult. The experience of learning about your personal identity, combined with the fellowship of the community, is often intense and profound. It's common for such furries to go through changes that might qualify as being born again: they reconsider their lives and find a new, more honest direction. A born-again furry immediately gains an important social group; commonly re-evaluates their sexual preference; and sometimes changes career, relationship status or living situation.

Finding furry later in life can be revelatory. Just like born-again Christians, the experience is so strong that born-again furries can be evangelical about the community. (Some of them go so far as to write articles for websites dedicated to meta-analysis of the community itself.)

To those people who discovered furry as they were maturing, for whom furry has always been a part of their adult lives, the evangelical attitude of born-again furries can be a bit ponderous or cloying. After all, the furry community is flawed: the drama, the toxic personalities, the popularity contests, and the cliques are all reminiscent of the worst aspects of tribal juvenile behaviour, otherwise known as high school. How could a self-destructive and childish community be so great and life-changing?

To which the born-again furry might answer: try being a latent furry who doesn't connect with their natural social groups. Or perhaps: if you think furry is childish and self-destructive, you should see what the rest of the world is up to.

The evil funhouse mirror version of the born-again furry is the anti-furry. An anti-furry (for the purposes of this article) is someone who thinks the furry community is a bad thing. In my experience anti-furries usually fit into one of two categories:

\begin{description}
  \item{The burned fur} a fur who has been hurt by or become disgusted with the furry community. They usually cite the endless whirlpools of drama, or the community's tolerance of unsocial or oversexed behaviour. These furries often ragequit by airing their grievances in some public forum, quietly lurk online for a while, and eventually return -- albeit harbouring ill will towards the community in general.
  \item{The troll} someone who rails about the horror of the furry community. Something Awful was a notorious hotspot for this sort of behaviour some years ago. I'm happy to say I have personally seen less trolls in recent times. The trolls who are obsessed with the furry community are often (surprise surprise) latent furries who haven't managed to admit it to themselves.
\end{description}

The anti-furries are similar to the born-again furries because they have equivalent confidence in their opposing convictions. They are also similar in that those strong opinions are rooted in personal experience: the revelation of the furry community at a late age; the feeling of being let down by the community; the pain of being a closet case.

Regardless of your own opinion, born-again furries and anti-furries alike have interesting and valid personal stories to tell. They are all worth your time and your fellowship.

There is a rough truism here -- a closely-held extreme belief often belies a transformative personal experience -- that can apply in any case where a strong opinion is expressed. It holds for born-again Christians and, to refer back to my recent post, it holds for controversial topics like zoophilia.

My article from two weeks ago makes a defence for ethical zoophiles, practising or otherwise. It's a topic that is rarely discussed in any sort of intelligent fashion. In my experience, most conversations devolve into flame wars between two people who hold extreme positions on either side. But there are strong reasons to defend and appreciate the people holding both the extreme pro-zoo and the extreme anti-zoo opinions.

A few weeks ago, there was a zoophilia thread over on Flayrah. The subject is a furry who has been prosecuted for bestiality. The subsequent comment thread is difficult to read without becoming enraged for the un-nuanced opinions asserted from all sides -- see below for a relatively mild example.

\begin{figure}
  \begin{center}
    \includegraphics[width=\textwidth]{content/assets/born-again--comments}
  \end{center}
  \caption{Comments from Flayrah}
\end{figure}

But let's consider a closely-held extreme belief often belies a transformative personal experience. This puts a different perspective on our flamers. I have already made my defence of the zoos, so let's look at the anti-zoos. Here are a couple of plausible scenarios:

\begin{itemize}
  \item The topic of zoophilia is important to our anti-zoo because she is a zoophile and is rejecting it in her own head. Participating in zoo-related discussions helps reinforce her belief that zoophilia is wrong. It also helps her believe that she will be able to escape her own sexuality. And perhaps, subconsciously, it's also a little titillating.
  \item The topic of zoophilia is important to our anti-zoo because he experimented at a young age with the family pet. What if our anti-zoo was discovered by a family member mid-act? (We all vividly recall our own shameful experiences and we know how powerful they can be.)
\end{itemize}

In either case, our angry furries are a lot more complex and interesting than their comments suggest.

Some of the most interesting furries I've met were angry. The topics that triggers their anger are all different, but they all share the same destructive consequences: anger is exhausting and, in the long run, unhealthy.

The furry community is a positive one for people exploring their own mind and their own personality. This holds true for the angry furries as well. Most of the angry furries I've met have changed or moderated their position over time -- the community has helped them accept their sexuality, or their past, or their familial relationships, or whatever it may have been. Everyone has a unique story.

It's easy to get fired up when someone disagrees with you. I'd suggest that we should be tolerant towards all parties -- even the trolls, even the born-again Christians. They might surprise you with their intelligence and ideas. (Your milage may vary if you befriend George W. Bush.)
