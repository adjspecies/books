\articlehead{Zoophilia in the Furry Community}{JM}{2012}

One in six furries self-identify as zoophiles. The real number is probably higher.

This piece of information comes courtesy of Klisoura's survey and I don't think it would surprise many furs. A quick mental poll of the furries I know -- the ones that I'm close enough to have an insight into their sexual preferences -- suggest that it's about right.

Like many things in the furry world, exactly what comprises a zoophile is a little blurry. It's arguable that furry porn, as appreciated by a large majority of the readers of this blog, might be considered zoophilic. Taking the non-furry world as our reference point, furry erotica is certainly a half-step in the zoophilic direction.

In the psychology world, there is a growing consensus that zoophilia is a legitimate sexual orientation. Research taking place this century is roughly equivalent to the human sexuality research famously performed by Kinsey in the middle of the twentieth century.

A sexual orientation is usually judged to be valid based on three criteria:

\begin{itemize}
  \item Affectional orientation (who we emotionally bond with)
  \item Sexual fantasy orientation (who we fantasize about)
  \item Erotic orientation (who we prefer to have sex with)
\end{itemize}

Using these three criteria, it's easy to qualify homosexuality as a legitimate orientation. (You would struggle to make an argument for plushophilia.) There is growing scientific evidence that zoophilia qualifies on all three counts.

There has been very little research into zoophilia. Up until very recently, scientific research focussed exclusively on mentally disadvantaged or low-IQ subjects. However research in the past few years has started to focus on so-called ‘high functioning subjects', which is a slightly weaselly euphemism for ‘normal people'.

Jesse Bering, a research psychologist and regular contributor to Scientific American, is probably the world's leading mainstream voice on zoophilia. Bering has explored the topic on several occasions in his Scientific American column and elsewhere. Among his data and discussion is the rather startling statistic that around 1\% of people probably qualify as zoophiles.

1\% is a lot. Consider that around 5\% of people are homosexual.

Bering, however, is a pragmatic scientist. He will argue that the facts support zoophilia being a legitimate sexual orientation, and that there are a lot of zoophiles out there. (And many more amongst us furries.) But Bering doesn't touch the other side of the argument: the moral argument. And it's a big one.

\textbf{Is it okay to be a practising zoophile?}

Peter Singer, the ethicist I mentioned in my article on vegetarianism a couple of weeks ago, bases many of his arguments on the simple premise that humans are animals and therefore not a special case. This is not to say that the life of a human being should be considered as valuable as, say, an ant: quite the opposite. Singer argues that the suffering of a human being should be given equal consideration as the suffering of any other species. So a species with little capacity for suffering, like an ant, gets proportionally little consideration.

Singer, in his tragically-titled 2001 article Heavy Petting, makes the point that interspecies attraction is completely natural. He mentions a few obvious examples including incidents of zoophilia in humans, but also sexual attraction towards humans by other species. He discusses a typical amorous housepet and also a case of a male orang-utan making overt sexual advances towards a female human.

The best documented case of a non-human anthropophilia is Lucy the chimpanzee. Lucy was observed to have no sexual attraction towards members of her own species, but would masturbate to pictures of naked (human) men displayed in Playgirl.

Dan Savage, the sex columnist and ethicist, responded to a zoophile correspondent in 2008 (link). The question posed was a simple one, and one probably on the mind of many zoophiles: I'm emotionally, mentally and sexually attracted to dogs. I'm not attracted to humans. What do I do?

Savage acknowledged the difficulty of the situation. Importantly, this included the tacit concession that his correspondent was a zoophile by orientation, and not by choice. The zoophile would not be ‘cured' by therapy (no more than a homosexual might be) and the zoophile would be well served to learn to accept, rather than fight, his orientation.

Savage suggested that his correspondent find a canine partner and keep his personal life to himself. This qualified endorsement was made on the condition that the zoophile keep himself safe (from prosecution or persecution) and his dog unharmed (from the sexual acts committed in the relationship).

This is the crux of the issue, I think: harm.

Both Singer and Savage make the obvious comparison between eating animals and having sexual contact with animals. They both conclude that bestiality is less harmful than eating meat.

I've touched on this topic before in other forums and I know it's a controversial statement. It's not easy to conclude that a societal norm like meat eating could possibly be worse than bestiality, a taboo sex act widely reviled for its perceived cruelty. However, if you can put aside those preconceptions, it's easy to see that the harm caused by a practising zoophile pales against that caused by someone eating (say) one factory-farmed chicken a week.

Even if you are vegan, I don't think you can hold a strong aversion to bestiality on ethical grounds. The harm caused by the myriad of meat-eaters is overwhelming compared to the relatively few practising zoophiles. This comparison holds even if you assume that the zoophile is harming the animal in question.

This is not to say that zoophiles don't have a responsibility towards the welfare of animals: of course they do. Most zoophiles are attracted to horses or dogs. The duty of care of a zoophile is exactly the same as that of an owner of one of these domestic creatures. From a harm point of view, the sexual component is not relevant.

The large majority of zoophiles will be ethical and responsible carers. Because of the emotional connection -- something required for the zoophilia sexual orientation to apply -- it's likely that zoophiles make excellent pet owners. There will always be a selfish, sex-driven cruel minority, however it's unfair to tar all zoophiles with that brush.

This allows me to wheel out one of my favourite phrases: the most visible members of a minority are rarely its best ambassadors. To put it another way: the majority of zoophiles are not doing harm and they are largely invisible. Recall that upwards of 1 in 6 furries are zoos.

Through most of last century, and still today in many parts of the world, homosexuality was considered to be abhorrent. This belief, of course, didn't prevent or reduce the number of homosexuals: it simply made for a lot of unhappy people. Freud believed that elevated suicide rates of young homosexuals was evidence that homosexuality is a mental illness. Fortunately this belief no longer prevails and homosexuality is accepted as a legitimate sexual orientation.

Zoophiles are in a similar bind in today's society. Tolerance and acceptance amongst the furry group, where zoos are so numerous, will do a lot of good.

Many furries argue that the community is too tolerant. This is a point of view with some merit; self-policing helps reinforce positive behaviour. However I think the sentiment is often misguided -- it's important to differentiate between what is innate, and what is a choice.

I'm not arguing for unconditional tolerance.

Instead, I'm arguing that zoophiles should be accepted for who they are. They should not be castigated or shunned for something that's innate. Zoophilia is a sexual orientation. We should encourage discussion about how one might become a happy, ethical zoophile.
