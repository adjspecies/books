\articlehead{The Geek Experience}{JM}{2012}

I received some fairly strong reactions to the short article I wrote about geeks a couple of weeks ago. The article is probably the slightest and most trivial contribution I've made to [adjective][species] in my ten or so articles to date, so I was expecting some criticism about its glib tone and sweeping statements. But I didn't expect that it would be read so differently by geeks, compared to non-geeks.

The geeks thought I was being unfair by connecting positive geeky behaviour to anti-social behaviour. Several people pointed out that geekiness is sexy and mainstream, whereas the anti-social behaviour I attributed to many geeks might be considered `nerdy'.

On the other hand, the non-geeks felt the opposite: they thought the article was an over-the-top love letter to geeks and geekiness everywhere.

As an aside, I'm happy to report that the geeks all responded to me on the internet; the non-geeks all commented as part of a `real life' conversation. (Stereotypes: confirmed.)

In this article, I want to explore the geeks in a bit more detail.

Geekiness is a big part of the furry experience. It's the most obvious, and accordingly laziest, way to characterize the community from the outside. For a new furry (or perhaps a journalist), it's a bit overwhelming -- and it's easy to come away slightly shell-shocked after exposure to those geeks who can be low-empathy and unsocial.

For those people who exist in a geeky world -- perhaps a programmer who plays tabletop RPGs with her friends -- the geekiness of furry is not going to be so pronounced. I was recently shown a Livejournal post by Kuddlepup, talking about how furry is seen by people inside geeky fandoms.

Visiting a steampunk convention, Kuddlepup reported that furries are seen as being anti-social. But not because of poor personal hygiene or unsocial behaviour, but because furries at conventions \textit{only go to see their friends}.

KP suggests that furries are being exclusive and vaguely cliquey, and I understand how it could be seen that way. However I'd argue that furry is fundamentally different from geek fandoms -- it's about personal identity. Furries are spending time with their friends because they can explore their identity beyond the superficial \textit{I'm just a wolf in a turtle-neck and glasses.}

Comparing the reactions to KP's post with those encountering the community from outside geekdom, I think there is a parallel with the contradictory reactions I received to my previous article. The different points of view are easy enough to explain, I think: there are many geeks in furry, but it's not a geek phenomenon. Our geeks and non-geeks just have different reference points. Everything, as always, is relative. (KP goes on to make a few other interesting observations and I'd encourage you to read his post in full.)

Regardless of how furry is seen through the lens of geekdom, it doesn't help us understand why we have so many geeks.

One thing we can say for sure: there is a preponderance of furries with non-traditional sexuality and gender identities. We've established that a very large number of furries re-evaluate their sexual preference after they discover the community; the full spectrum of gender identities are well-represented; furries exhibit a lot of unusual sexual behaviour (like the zoophiles, for example).

The breadth of humanity seen within furry is one of its great traits, because exposure to different people helps us all to become more understanding and tolerant. Furry is great for those people with unusual sexual and/or gender identity: furries can roleplay these parts of themselves, online or even in person.

Roleplaying as an animal person allows many furries to slowly accept and ultimately embrace their unusual gender and/or sexual identity. This `baby-steps' approach is healthy and low risk from a physical and mental point of view. It's one of the ways that furry provides a positive environment for personal change, and it allows us to enjoy the irony of furries using an imaginary animal person as a vehicle to acceptance of their true self.

This virtue of the furry community is why, I think, that furry attracts so many (male) geeks. Geeks are often perceived to be weak, and this doesn't mesh well with a society that conflates `masculinity' with `strength'. Geeks are often bullied because they don't meet society's assumed roles.

Disclaimer time: I'm aware that geek culture has become more mainstream in recent years. However `geek chic' is still very marginal. Looking into the most important reflector of mainstream human society: geeky TV characters are still, for the most part, held up as figures of fun. There is not a significant difference between the characterization displayed in The Big Bang Theory and, say, Steve Urkel.

At the risk of making a serious philosophical point using Urkel, it's telling that SU would sometimes transform into a masculine, suave, jamesbondian alter ego, named Stephan Urquelle. The contrast between our nerd and his masculine alter ego -- like Superman \& Clark Kent, Urkel \& Urquelle, and many other similar examples -- clearly delineates society's position on what it means to be a man. The opposite of a man?: the geek.

There are parallels here with many furry geeks, who present as a furry self with a very different outward attitude, in mind and body. Furry avatars often bear little relation to the human behind them and this is particularly prevalent amongst the geeks. There are very few overtly geeky furry avatars.

Happily, we furries are accepted as the animal person that we purport to represent. This means that our geek can present as Lord HyperDragon and be accepted on those terms. Just like someone can experiment with gender online, our geek can mentally test-drive Lord HD's body and attitude. Such roleplay often leads to change: it may be that our geek starts to become more like Lord HD over time -- maybe in attitude, maybe in sexual behaviour, maybe by hitting the gym.

Now, before I get castigated (again) for lazily stereotyping geeks as unsocial and unwashed, I want to be clear that this isn't my intent. Geeks may be the single most obvious feature of the furry community, but the definition of `geek' is slippery. I can't discuss the geeks without making some broad statements that will be fundamentally wrong in some way. But the geeks are different from the non-geeks -- and there is some evidence that the two groups have different experiences within furry.

In my previous post, I tried to assess the number of geeks by looking at the response rate to three questions from the Furry Survey:

\begin{quotation}
  Would you describe yourself as:
  \begin{itemize}
    \item a fan of RPGs? (Yes 55)
    \item a fan of science fiction (Yes 61)
    \item a fan of anime (Yes 49)
  \end{itemize}
\end{quotation}

I'm going to use the response to `science fiction' to explore survey data from those that answered `yes' or `no', two groups that are respectively more or less likely to be geeks. I chose `science fiction' because I think it's the least ambiguous question of the three. However the trends are similar for the other two questions.

With Klisoura's help, I compared new furries with furries who have been in the community for at least 5 years. I looked at the decline in heterosexuality, which is the clearest statistic for the phenomenon of furries re-evaluting their sexual preference.

\begin{itemize}
  \item for science fiction fans, the decline in heterosexuality is 42\% (over 5 years)
  \item for everyone else, the decline in heterosexuality is 54\% (over 5 years)
\end{itemize}

Statistics alert: this data is of very low quality. There are many reasons why it is statistically and logically flawed beyond all reasonable use. I present it because I expected to see this pattern before Klisoura and I performed the data mining -- it supports my hypothesis about the geeks.

To summarize, my totally unscientific hypothesis and data analysis suggests that furry disproportionally attracts people who don't fit into mainstream society.

\begin{itemize}
  \item many people are attracted to furry because of unusual sexual and/or gender identity
  \item many (male) geeks are attracted to furry because they are not traditionally masculine
\end{itemize}

The obvious question raised by this theory:

\begin{itemize}
  \item Does furry exist simply as a mental stepping-stone for people to understand and accept their true selves?
  \item Or do people within furry learn to love themselves because the community is such a positive environment for introspection and real change?
\end{itemize}

It's difficult to say which premise is the cart, and which is the horse. It's a question that, I think, is fundamental to exploration of furriness itself. So as a horse, I'm going to continue to pull this philosophical cart, exploring the question within the pages of [adjective][species]. (I'm also going to continue to torture equine-themed metaphors.)

But the cart can wait for future articles.

For now, I'm happy to reflect on all the positive things that furry brings. We
all know people who, through furry, have improved their lives. This might be through re-evaluation of sexual preference or gender identity, or maybe they've lost weight, or improved their job situation, or maybe just gotten a little better at dealing with the difficult job of being human.

I think that the abstraction of one's self into an anthropomorphic avatar is what makes furry so personal and rewardingly social. Hopefully we're all better people for this weird and wonderful furry experience.
