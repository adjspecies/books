\articlehead{The Hypnotic Beast}{JM}{2012}

Hypnobeast is the cheery face of furry hypnotism. Best known for his retro FurAffinity banner ads, HB is a qualified hypnotist offering a service tailored to the furry community.

I enjoyed a long chat with him recently where we talked about the utility of hypnotism within furry, the trials of being a professional furry, and how to react when people ask you to rape them.

My reflections and some highlights follow in the article below.

Hypnotism is a natural pursuit for furries. The trance state is like a vivid waking dream: hyper-alert yet perfectly relaxed. This state, with guidance, is the closest that any of us will get to living and feeling our imaginary furry bodies.

It helps that furries usually have strong imaginations and vivid internal lives. Our furry alter egos are already well realized within the community: hypnotism allows us to take that a half-step further.

In a session with HB, he will ask you about your furry identity and your reasons for choosing their form. Given the premise that furries craft their avatar with great care, HB will reflect the language of that creation. So if you associate (say) your inner fox with intelligence and creativity, HB will use language during the session that reinforces those concepts.

The process itself is simple enough: a few minutes of relaxation, followed by some image-rich wordplay designed to engage your imagination, followed by exploration of your furry body as if you were inside it. The experience is different for everyone, but the biggest variable is the skill of the hypnotist.

HB is a proper old school furry. He started by lurking around alt.fan.dragons as a tween in the early 90s, eventually graduating into the growing furry world. Like a lot of furries who discover the community at a young age, HB found high school difficult and was drawn into the open-minded, respectful, and tolerant online furry community as his primary social outlet.

HB developed an interest in hypnosis and dabbled amongst other amateurs online. I suspect that HB's interest in hypnotism originated from a desire to draw away from life where he was an outcast, to feel closer to the virtual furry world where he was accepted and loved.

After school, HB earned a degree in psychology and, with no interest in becoming a therapist, underwent formal training as a hypnotist. Nowadays, he works as a hypnotist inside furry (as Hypnobeast) in combination with a more traditional practice out in the real world. He has an office but, curiously for a profession that requires a close connection between practitioner and client, prefers to use Skype.

Hypnobeast is his professional virtual furry hypnotist. The Hypnobeast identity allows HB to separate his furry work from his regular practice. It also provides distance from his personal furry identity. This simple idea has proved surprisingly complex… but more on that later.

The cheesy imagery of Hypnobeast -- all swinging pendulums and mesmeric spirals -- is probably a marketing masterstroke. HB is a little less sure because he only gets exposed to the extreme reactions: either prospective furry clients or those making fun of Hypnobeast's 1930s-travelling-mesmerist image. But it's attention-grabbing, fun, and unforgettable.

HB's marketing may suggest that his style is dazzling and demanding, yet this is not the reality. HB is an Ericksonian hypnotist, which means that his style is friendly and permissive. He will look for, and ask for, regular feedback during a hypnosis session.

He controls the rhythm of his voice and chooses carefully crafted phrases, delivered to guide you into a relaxed state. He does this while reflecting your own words and conversational style, noted during the getting-to-know you chat at the beginning of a session.

Craftsmanship is the value of a professional hypnotist, and HB is a true craftsman of words. HB understands the language of furry introspection.

Early in our chat, HB told me that he is shy, which he immediately disproved by happily chatting away with a relative stranger for the next hour or two. He might be better described as vulnerable, as he reveals a lot of himself in conversation. The rapport between hypnotist and client is all important, and HB's openness is charming and disarming.

While chatting about his path into becoming a hypnotist, HB was open about his difficult and relatively unhappy adolescence. Throughout school, he coped by disengaging from the world and spending a lot of time inside his own head. Like a lot of people who study psychology, I suspect that HB chose his degree because he was hoping to learn about himself.

Starting up as a professional hypnotist is a difficult task. There are set up costs -- insurance, office space, union fees -- and no client base. The task is especially difficult because hypnotists are usually focussed on treating a symptom rather than exploring a cause. Most clients will see a hypnotist two or three times. Regular clients (like a therapist might have) are rare.

HB is less interested in hypnotism as a cure. He sees his ideal role as providing relaxation sessions, on the premise that relaxation is good for long-term physical and mental health. And inside the furry community, HB hopes to build up a client base who appreciate the joy of an occasional walk inside their furry body.

It's tough for anyone looking to be a professional furry and HB is no different. He has found new clients to be rare, and last-minute cancellations to be common. He has found it difficult to market himself in person: he's attended cons in a professional capacity, but learned that a well-dressed professional hypnotist won't receive walk-up trade; he's performed free group hypnosis sessions at Further Confusion and Califur, for which he received positive feedback but little paid follow-up.

HB has had most success by advertising on FA directing people to hypnobeast.com. His current offer -- \$5 for a first session -- is going well, drawing some new clients and increasing his visibility inside the community. But his business has not yet grown to a significant number of paid furry sessions.

And then there is the sex. The trope of hypnotist-as-rapist is common in furry erotica and pornography, and it tests HB's patience and ethics.

HB isn't anti-sex, and advocates hypnosis to enhance sex. Hypnosis is commonly used to treat sexual problems -- erectile dysfunction, or management of vaginal pain during intercourse are probably the two most common -- and also to broaden sexual horizons. This requires utmost professionalism from the hypnotist, to ensure a controlled environment for the client. This does not work well with furries.

One of the strengths of the furry community is its openness towards sex. Furries are okay talking about sex freely but this leads to problems in a therapeutic context. HB has found it challenging to maintain a professional bubble when dealing with sex-related issues: amongst furries, there is a fine line between discussing sex and flirting.

After a few clients who looked to inappropriately cross HB's professional boundaries, he has become wary of offering such services. Hypnobeast, the character, has been created to provide a professional identity. Professional conduct is important, not just for HB's integrity but also under the terms of his insurance.

It's nice to be considered an object of lust. Hypnobeast is such an object in the eyes of many furries, and some have been very direct in suggesting rape-fantasy roleplay. HB has asked me to keep details confidential, but suffice to say that some of the sexual offers he has received via Hypnobeast are surprising in their complexity, scope, and creativity.

HB is a professional and gives short shrift to anyone flirting with/at Hypnobeast. In a lighter moment, he expressed wry frustration that his personal furry characters rarely, if ever, receive such attention. Or, as he puts it: “please, please, stop hitting on Hypnobeast”.
