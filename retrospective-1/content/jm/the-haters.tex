\articlehead{The Haters}{JM}{2012}

In the April 2012 Journal of Personality and Social Psychology, there is an interesting piece of research that presents evidence that ``homophobia can result, at least in part, from the suppression of same-sex desire''.

There are two ways that this conclusion might be perceived:

One: hypertolerant types might think this provides a bit of scientific ammunition against the bigoted. We can take the logical next step and apply this idea to haters within furry, which reframes them as closeted versions of the object of their hatred.

Two: skeptical types might think that psychological experiments are never statistically sound, and that academics are pre-disposed to presenting conclusions that match up with their pre-existing beliefs.

Both of these perspectives are valid if extreme. As always, the truth is somewhere in the middle. I'm going to explore this, and how this is reflected within the furry community, but first I'm going to talk about cognitive psychology and chronobiology.

The research on homophobia used the phenomenon known as priming, a common tool in psychological experiments. The classic priming example is as follows:

\begin{quotation}
  Fill in the blank to create a word. Answer with the first word that comes to mind.
  \begin{quotation}
    For example: W\_SH becomes WASH

    Create a word from: SO\_P
  \end{quotation}
\end{quotation}


Presented with this question, a very large majority of people will answer SOAP. If you remove the example from the question, the most likely response is SOUP. The responder is ``primed'' with the word WASH, so SOAP comes to mind first.

This well-understood phenomenon can be combined with a tool used in the study of chronobiology, which is the science of how we perceive time.

It has been shown that, if you flash a word or phrase on a screen quickly enough, it will not be consciously registered by the part of the brain that deals with language. However it will be read and understood on an unconscious level.

As long as the word is flashed up quickly enough -- typically less than 50 ms -- it will not be consciously registered. The threshold at which messages are not consciously registered is key to chronobiological experiments, which study how time is experienced under different circumstances. This word-flashing technique is used in experiments testing the phenomenon of ``slow time'', commonly experienced in stressful situations. Scientists measure the change in message-recognition threshold for subjects under extreme stress.

(My favourite experiment: subjects lie face-up on a net at the top of an old silo. The net is dropped, and a word is flashed on a screen for the subject to read while in free-fall.)

Words flashed in such a fashion are known as subliminal messages. And subliminal messages can act as a ``prime''. Someone can be primed with WASH subliminally, and will be very likely to choose SOAP.

This technique isn't restricted to word-association games. Priming also affects reaction time to certain tasks. In psychological experiments, this is often a simple sorting task where a person will be asked to categorize an item.

In our homophobia experiment, subjects were asked to categorize images as being ``gay'' or ``straight''. The subject would be presented with a homo- or hetero-normative image or word (e.g. pictures of same-sex or straight couples) and asked to press a button associated with the appropriate category. The computer measured reaction time.

The catch? Subjects were subliminally primed with a word -- either ``ME'' or ``OTHERS'' -- before each test. Previous experiments have shown that this technique will reliably distinguish between self-identified heterosexuals and self-identified homosexuals.

A gay person would, in general:

\begin{itemize}
  \item React quickly when presented with a gay image after being primed with ME, or when presented with a straight image after being primed with OTHERS.
  \item React slowly when a gay image was primed with OTHERS, or when a straight imagine was primed with ME.
\end{itemize}

A straight person will usually react in the opposite way.

This particular experiment was designed to test the effect of upbringing. The participants were asked a series of questions about their childhood and family. Among these questions, each participant was asked about their own attitude to homosexuals (for example: would you feel comfortable if your roommate was gay?).

Based on these responses, participants with intolerant attitudes were lumped into a group loosely termed ‘homophobes'. (As you might expect, this group was mostly populated with people who grew up in a homophobic environment.) The experimenters compared the results for three groups:

\begin{enumerate}
  \item Self-identified homosexuals
  \item Homophobes
  \item Everyone else
\end{enumerate}

Surprise, surprise: the experimenters discovered that a significant proportion of the homophobic group reacted the same as the homosexual group.

The scientists concluded that there is ``a discrepancy between self-reported sexual orientation and implicit sexual orientation'' because ``given the [parental] stigmatization of homosexuality, individuals may be especially motivated to conceal same-sex sexual attraction''.

To put it another way: they concluded that about 20% of homophobes are actually closeted homosexuals.

The leap of logic from ``reacts the same way as a homosexual'' to ``is a homosexual'' is questionable and difficult to prove. This technique is classically used to test covert inclinations such as racial prejudice. Our homophobia tests are going a step further: they're not just measuring attitude. A potential counter-hypothesis might be that our homophobic subject becomes unconsciously enraged, thereby improving reaction time, after having ``ME'' linked with homosexual images. I'm not aware of work that has tested the validity of this idea.

However, it's compelling to conclude that someone closeted with an unusual sexuality might exhibit hatred towards that sexuality. If someone is hostile towards a certain sexuality, it may help them feel as if they are internally ‘proving' that the sexuality doesn't personally apply. What manifests as negativity towards others is actually self-hatred.

In the furry community, we don't have a significant problem with homophobia. But we do have a problem with hatred towards some of the more unusual sexual orientations and interests, such as transexuals, babyfurs, zoophiles, and more. In all cases, people are being attacked for things that are innate.

Here is a high-profile example of hatred, which was linked to me by a babyfur friend of mine. Back in the salad days of Livejournal, furry humourist 2 The Ranting Gryphon posted an offensive rant aimed at babyfurs. It's particularly egregious for several reasons:

\begin{itemize}
  \item 2's high profile means that his article is easy to find -- it appears if you google ‘babyfurs'.
  \item The events that 2 relates are almost certainly apocryphal. (In the comments, FWA security staff claim that they never heard about the events described.)
  \item Even if true, 2 takes one anti-social act and blames all babyfurs for it. He is being hostile towards an innocent group of people, whose only crime is having an unusual sexual interest.
  \item Plus, of course, the direct threat of violence.
\end{itemize}

2 posted a partial apology for his outburst a few days later.

I can't say whether 2 is a closeted babyfur but his behaviour is certainly consistent with someone struggling with self-hatred. It's safe to say that at least some of the haters are closeted versions of their target.

This means that our haters are not just angry: they are struggling with self-acceptance. It's unfair to take a hater to task for his position. Our hater is just reacting in a natural fashion to his own sexual interests or orientation: the anti-zoophile is very often a zoophile himself.

This is a natural, and unconscious, coping strategy. If you hate the hater, you're making the same mistake that he is: you're castigating him for something he has no control over.

Nor is it helpful to try to show our hater that he is wrong. As I have mentioned in previous articles, self-deception is a powerful force. If we see evidence that is contrary to our version of the world, we disregard it in a way that reinforces our existing belief.

It is far better, I think, to treat everyone -- even the haters -- with respect and nonjudgemental curiously. Furry is a great environment for people to grow, and learn about themselves. There are many examples of ex-haters out there and none of them have changed their ways by being shouted down. Furry fellowship and understanding is a powerful force for good.

Personally, I recommend this is best done in person over a beer. But given that we're furries, I assume it would also work while engaging in a statistically unlikely sexual act on FurryMUCK. It's worth a try.
