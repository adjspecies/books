Most people are familiar with feelings of isolation and loneliness. Loneliness can lead to feelings of depression.

It's worse if you are young. It takes a long time to become happy with yourself, if that is ever fully achievable. Most of us experience personal growth as we age. If you don't like yourself, which is much more likely if you are young, it's easy to assume that you're somehow at fault for being lonely.

It's worse if you are male. Men are more prone to depression and suicide. It's believed that this is biological.

It's worse if you have an unusual sexuality or gender identity. Someone who doesn't fit into society's mainstream will often find themselves marginalized. This adds stress to day-to-day activities, possibly a feeling of ‘being judged' or feeling outcast.

Furries fit the description of a high-risk group for depression. We're young (median age 22); male-dominated (80\%); unusual sexualities (69\% self-report as ‘not heterosexual') and genders (26\% self-report as neither completely male nor female).

Furries are more likely to be socially isolated than non-furries. Members of the furry community -- our friends, peers and, in some cases, de facto family -- are spread across the globe.

Non-furries are more likely to make friends amongst those they grew up with. It's common for people to make friends at school and keep them for life. Their friends and support groups tend to be located nearby, and they are more likely to find value in mainstream bonding activities, such as those you might see depicted on a billboard advertising cornflakes.

Much furry socializing, especially amongst the isolated, occurs online. Online contact can lack nuance and is often a poor cousin to face-to-face contact. Anyone listlessly lurking around social corners of the internet (like FA, Facebook, Twitter or IRC) can attest how easy it is to feel lonely online.

It's easy to become downhearted by loneliness. However loneliness, isolation, and depression are very normal feelings, familiar to everyone. There is nothing innately wrong with feeling disconnected from the world.

(There is a big difference between feeling lonely and being clinically depressed: the first is a negative feeling; the second is a mental illness. Just like feeling outcast in a social situation doesn't make you autistic, feeling lonely doesn't mean you're clinically depressed. Anyone with doubts should consult a doctor.)

There are effective ways to combat loneliness that are especially applicable to the furry community. Our online culture, our animal-person roleplaying, and our introspective assessment of ourselves and the world are all great tools.

Loneliness and depression is a common human trait. The problem -- and a solution -- is hinted at in Jonathan Swift's 1726 novel Gulliver's Travels. In this excerpt, the yahoo race is an analogue for humans, curiously regarded by a race of rational horses:

\begin{quote}
  A fancy would sometimes take a Yahoo to retire into a corner, to lie down, and howl, and groan, and spurn away all that came near him, although he were young and fat, wanted neither food nor water, nor did the servant imagine what could possibly ail him. And the only remedy they found was, to set him to hard work, after which he would infallibly come to himself.
\end{quote}

To use a more modern concept, consider Maslow's hierarchy of needs (link), a broad psychological theory. The hierarchy of needs is not used in serious psychological circles, but is a useful blunt instrument to frame the problem of isolation.

Maslow posits that we are fundamentally driven by (1) atavistic impulses, like sex and sleep. Once these needs are met, we require (2) personal safety. This is followed by (3) a need for social contact. When this is met, we are able to pursue further needs up towards ``self-actualization''.

If you are reading this article here on [adjective][species], it's likely that you live in a world where you are able to meet these first two basic needs, like food and shelter. For most furries, the need for social contact is the first real hurdle towards reaching self-actualization.

Swift identifies the occasional need for an external impetus to get us out of a funk. Maslow shows that social contact is a fundamental need. With this as a guide, we can take action to draw other people into our world such that we become more connected and engaged. The following suggestions are mine, tailored towards the furry experience -- they are by no means exhaustive. Consider it a starting point.

\textbf{Firstly}, consider that happy people are the easiest to get to know and like. Unhappy or aggressive people are intimidating; happy people are welcoming.

It's very easy to be negative online. This is especially true if you are feeling lonely and depressed, and you're hoping to share your own experiences.

But there is great value in emulating the way that happy people express themselves: ``act happy'', regardless of how you feel. There are three immediate positive effects:

\begin{itemize}
  \item If you appear happy, you will be more approachable. This will help you make a connection because others will find you easier to chat with.
  \item Acting happy will give you some of the experience of being happy. You will learn the lexicon of happiness, and your body language will change as well (even if you are tapping away at a laptop). The words, expressions and feelings of happiness will then be available for you when you need them -- feeling happy will feel normal, not alien.
  \item Acting happy changes your brain chemistry in much the same way as actually being happy, which means pretending to be happy will make you happier. The adage ``fake it till you make it'' implies a cause and effect that is very real.
\end{itemize}

\textbf{Secondly}, try to chat with people in ways that make them comfortable. This means chatting on their turf, and choosing a topic that is the favourite of your conversation partner(s). This might mean visiting someone's house and asking about their day at work; in the furry world this is more likely to mean using IRC to chat about someone's thoughts on operating systems.

You're practising an valuable personal skill -- that of empathy -- but more importantly you're helping your conversation partner. People are always more engaged when talking about things that are personally important. Even though it might be a topic with which you have no familiarity, ask questions and try to keep your own thoughts out of it. Your conversation partner is more likely to seek you out for future chats, and the range of topics will naturally broaden.

This technique has the added bonus of removing any personal pressure on the social experience. You don't need to think of a topic or something clever to say, yet you can drive the conversation.

\textbf{Thirdly}, be active and risky in your conversation. Speak on a controversial topic, or be very direct. This will encourage other people to chime in. As they become engaged, switch to a passive role and ask for more information about their thoughts.

This can be a short route to a fun and active conversation. It's especially useful in group situations, like IRC or in-person furmeets, where people often tend to idle quietly.

\textbf{Fourthly}, try out some ``life hacks'', to trick yourself into doing things you know are good for you.

A 1999 study published in the Journal of Behavioural Decision Making asked people to participate in a (fake) experiment. They were asked to choose a DVD to watch while they waited for the ``experiment'' begin. They were given a choice between a popular highbrow film (like Schindler's List) and a popular lowbrow film (like Mrs Doubtfire). Participants who made their selection three days before the experiment were much more likely to select a highbrow film than those who made their selection on the day.

The participants knew that seeing a highbrow film would confer greater value over the long term, while the lowbrow film would be less challenging. People chose the lowbrow film on the day because they were, essentially, procrastinating. (We all want to improve ourselves, but right now we're listlessly lurking around social corners of the internet.)

The simplest way to overcome this natural procrastination is to plan things in advance. If you commit yourself to a social activity that you know is good for you -- perhaps exercise, or webcam chat, or some tabletop gaming -- you won't give yourself the option of procrastinating.

There are some excellent mind hacks, or productivity tools, available online for free. There are three geared towards geekier types that I'd recommend:

\begin{itemize}
  \item Getting Things Done (GTD) (http://www.43folders.com/2004/09/08/getting-started-with-getting-things-done)
  \item The Pomodoro method (http://www.pomodorotechnique.com/)
  \item The Hacker's Diet (http://www.fourmilab.ch/hackdiet/)
\end{itemize}

\textbf{Finally}, please allow yourself to good-naturedly fail from time to time. It's inevitable that we all feel like failures, or feel depressed, or feel lonely. It's normal and natural.

If we feel bad about something, we tend to use black-and-white language. We use words like ``failure'' or ``fat'' or ``useless''. These terms make the obstacles to success look insurmountable.

But if we feel good about something, we use relative language. We use words like ``better'' or ``thinner'' or ``improved''. These incremental terms make much more sense, because they reflect the way we change -- slowly and steadily. When presented with a challenge, it's helpful to think of it using relative language.

This article is about loneliness but it also touches on depression and suicide. I encourage everyone to give themselves a free pass for depressive or suicidal thoughts, because they are a normal and common experience. But I'm not qualified to give advice to someone who is worried they may be suicidal.

Fortunately a furry friend of mine is a qualified medical doctor. (And a horse.) His advice follows -- please heed it if you've read this article and are worried about yourself.

\begin{quotation}
  If at any point you don't feel safe within yourself, call emergency services. Don't hesitate or second-guess yourself. However you may have arrived at this point, it's not the kind of thing that gets better when you think about it. You can think about consequences when you feel you are safe.

  Regardless of how you view hospitals (and possibly psychiatric wards), I cite the emergency services here partly because of my professional background, and partially because their role is to guarantee that your emergency is taken seriously. Somebody will respond and will be there for you.

  If you don't like hospitals, or psychiatric wards, or doctors, to the point where you would rather die than see one, and you have a friend that is so good that they could guarantee you the above, then perhaps that's a viable alternative. They may not be equipped to deal with your crisis, but if you find yourself in this situation then it's pretty desperate, and may require desperate measures that you can only access in a hospital.
\end{quotation}
