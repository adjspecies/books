\articlehead{Relationships Within Furry}{JM}{2012}

Last week I wrote about one of the great inequalities within furry, the gender imbalance. Furry, measured by self-reported gender or by biological sex, is around 80\% male and 20\% female.

I talked about how the community can be unwelcoming for women, and how the dearth of furry women has a negative effect on some heterosexual furry men. When it comes to furries available for heterosexual relationships, the guys outnumber the girls by around three to one. This is a problem for those heterosexual male furries who feel that their partner needs to be a furry too.

In my article, I assumed that a relationship is defined as something exclusively shared between two people. But it's wrong to characterize furry relationships as an exercise in `coupling up'.

Many furry relationships are non-monogamous, non-exclusive and/or only involve more than two people. I don't have any data to draw upon, but it's possible that exclusive monogamous couples are in the minority. Furry relationships, in my experience, can cover a very wide range of potential arrangements.

Online relationships within furry are relatively common. We think of online relationships as routine, but the entire concept is a new one. The relatively recent advent of the internet has brought about these relationships: it allows the sort of intimate communication necessary for the formation of a close emotional bond.

The boundaries of online-only relationships are being explored within furry. Owing to the international nature of the furry community, we tend to congregate online. This has inevitably led to the concept of an ``online partner'', where someone will consider themselves to be in a relationship -- even virtually married -- with someone they interact with online (and sometimes solely online).

This is an inevitable direction for society, outside furry, as the world moves online and becomes more virtual. We furries, with our virtual animal-person alter-egos, are testing the possibilities of online relationships. We live a kind of virtual ghost life where our furry character lives inside a collective imagination as a real-but-not-real apparition. We're taking the virtualization of life to a new, but logical, level.

Virtual lives and virtual relationships will become more common outside furry as time goes on. Society will adapt to the changes, with furries at the vanguard, just as society is adapting to the radical changes in relationships over the past 60 years or so.

Since the 1950s and 1960, society has moved away from a patriarchal relationship model. Thanks mostly to advances brought about by the second wave of feminism, we live in a more equal society where a relationship is not based on the assumption that one gender is superior. This has benefitted women but also gay couples: a relationship of equals provides a great model for a monogamous gay relationship, because society no longer expects a couple to respectively occupy the roles of ``man'' and ``woman''.

This is not to say that there are no opponents to these changes. In conservative parts of the world, there is a natural resistance to the way that relationships have changed -- see this Queensland, Australia election ad (from 2012!) (link updated) -- which assumes a gay relationship must involve gender roles.

That advert and the attitudes it embodies are fundamentally sexist and homophobic. Many people are anti-gay-marriage, however I don't think they are necessarily bigots. In many cases, people are anti-gay-marriage because they see it as another step in the wrong direction: another step away from a 1950s-style marriage.

The change in relationships towards a gender-neutral model is a positive step, but there have inevitably been some negative outcomes. The clearest is a ballooning divorce rate. Divorce is always difficult, for the couple and any children, as many people reading this will know. The previous model, where a woman had limited power within a marriage, was more stable -- much as slaves are more stable than employees.

There are always growing pains during times of rapid change. Society's shift towards the virtual world -- as explored by furries -- is also fraught. Many furries reading this will have learned that non-traditional relationships, whatever their flavour, are difficult to maintain over a long period of time.

There are a few reasons for this:

\begin{itemize}
  \item There are limited role-models for unusual relationships, which means that the participants must learn by making mistakes, with limited experience and wisdom to draw upon.
  \item Unusual relationships don't mesh well with society, which can cause friction with the outside world, perhaps when dealing with family or workmates.
  \item There may be a lack of legal recognition.
  \item If more than two people are involved, jealousy issues can become more significant.
  \item Communication, which is the cornerstone of any relationship, may become more difficult.
\end{itemize}

Because of these problems, and others, unusual relationship structures will fail more regularly. Relationships that most closely match the norms of society are more likely to be successful.

Regardless of your own preferences, aspiring to be in a healthy relationship is a good thing. But that doesn't mean you need to be in a rush.

The average age of marriage for men is close to 30 in the US. For furries looking to enter a long-term commitment, whatever that might be, age is less important.

In the furry world, it's easy to meet new people. This doesn't always happen in the non-furry world -- people often find their social circles shrink in their 20s and 30s. This is the opposite of the furry experience.

Single furries may also find it easier, compared to non-furries, to find rewarding intimacy and sexual contact within the context of a friendship. I think that the intimacy of many furry friendships puts us in a good position to learn about ourselves, and what we desire from a partner. Most good relationships can only work when all parties are fully-formed adults. People who get married young often find themselves yearning to explore life when they get older. This phenomenon is common enough to be almost stereotypical -- a midlife crisis or a seven-year itch.

Anyone in a relationship or seeking a relationship -- which covers almost all of us -- is most likely to be successful if they are introspective and autonomous. Fortunately we are furries, which is all about exploration of identity.

I'd love to hear some stories -- good and bad -- from readers who have explored unusual relationship structures.
