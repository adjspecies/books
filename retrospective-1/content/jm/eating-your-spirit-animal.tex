\articlehead{Eating Your Spirit Animal}{JM}{2012}

Food, like sex, sometimes has a complex political subtext.

The politics of food made from animals can be especially complex. There are ethical, environmental, moral, and health arguments for and against the consumption of meat.

There are also gender issues associated with meat consumption: why is meat, particularly red meat, associated with masculinity? For example, check out these adverts from Australia, the UK, and the US: all satirical, and all accept the premise that masculinity is inexorably connected with meat consumption. Some feminists believe this connection reinforces objectification of women, arguing that it casts women as the passive supplier of flesh, and men as the active devourer.

Regardless of your own point of view, this seemingly simple basic need for the sustenance of life – the need to eat – has become a complex political subject.

And it's complicated further if you're a furry. If you identify as an animal person, it's impossible to ignore that we live in a world where animals are commodities.

Furries tend to celebrate their animal identity in varied and creative ways. Most furries create an alternate personality – an anthropomorphic animal avatar – and accept this identity as a version of themselves. We furries `become' this identity, because we act like we genuinely are our avatar. This belief in our alternate self makes our furry identity real.

Our day-to-day actions are often interpreted through the lens of our furry identity. This extends to the food we eat. So a furry who identifies as a carnivore might enjoy eating meat. Or perhaps a meat-lover might choose to identify as a carnivorous species.

The arguments for and against meat consumption can be summed up succinctly: meat eating is bad because it causes suffering; meat eating is good because it's tasty and a societal norm.

These points of view (both of which are valid and true) can be given a furry twist. A carnivore furry might be drawn towards eating the natural prey of his species, such as a cheetah with a taste for venison. Or a vegetarian might make a connection with their herbivorous avatar. And if your furry species is available for human consumption… it gets complicated.

The four most common furry species available for human consumption are listed below. This data, as ever, comes from the Furry Survey.

\begin{itemize}
  \item dog, 8.4\% of furries
  \item rabbit, 2.7\% of furries
  \item horse, 1.4\% of furries
  \item kangaroo, 1.0\% of furries
\end{itemize}

All other species commonly consumed by humans are chosen by less than 1\% of furries. The full league table can be found on an old Livejournal post of mine, here.

Some of these furries go out of their way to eat their spirit animal. I can personally think of two examples: a friend of mine once species-hopped to kangaroo largely because of his affinity for roo meat; and a deer friend who was thrilled to find venison ham for sale.

But for other furries, eating their spirit animal is taboo. In many cases, the reason for their revulsion is closely tied to their choice of species in the first place. For people who work or live with animals, and feel a strong affinity for them based on that social experience, the idea of eating those animals can be akin to cannibalism.

It's also common for the taboo meat to be one that is not normally culturally considered to be food. It's especially likely where the species in question is normally thought of as a pet or companion animal. In western-centric furry circles, this often applies to dog meat and horse meat.

Horse is a common meat in France and Japan, among other places around the world. Dog is a common meat in parts of Asia and Africa.

The ethics of raising dogs or horses for meat is no different from other animals. Whenever an animal is raised as a commercial enterprise, there will sometimes be a conflict between the best interests of the animal and the greatest profit. Sometimes the best interests of the animal will come second. This is true even where the animals are not being raised for meat: it's true whenever there is a profit motive, including work animals (such sheepdogs) and animals raised for sport (such racehorses).

This ethical argument does not apply when there is no commercial interest, such as raising a pet.

There is suffering involved in the raising and slaughtering of any animal. There is no reasonable argument that raising horses or dogs for meat is `bad', but raising, say, cows or pigs is `okay'. Horses and dogs are domesticable and intelligent, but so are pigs: pigs can be domesticated as pets or as work animals (truffle farming for example).

The commonly-held taboo on whale meat is similarly flawed. Whale is eaten in Japan, Norway, Iceland, and elsewhere. The arguments against whale hunting and consumption are hypocritical unless you are applying the same arguments to mainstream meats.

The arguments against whale meat can be roughly condensed into:

\begin{itemize}
  \item Whales are intelligent creatures who suffer during the hunt. (It's likely that more suffering is caused by pig farming, as they are very intelligent and often subject to poor conditions during life.)
  \item Whales are endangered due to overfishing. (Much like many fish species around the globe.)
\end{itemize}

I've always thought that arguments against consuming whale, much like arguments against consuming dog (or horse), often smack of racism. Firstly, I don't think people would hold such opinions if they lived in a culture where whale or dog meat is the norm. Secondly, the argument is often framed such that the target (eg Japanese for whale; Koreans for dog) is presented as a barbaric `other', a subtle dehumanizing practice common to much racist hatespeak.

That said, there is no problem with having an aversion to the idea of eating a particular type of animal. The emotional response associated with eating your spirit animal can be particularly strong. For many people, this is an important part of being a furry.

There is no requirement for any personal choice to be irrefutably logical, be it religion or politics or attitude towards food. It's natural to think of one's self as rational, but this is wrong: we are animals and therefore driven by basic survival instincts. There is only one requirement for a personal choice: don't try to enforce your choice on other people.
