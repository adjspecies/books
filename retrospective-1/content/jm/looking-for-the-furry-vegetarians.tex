\articlehead{Looking for the Furry Vegetarians}{JM}{2012}

In 2008, Klisoura’s furry survey asked ``Would you describe yourself as an advocate of animal rights?''. 43\% of you chose `yes'.

In surveys from 2009 onwards, Klisoura asks exactly the same question but only 27\% of you choose `yes'. What changed?: in 2009, a new question was added on the following line: ``Would you describe yourself as a vegetarian?''

This is an example of a phenomenon known in the psychology world as `priming'. When asked about animal rights and vegetarianism together, the thoughts of some users will have been drawn to their latest bacon sandwich and decided that, no, they weren't an animal rights advocate.

My favourite example of priming is a study that investigated voting patterns in Arizona in the 2000 election. That year, there was a proposition to increase school funding. Support for the proposition was significantly greater when the polling station was in a school, compared to support at other nearby polling stations.

It's natural to disbelieve the effect of priming in the furry survey, or in the Arizona school district examples. It suggests that we are all susceptible to change our opinion based our immediate surroundings. However priming is a common phenomenon and there are many examples: the science is unarguable.

The results in the furry survey could have been skewed in the other direction of course: if the question about vegetarianism were replaced by ``Do you support the prosecution of negligent pet owners?'', the number of animal rights advocates would have gone up.

The large priming effect in the furry survey demonstrates two things:

It's very difficult to write a survey, especially when you're asking for opinions.
Many people see a link between caring for animals, and choosing to eat them. This apparently simple connection is surprisingly controversial to many people.
I am vegetarian and I'm keenly aware that nobody likes a holier-than-thou attitude. The intent of this article is not to advocate vegetarianism. So let me get a few things off my chest:

\begin{itemize}
  \item Meat is delicious. It's delicious because the human body has evolved to take advantage of the copious nutrients in meat.
  \item But you don't need meat to be healthy. Studies of vegetarian and non-vegetarian Indian Hindus show no significant difference in life expectancy. (Western vegetarians live longer than non-vegetarians but this may be due to other lifestyle choices, such as smoking.)
\end{itemize}

Being vegetarian can be a hassle and requires vigilance. As far as I am concerned, the convenience and deliciousness of an omnivorous diet is a good enough reason to eat meat. It's just not for me.

Some vegetarians, like me, are ethical vegetarians. These people follow the general philosophy laid out by Peter Singer in his 1975 book, Animal Liberation. Singer's utilitarian philosophy can be summarized simply as `minimize harm'. An ethical vegetarian might consider their options for a meal and decide that a vegetarian pizza does less harm than a pepperoni (which does less harm, in turn, than a meatlovers).

A key premise for Singer's philosophy is that you must believe humans to be an animal. (This may be a problem for you if you are religious and you believe that God created man in his image.) If you accept that animals are capable of suffering, then you can weigh the suffering of those non-human animals against the suffering of a human animal. This explains why it's okay to slap a horse but not okay to slap a baby; this also explains why animal testing of medicines is a good thing.

It seems logical to me that this reasoning would be more likely to resonate with furries, people who usually identify with or as non-human animals. Furries are much less likely to consider human beings to be a special case in the animal world, and much more likely to think about animal welfare. Consider the charities supported at furry events, or the 27\%+ animal rights advocates.

So is there a higher proportion of vegetarians amongst furries? No.

\begin{itemize}
  \item About 4\% of furries taking Klisoura's survey ``consider themselves to be vegetarian''.
  \item About 4\% of people in western countries identify as vegetarian.
\end{itemize}

It's been suggested to me that meat-eating might form an important part of the identity of a furry with a carnivorous character. This may be the case for some furries, but it's not the case in general: analysis of survey data shows that a furry with a pure-carnivore character is just as likely to be vegetarian as a furry with a pure-herbivore character.

The key to furries and vegetarianism comes down to gender bias. Anyone reading this will be keenly aware that furry is male dominated. Survey data suggests that around 80\% of furries are male. (The women are also more likely to consider themselves only `weakly' furry.)

This is important because, out in the non-furry world, women are twice as likely to be vegetarian than men. (If you are male and vegetarian, like I am, the question you'll be most often asked is ``so is your girlfriend vegetarian?'' The correct answer, by the way, is ``I reject the premise of your question''.)

In the furry world, the same ratio holds: women are twice as likely to be vegetarian than men. If you adjust the data for this gender bias (the male:female ratio is 50:50 outside furry; 80:20 inside furry), the relationship between furry and vegetarianism looks very different.

\begin{itemize}
  \item If you are a male furry, you are twice as likely as a male non-furry to identify as a vegetarian.
  \item If you are a female furry, you are twice as likely as a female non-furry to identify as a vegetarian.
\end{itemize}

It's probable that the gap between furries and non-furries is starker still. Incredibly, a full two-thirds of non-furries who identify as vegetarian regularly eat meat and/or fish. I suspect that furries have a far stronger grasp of the definition of `vegetarian'.

Even so, I remain surprised that vegetarianism isn't more common amongst furries. The logic, while not for everyone, seems straightforward to me. I wonder if there simply isn't the critical mass for many furries to be exposed to the idea - vegetarians certainly have a reputation for being obnoxious and evangelical.

I saw Peter Singer plugging his latest book a few years ago. He talked about the publicity and positive criticism generated by Animal Liberation back in 1975, and how he expected that vegetarianism would quickly become more commonplace. He talked about his surprise that the proportion of vegetarians has remained static since then. (Not coincidentally, his new book explores the idea of the ethical omnivore.)

So perhaps I'm being naïve. As the priming example demonstrates, none of us are purely logical beings.
