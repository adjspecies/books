\articlehead{Why Language Is Important}{JM}{2012}

In a recent article, I talked about the importance of language in self-criticism. If you are trying to lose weight, it's useful to use relative terms (\textit{I'm getting thinner}) and counter-productive to use absolute terms (\textit{I'm fat}).

It's helpful to use language that suggests self-improvement, compared to the language of self-hatred. Even though both phrases (\textit{I'm getting thinner} / \textit{I'm fat}) describe the same thought, they imply different things: language affects perspective.

Writing here on [adjective][species], I do my best to use specific and neutral language. But it's difficult, especially when writing about sensitive topics such as sexuality, or fluffy concepts such as ``furry''.

An example of a sensitive topic: I used the term ``paedophile'' in my article on cub porn. I suggested that a subset of those furries attracted to cub porn are sexually attracted to children. I called them paedophiles.

It's a strong term. It is regularly used to refer to child molesters, however I meant it to cover anyone with an innate sexual attraction to children. I went on to use the term ``gold-star paedophile'', which is someone with such an attraction and the good sense not to act on it.

I used these terms because they are used by the psychologists I cited my article. However I knew that free usage of ``paedophile'' would colour my post, making it difficult to read and potentially enraging. Ideally, there would have been a term with the same definition, without the implied negative connotation, and less cumbersome than ``someone who is innately sexually attracted to children''.

An example of a fluffy concept: gender and sexual identity. A full 25\% of furries consider themselves to be neither completely male nor female, while less than 50\% of furries identify as homosexual, heterosexual, or bisexual.

This makes data analysis difficult. Here at [a][s], we tend to lump groups into large categories, so that someone identifying as ``mostly homosexual'' is considered ``homosexual''. This is a potentially offensive simplification, but it's the best we can do.

In the ultra-PC world of sexual politics, the cover-all term LGBT isn't always considered inclusive enough. This has led to propagation of ludicrous terms like LGBTQIA. While more specific, such terms are not useful: mainstream usage of LGBT includes all people with an unusual gender or sexual identity.

Such limitations of language are one of the biggest challenges of writing for [adjective][species]. There have been very few attempts to discuss the furry community intelligently, which means that there is limited language to draw upon.

A lot of furry analysis occurs in crowd-sourced, informal media such as forums or Twitter, or published in a news-byte format by the likes of Flayrah. This is all great stuff, but it doesn't generate discussion of sufficient depth to create a new lexicon, a foundation on which we can grow complex ideas.

There is little detailed furry analysis. Journalists barely scratch the surface, and the only scientific study (Gerbasi et al) proposed a ``species identity disorder'', interesting but not really relevant to the furry experience.

Here on [a][s], and in other long-form pieces (such as the occasional LJ article or Flayrah editorial), it's easy to get smothered by linguistic quicksand. Our occasional awkward phrases (like ``all people with an unusual gender or sexual identity'') and painfully self-referential articles (like, uh, this one) are partly a product of our limited furry lexicon.

This is all, of course, Makyo territory. He has written around the topic of language on several occasions, notably Doxa (exploring the ideas that make up our community) and On Words (discussing the varied interpretations of ``furry''). If you've read this far, then I'd recommend a read (or re-read) of both those articles.

I hope that [adjective][species] will help the introduction of new ideas and terms. Language does change within growing communities, notably the appropriation of the word ``gay'' to provide a non-perjorative alternative to the likes of ``queer'' or ``fruit'' or ``faggot''.

Many of the articles at [a][s] are merely intended to provide disinterested perspective on difficult topics. As we continue -- with our growing army of readers and contributors -- we can build on those foundations, grow our understanding of furry, and learn more about our community.
