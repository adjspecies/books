\articlehead{My Fursona is a Mole}{Klisoura}{2011}

Yeah, you probably haven't seen it. It's pretty underground.

I first began to suspect a furry:hipster overlap in the dealer's den at Furry Weekend Atlanta, when I observed that the ratio of hat-wearing men was precipitously high. Not ballcaps, mind you — fedoras, flat caps, bowlers, and other examples of the sort of headwear that one would expect to find less in Atlanta than in, say, 1954.

If hats aren't your thing (and how do you fit your ears through them, anyway?) you may defer instead to the Skinny Jean Quotient, which is also elevated. If anybody asks why you're staring at their pants, just tell them it's for research. Nobody wants to stand in the way of science.

As it happens, this helps explain a lingering geographic dilemma I've had. If you take a bunch of furries and group them by state, you can create a sort of density map: what percentage of furries live in any given state compared to what percentage of Americans in general reside there.

When you do this, you don't find too many anomalies. Furries are underrepresented in New York, possibly because, let's face it, most of us can't afford to live there. And in general it's what you'd expect: we're slightly less common in the American South; more common on the west coast. Standing out as islands, as compared to their surrounding states: the Pacific Northwest, Michigan, and Colorado.

What's the common thread?

By instinct, you want to look for furries in high tech density areas, because the basic idea that ``furry=geek'' is pretty well established. But only 8\% of furries work in technology fields; a majority, 60\%, are students of some stripe or another. This latter angle bids I point out that these islands are, for example, also where you can score high-quality pot. But I'm sure furries know nothing of that (certainly I don't; I don't like smoking, and I can't eat brownies because chocolate is poisonous for dogs).

Anyway, when seen through a hipster lens, the inclusion of places like Portland, Seattle, Denver, and Ann Arbor suddenly fall into place. And this helps to explain the hat-wearing. It also helps to explain the results of a microsurvey I put together a few weeks back. I asked several hundred people 32 questions on their personal beliefs and behaviors, and I plugged this into a sinister machine of my own devising, the Behavioral and Attitudinal Tabulation, Mapping, and Analysis Navigator.

I asked BATMAN for ``two-box'' responses: when it tells me a general skew, it's because a given respondent either ``agreed'' or ``strongly agreed'' (conversely: ``disagreed'' or ``strongly disagreed''), leaving out the middle parts and focusing on the extremes. The computational engine whirred and thunked, and then it told me this:

\begin{itemize}
  \item 80\% of furries say that trying new things is fun and interesting
  \item 36\% say they're often the first person in their group of friends to try something new
  \item 29\% say their friends look to them for advice on music, movies, games and so forth
\end{itemize}

BATMAN pointed out, in its surly fashion, that only 10\% of furries say they're ahead of the curve where pop culture is concerned. But that's not surprising, actually: 44\% describe mass media as being too ``lowest common denominator'' for them and 41\% describe corporations and their products as ``rather soulless.'' That probably explains why 55\% of furries agree or strongly agree that they'd rather patronise a small business.

These figures are all positively correlated with each other: the more likely you are to be asked for advice by your friends, the more likely you are to reject mainstream pop culture and the more likely you are to gravitate to small businesses. They're also related to the creative spark: 58\% of furries say creativity is one of their strongest assets, and 37\% say they'd rather make something than buy it.

Beyond the numbers, this shouldn't really be surprising to anyone who has spent much time around the fandom. We trade heavily in social currency — who you know and how well you know them. Listen in at the discussions at your next convention:

``Oh my god, who did that drawing?''
``Is that conbadge one of\ldots?''
``Have you read\ldots?''
``This is the new work by\ldots''
``I'm getting a commission from\ldots''

It's all about the names. And since nobody is, let's be honest here, really going to break out of the furry fandom, celebrity here has to be milked for what it's worth. So there's a fair degree of bandwagoning, as well, and you can get props for picking up a ‘famous' person's work at the auction just as easily as by discovering an up and coming artist on FA.

I don't think it's a particularly mysterious phenomenon. Hipsterism tends to arise in bohemian cultures where monetary capital is undervalued (either because everybody has money or nobody does). We're certainly bohemian — on the fringes of social acceptability, wildly creative, anti-establishment, consuming mass media and pop culture only as far as it lets us repurpose it\ldots

And, of course, money has no value in the fandom because we're all digital here. Physical possessions and the means to acquire them are, more or less, completely irrelevant. As long as you have enough money to pay your ISP, you can plug into the fandom. So establishing your credibility has to rely on something else, and social capital steps in to fill that gap.

Some of this we can acquire by dint of our own creativity — those of us who are skilled at drawing, writing, music-making, fursuiting, roleplaying, or any other audience-focused activity can trade our abilities there for recognition and status. And if we can't make things ourselves, we can know people who make things, and serve as a proxy to their own works: being the first person to share a new picture or story is the next best thing to having written it yourself.

Every meeting of furries I have ever been party to inevitably involves some modicum of gossip and discussion, frequently about those people whose talents we respect (or envy) and whose work we enjoy. And gossip, too, is essentially hipsterish: we prove how ``in the know'' we are by being the first to a scoop (or, if not the first, by having the most information!). It's the common ground, for when novelty-seeking iconoclasts band together.

So we have attracted some of the trappings of what, ironically, I would have to call mainstream hipsterdom: the self-referential humor, the love of memes, the unorthodox fashion. To this we have added our own spin: I joke about having a mole for a furry avatar, but I've seen species propagate from a single point — somebody cool decided they were going to be something, and a bunch of people jumped right on. And artists acquire the same fetishistic attachment here as they do in any Seattle enclave.

But before you all try and close the circle by Rule 34ing Hipster Kitty, let me suggest that it's not such a terrible thing. It's who we are, and in a sense it's what makes us unique: a shared sense of identity, a shared love of the new, the interesting, the exciting, the different, the crazy, the creative, the passionate. You could do worse than that, and if it helps you find a great new artist or two, your life's all the richer.

Besides, at least PBR is cheap.
