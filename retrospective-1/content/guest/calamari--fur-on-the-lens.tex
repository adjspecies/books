\articlehead{Fur on the Lens}{Calamari}{2012}

As a reader of this article, you may or may not have seen a recent mini-documentary on the furry fandom, filmed by National Geographic. Although I'm not a subscriber to this fine organisation, I've read many of their articles. Time and time again they produce splendid pieces on interesting aspects of history, the environment, and culture. So my first thought upon hearing that they'd chosen furries to be the subject of one of their small documentaries, was confusion. Why had they let their standards slip this time?

Don't get me wrong. I love being a part of this fandom. Despite the various oddballs and undesirables, there are a number of absolutely fantastic people within this little group. The artwork can be fantastic, and the process of thinking up a character is more fun than you'd like to admit. But the reason that the public has never properly welcomed us with open arms, attacked us with pitchforks or even acknowledged us is because we aren't all that interesting. Apart from the fact that we all like anthropomorphic animals in some way (and that we maybe have a tendency to be a bit nerdy) we aren't one collective body that thinks and does the same thing.

If you watch another documentary or read another article from National Geographic, you'll know what I mean. A well written piece about gang culture in Peru has more going for it than a bunch of nerds that dress up like anthropomorphic animals. Because that's really what we are.

When I first saw this mini-documentary, it was posted on a forum for furs from the United Kingdom. I was at a friend's house at the time, who also happens to be a fur. I pointed at the screen with a grin, and she sighed. While the documentary was unintentionally entertaining, it was also extremely predictable and more than slightly embarrassing.

The main problem with these films is that they all seem to have a similar layout. First we are introduced to an individual, who is supposedly usually `shy and reserved' but is `happy and playful' when it comes to wearing a fursuit or roleplaying as their character. I always find this part rather annoying. If these people want to portray us as likeable, interesting individuals, they aren't doing well by describing themselves as semi-autistic.

As someone who is fairly heavily involved in fandom, I don't need to take on an alternative persona to socialize with people. Most of us are very capable of doing it ourselves. I mean no malice towards people who have difficulty interacting with others, but it does seem that we are too often portrayed as shoe-gazers that need to pretend to be an animal in order to cope with reality.

And so we come to, wait for it, deep breath everybody, the Sexual Side of the fandom. In a number of furry related documentaries, everyone always quips without hesitation, ``There is a porn aspect, but it isn't really that strong!''

This was interesting coming from people like `The Ranting Gryphon', who has a whole skit dedicated to roleplaying online sex with furry characters. When we talk about the sexual side of the fandom, it is nothing less than stupid to deny that porn is not a big part of furry culture.

I have no idea if FurAffinity is the most used furry site. I do know, however, that it is very much up there in terms of the number of furries that use it. Refreshing the front page with the filter off, it is very rare to not see a single piece of `adult' artwork on the front page. Doing a Google search of furry with mature content doesn't give a particularly clean result, either. I'm not going to deny that there are people in the fandom not interested in the sexual element. I do feel sorry for them. But even they acknowledge the fandom's `darker side'.

All in all, I'd prefer it if there were no more documentaries, films, or TV programmes based on the furry fandom. With that said, if someone ever shoves a camera in my face, asking if I'd like to give my opinion on the fandom, I'll put down my drink, I'll sigh, and I'll look straight into the lens.

``We're all a bunch of porn obsessed freaks''.
