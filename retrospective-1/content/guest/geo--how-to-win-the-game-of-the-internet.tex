\articlehead{How to Win the Game of the Internet}{Geo Holms}{2012}

Life is too short to worry about the Internet.

Don't get me wrong, the Internet is awesome. All my best friends are from the Internet. The Internet helped prompt me into writing and drawing and creative adventures. The Internet is just fantastic.

However, the Internet is also huge and amorphous and weird and can sometimes cause people to do dumb things. The Internet is people throwing thoughts into the void. Thanks to things like Twitter and Facebook and e-mail and IM, some of those thoughts are countered with more thoughts, and so forth ad infinitum. That social network of thoughts and counter-thoughts is the foundation of the Internet.

Thoughts can be written ideas, pieces of art, recorded music, a link to an article or cute puppy GIF. Counter-thoughts (CTs), the reaction to thoughts, which can be in the form of page views, or comments, or replies, or retweets, or follows. For some people, if they do not get those CTs to their thoughts, it means they are failing at the Internet and they must rectify it by any means necessary to win the Game of the Internet.

Spoilers: The Game of the Internet does not exist. All those followers and page views and comments you collect are not points towards anything but some archaic scoreboard in your imagination. If you're going to use your imagination for something, write a story about a cute bunny with an ax.

Yes, getting CTs from people is nice. I've been there.

A long time ago, before I was a northwoods raccoon in Wyoming,  I lived in the middle of nowhere in the North, no friends in Real Life, and the Internet was my connection the outside world. There so many new and shiny things there. I loved a movie, there was a group of people on the Internet who also liked that movie. I enjoyed a book series, there was a group of people on the Internet I could discuss it with (you peeps really need to read the Redwall series if you haven't already). The Internet, a magical place at the time, gave me a portal to people who were like me.

I thought the Internet brought justification to my existence.  I drew art which I thought was Amazing. I wrote LiveJournal entries that I thought were Poignant. I posted random comments on forums that were Awesome (because random is funny, right?).

Spoilers: I was a dumb kid who didn't know how to socialize.

This went on for a few years. I refreshed every few minutes to check my page views on deviantArt.  I wrote LiveJournal entries and hoped for comments. I wrote dozens of forum comments and hoped for replies. This went on until I wrote a entry about how I never got any comments and poor-pitiful-me. Then, one of my LiveJournal followers (have no idea why they were following me, come to think of it), came out called me a ``comment whore''.

This was the bucket of cold water a dumb kid needed at the time. Some might have brushed this off and thought it was someone mean just saying something mean because they were a meanie head. I took it to heart, and reevaluated my stance on the Internet. I looked at how I was acting on the Internet and realized not only was a jerk, but I was miserable too. I wasn't even being myself. I was being a creature who was feeding off counter-thoughts, being a self-entitled jerk to all I interacted with.

That was a turning point in my life on the Internet. I realized that counter-thoughts meant nothing in the grand scheme of life. Comments and replies could be appreciated, but they were not necessary for Internet happiness. As long as I could be myself online and put things online I was happy with, that was enough. I look back at my past contributions and think how funny it was that I drew a killer-burrito, rather than thinking how few comments or views it got (I totally should commission a Mexican Wolf with a killer-burrito at some point).

These days I play a raccoon on the Internet, and though I still have dumb raccoon moments, I try to be a good fluffbutt ringtail. Though, as I've grown older, and entered the furry fandom, I still struggle sometimes with premise of Internet popularity.

Some people think the goal of the Game of the Internet is to be ``popular'' and in the furry fandom, some percieve the way to be ``popular'' is to make ``popular'' friends. Or, as they are known in the furry fandom, ``popufurs'' (a phrase that annoys me to no end).

Admittedly, I've sent things to people that I regret. I've sent things to people I admire that I really regret. I understand talking to people you admire is tricky. I still put my footpaw in my mouth every single time I go to a furry convention dealer's den. Just remember that supposed popufurs are just Normal Fuzzies. They aren't different than anyone else except they are perceived to be more talented and therefore more popular than you. Most importantly, being friends with them does not make you popular.

It won't change your life to be friends with a supposed popufur. And you're not going to be friends if you try to force yourself into being friends with every single popufur you want to be friends with. Friendships are something that happens when they happen. You can't predict a good friendship. I have met supposedly popufur people who are just really nice people. I have met some who are jerks. I've become friends with a very select few furries, and always ALWAYS based on if they are a good person rather than if they are a popufur. Honestly, you'd be happier interacting with a good-natured wolverine who's obscure in the fandom than forcing yourself to be friend with a jerky platypus because they are a popufur (apologies to any platypui out there).

The friends you have don't make you popular. You, make you popular. You do that by being yourself, doing something you love, keep going forward, and not caring about being popular. Simple as that.

Popufur or not, just treat furries as furries. Yes, there are jerks on the Internet, but that doesn't give you the right to be a jerk yourself. Just because there are untold distances and a cute wolf avatar between you and the Internet denizens does not give you the right to be a jerk. You may be a jerk and not know it, like I was, but you can still change. Be willing to take criticism. Sometimes it's worth taking into account. I'm not saying take someone seriously who calls you a ``stupid stupidhead''. I am saying evaluate if there may be some credence to some claims.

Worrying about if certain people respond to certain things in certain ways does nothing. If someone responds: great. But really doesn't matter in the grand sceme of things. The only thing that's gonna happen if you berate someone for not responding is drama. And with drama comes a reputation. And I don't think anyone wants a reputation flavored by drama.

If someone on the Internet does not respond to you it means a few things: either A) they have life and cannot respond to everything B) they can't think of anything to respond to it with or C) they are avoiding you because you kept berating them for not responding to you because of A or B. I like to imagine they are doing D) having a grand adventure away from the computer. (It's hard to get angry at someone not responding because they are trying to steal cookies from Santa's Workshop with the help of a rogue reindeer.)

There are not many people out there who don't respond with the express purpose of wrecking your life.

There is no Game of the Internet, getting counter-thoughts does not justify your existence; just be yourself and be nice, and you might meet some nice people in the process. If you can realize that the Internet doesn't matter that much, you might just be a little happier.
